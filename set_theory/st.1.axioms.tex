\section{Axioms}

\begin{definition}[\cindex{set}]
    A set is a collection of objects.
\end{definition}

In ZFC, the only noun is $\emptyset$, so there is no atoms. And the only vocabulary is $\in$.

Set theory measure the ability to regard any collection of objects as a single element (set).

\begin{axiom}[\cindex{Axiom of Extensionality}]\label{axiomofextensionality}
    If $X$ and $Y$ have the same elements, then $X=Y$.
    \begin{equation}
        \forall u (u \in X \mleftrightarrow u \in Y ) \rightarrow X = Y
    \end{equation}
\end{axiom}

It means that a set is determined by its elements.

\begin{axiom}[\cindex{Axiom of Pairing}]
    For any $a$ and $b$ there exists a set $\set{a,b}$ that contains exactly $a$ and $b$.
    \begin{equation}
        \forall a \forall b \exists c \forall x (x \in c \mleftrightarrow x = a \vee x = b )
    \end{equation}
\end{axiom}

By \theoref{axiomofextensionality} this pair is unique. A \cindex{singleton} $\set{a}$ is the set $\set{a} = \set{a, a}$. An \cindex{ordered pair} $(a,b)$ is the set 
    \begin{equation}
        (a,b) = \set{\set{a}, \set{a,b}}
    \end{equation}



\begin{axiom}[\cindex{Axiom Schema of Seperation}]
    If $P$ is a property with parameter $p$, then for any set $X$ and $p$ there exists a set $Y = \set{u \in X: P(u,p)}$ that contains all those $u \in X$ that have property $P$. $P$ must be defined in first order formula.
    \begin{equation}
        \forall X \forall p \exists Y \forall u \mleft(u \in Y \mleftrightarrow u \in X \wedge \varphi(u, p) \mright)
    \end{equation}    
\end{axiom}

This axiom defines $\cap$. For example $A \cap B$ could be defined as $\set{x \in A: x \in B}$.


If we define class $C = \varphi(u, p)$, then $\forall X \exists Y (C \wedge X )= Y$, so a subclass of a set is a set. The empty class \cindex{$\emptyset$} $ = \set{u: u \neq u} $ is an \cindex{empty set}. $\emptyset \neq \set{\emptyset}$.

\begin{axiom}[\cindex{Axiom of Union}]
    For any $X$ there exists a set $Y = \cup X$, the union of all elements of $X$.
    \begin{equation}
        \forall X \exists Y \forall u \mleft(u \in Y \mleftrightarrow \exists z (z \in X \wedge u \in z ) \mright)
    \end{equation}
    
    It is a generalized version of two set union. There is \emph{an assumption that the elements in $A$ are sets}.
\end{axiom}

\begin{axiom}[\cindex{Axiom of Power Set}]
    For any $X$ there exists a set $Y = \powerset{X}$, the set of all subset of $X$.
    \begin{equation}
        \forall X \exists Y \forall u (u \in Y \mleftrightarrow u \subset X )
    \end{equation}
\end{axiom}

\begin{axiom}[\cindex{Axiom of Infinity}]
    There exists an infinite set.
    \begin{equation}
        \exists S \mleft( \emptyset \in S \wedge (\forall x \in S ) x \cup \set{x} \in S \mright)
    \end{equation}
    A set $S$ with above property is called \cindex{inductive}.
\end{axiom}

\begin{axiom}[\cindex{Axiom Schema of Replacement}]
    If a class $F$ is a function, then for any $X$ there exists a set $Y = F(X) = \set{F(x): x \in X}$.
    \begin{equation}
        \forall x \forall y \forall z \mleft( \varphi(x,y,p) \wedge \varphi(x,z,p) \rightarrow y = z \mright) \rightarrow \forall X \exists Y \forall y \mleft( y \in Y \rightarrow (\exists x \in X ) \varphi(x,y,p) \mright)
    \end{equation}
    So if a class $F$ is a function and $\domain{f}$ is a set, then $\range{f}$ is a set.
\end{axiom}

\begin{axiom}[\cindex{Axiom of Regularity}]\label{axiomofregularation}
    Every nonempty set has an $\in$-minimal element.
\end{axiom}

\begin{axiom}[\cindex{Axiom of Choice}]\label{axiomofchoice}
    Every family of nonempty set has a choice function.
\end{axiom}

The \theoref{axiomofextensionality} to \theoref{axiomofregularation} is the \cindex{Zermelo-Fraenkel} axiomatic set theory \cindex{ZF}. \cindex{ZFC} denote the ZF + \cindex{AC}, the axiom of choice.

Now we have three ways to create new set from existing sets:
\begin{enumerate}
    \item take the pairing.
    \item take the union.
    \item take the power set.
\end{enumerate}

\begin{definition}[\cindex{product}]
    The product of $X$ and $Y$ is the set of all pairs $(x,y)$ that 
    \begin{equation}
        X \times Y = \set{(x,y): x \in X \wedge y \in Y }
    \end{equation}
    
    The product exists because $X \times Y \subset \powerset{ \powerset{X \cup Y}}$. The $n$-ary relation $R$ is a set of $n$-tuples.
\end{definition}

\begin{definition}[\cindex{function}]
    A binary relation $f$ is a function if $(x,y) \in f$ and $(x, z) \in f$ implies $y = z$. For a function $f$ from $X$ to $Y$ $f : X \rightarrow Y$, if $Y = \range{f}$, $f$ is \cindex{onto}. If $f(x) = f(y)$ implies $x=y$, $f$ is \cindex{one-to-one}. The \cindex{inverse image} $f_{-1} (Y) = \set{x: f(x) \in Y}$. If $f$ is one-to-one, then the \cindex{inverse} is $f^{-1} (y) = x$.
\end{definition}

\begin{definition}
    The \cindex{restriction} of a function $f$ to a set $X$ is:
    \begin{equation}
        f\restriction_X = \set{(x,y) \in f : x \in X}
    \end{equation}
\end{definition}


\begin{definition}[\cindex{class}]
    if $\varphi(x, p_1, \dots, p_n)$ is a formula, then $C = \set{x: \varphi(x, p_1, \dots, p_n)}$ is a class. So a formula defines a class.
\end{definition}

\begin{theorem}[\cindex{Russell's Paradox}]
    There is no set whose elements are all those sets that are not member of themselves: $S = \set{X : X \notin X}$. So the set of all set does not exist.
\end{theorem}


\begin{definition}[universe]
    The \cindex{universe} is the class of all sets: $V = \set{x: x = x}$.
\end{definition}

\begin{definition}
    A class that is not a set is a \cindex{proper class}.
\end{definition}


To memorize:
\begin{enumerate}
    \item A formula defines a class.
    \item Proper class is not a set.
    \item The universe is the class of all sets.
\end{enumerate}






