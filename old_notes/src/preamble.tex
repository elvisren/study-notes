\documentclass[reqno,16pt]{book}
\usepackage{amssymb}
\usepackage{latexsym}


% algorithm
\usepackage{algpseudocode}
\usepackage[section]{algorithm}
\usepackage{algorithmicx}
\usepackage[colorlinks,linkcolor=red,anchorcolor=blue,citecolor=green]{hyperref}
\usepackage{tikz}
\usepackage{multirow}
\usepackage{float}

% math theorem, lemma, proof
\usepackage{amsthm}
%\theoremstyle{definition}
%\newtheorem{definition}{Definition}
%\newtheorem{theorem}{Theorem}

\usepackage{thmtools}
\declaretheorem{theorem}
\declaretheorem{definition}
\declaretheorem{example}
\declaretheorem{axiom}



% for mathematical integratoin
\usepackage{commath}


% for drawing beautiful math commutative diagram
% note:
%   it treat nodes as a matrix and use "&" to separate row and "\\" for line.
%   "'" change the label from above arrow to below arrow.
%   "l", "r" and "d" means move among matrix nodes.
\usepackage{tikz-cd}


% create index
\usepackage{mathtools}
\usepackage{makeidx}
\makeindex


\usepackage{listings}



% set font

\usepackage[T1]{fontenc}
\usepackage{fontspec}
\usepackage[theoremfont,trueslanted,largesc,tighter,p,osf]{newpxtext}
\usepackage[T1]{fontenc}
\usepackage{textcomp}
\usepackage[varqu,varl]{inconsolata}
\usepackage{amsmath} % must be loaded before amsthm
\usepackage{amsthm} % must be loaded before newpxmath
%\usepackage[vvarbb]{newpxmath}


% no space in itemize
\usepackage{enumitem}
\setenumerate{itemsep=0pt,partopsep=0pt,parsep=\parskip,topsep=2pt}
\setitemize{itemsep=0pt,partopsep=0pt,parsep=\parskip,topsep=2pt}
\setdescription{itemsep=0pt,partopsep=0pt,parsep=\parskip,topsep=2pt}


% set line space
\usepackage{setspace}



% set page size
\usepackage{geometry}
%\geometry{a4paper}
\geometry{a4paper,top=2cm,bottom=2cm,left=2.5cm,right=2cm}


% set programming code highlight
% \usepackage[chapter]{minted}


% setup bibtex
\usepackage{cite}


% include eps file
\usepackage{epsfig}



% set toc and section level
\setcounter{secnumdepth}{7}
\setcounter{tocdepth}{7}



% customized commands
%\newcommand\cindex[1]{\underline{#1}\index{#1}}
%\newcommand\cindex[1]{\textcolor{blue}{\textbf{#1}}\index{#1}}
\newcommand\cindex[1]{\textcolor{blue}{#1}\index{#1}}


\newcommand\mathhilight[1]{\mathop{\bf #1\/}}


% set theory
\newcommand\powerset[1]{\mathcal{P}\left( #1 \right)}
\newcommand\range[1]{\mathbf{ran}(#1)}
\newcommand\domain[1]{\mathbf{dom}(#1)}
\newcommand\allordinals[0]{\mathbf{Ord}}
\newcommand\transfinitesequence[1]{\left< #1 \right>}
\newcommand\supremum[1]{\mathbf{sup} \set{#1}}
\newcommand\infimum[1]{\mathbf{inf} \set{#1}}
\newcommand\maximum[1]{\mathbf{max} \set{#1}}
\newcommand\minimum[1]{\mathbf{min} \set{#1}}


% topology
\newcommand\closure[1]{\overline{#1}}
\newcommand\real[0]{\mathbb{R}}


%linear algebra
\newcommand\nullspace[1]{\mathcal{N}(#1)}
\newcommand\rangespace[1]{\mathcal{R}(#1)}
\newcommand\absolutevalue[1]{\abs{#1}}
\newcommand\determinate[1]{\absolutevalue{#1}}
\newcommand\determinatetext[1]{\displaystyle \mathbf{det} ~\displaystyle #1}
\newcommand\coordinate[1]{\sbr{#1}}
\newcommand\projection[2]{\mathbf{proj}_{#2} #1}
\newcommand\rowvector[1]{\left[ \displaystyle #1 \right]}


\newcommand\dimension[1]{\displaystyle \mathbf{dim}\left( #1 \right)}
\newcommand\vectorspan[1]{\displaystyle \mathbf{span}\left( #1 \right)}
\newcommand\rank[1]{\displaystyle \mathbf{rank}( #1 )}
\newcommand\innerproduct[2]{\left\langle \displaystyle #1, #2 \right\rangle}
\newcommand\trace[1]{\displaystyle \mathbf{tr}( #1 )}


\newcommand\adjugate[1]{\displaystyle \mathbf{adj} ~\displaystyle #1 }
\newcommand\cofactor[1]{\displaystyle \mathbf{cof} ~\displaystyle #1 }

\newcommand\columnvector[1]{\boldsymbol{#1}}


%probability
\newcommand\probability[1]{\mathop{\bf P\/}\displaystyle \left\{#1\right\}}
\newcommand\expect[1]{\mathop{\bf E\/}\displaystyle  \left[ #1 \right]}
\newcommand\variance[1]{\mathop{\bf Var\/}\displaystyle \left[ #1 \right]}
\newcommand\covariance[2]{\mathop{\bf Cov\/}\displaystyle \left(#1,#2 \right)}


% machine learning
\newcommand\subscription[2]{\boldsymbol{#1}^{(#2)}}

