\section{Schema Normalization}



\begin{definition}[Functional Dependency (\cindex{FD})]
    Consider a relation $R$, and let $X \subseteq R$ and $Y \subseteq R$. The \cindex{functional dependency} $X \rightarrow Y$ holds on schema $R$ if for any pairs of tuple $t_1$ and $t_2$ in $R$, $t_1 [X] = t_2 [Y] \Rightarrow t_1 [Y] = t_2[Y]$.
    
    It is called functional dependency because it is a function that given a set of input attributes, it will produce a unique value.
\end{definition}

For a set of functional dependency $F$, the \cindex{closure of functional dependencies} is called $F^{+}$. The \cindex{closure of attribute sets} is called $A^{+}$.



\begin{theorem}
    \cindex{Armstrong's axioms}:
    \begin{itemize}
        \item Reflexivity property: if $X$ is a set of property and $Y \subseteq X$, then $X \rightarrow Y$.
        \item Augmentation property: if $X \rightarrow Y$ and $\gamma$ is a set of attributes, then $\gamma X \rightarrow \gamma Y$.
        \item Transitivity property: if $X \rightarrow Y$ and $Y \rightarrow Z$, then $X \rightarrow Z$.
    \end{itemize}
    
    other properties are:
    \begin{itemize}
        \item Union property: if $X \rightarrow Y$ and $X \rightarrow Z$, then $X \rightarrow YZ$.
        \item Decompisition property: if $X \rightarrow YZ$, then $X \rightarrow Y$ and $X \rightarrow Z$.
        \item Pseudotransitivity property: if $X \rightarrow Y$ and $\gamma Y \rightarrow Z$, then $X \gamma \rightarrow Z$.
    \end{itemize}
\end{theorem}

\begin{definition}[Decompoisition]
    Suppose a relation $R$ consists of a set of attributes $A(R) = \set{A_1, A_2, \dots, A_n}$. A \cindex{decomposition} of $R$ is a set of relations $\set{R_1, \dots, R_m}$ such that both of the following holds:
    \begin{itemize}
        \item $\forall i, A(R_i) \subset A(R)$
        \item $A(R_1) \cup A(R_2) \cup \dots \cup A(R_m) = A(R)$
    \end{itemize}
\end{definition}

\begin{definition}[basis]
    For a set of functional dependencies $S$, any set that is equivalent to $S$ is called \cindex{basis}. It is called \cindex{minimal basis} if the basis $B$ has the following conditions:
    \begin{enumerate}
        \item All $FD$s in $B$ have singleton right sides.
        \item If any $FD$ is removed from $B$, the result is no longer a basis.
        \item If any attribute is removed from any of the left side of $FD$, the result is no longer a basis.
    \end{enumerate}
\end{definition}


It is better that the decomposition of a relation satisfy the following:
\begin{itemize}
    \item \cindex{lossless join}: the same relation could be reconstructed by joining.
    \item \cindex{dependency preservation}: all functional dependencies are preserved.
\end{itemize}

\begin{theorem}
    A decomposition of $R$ into $\set{R_1, R_2}$ is \cindex{lossless join} if and only if $A(R_1) \cap A(R_2) \rightarrow A(R_1)$ or $A(R_1) \cap A(R_2) \rightarrow A(R_2)$ in $F^{+}$.
\end{theorem}


the lossless join could be verified using \cindex{chase} algorithm \cite{Aho1979}. Suppose we have a relation $R(A,B,C,D)$ and we decompose it into three relations $S_1 = \set{A,D}$, $S_2 = \set{A,C}$ and $S_3 = \set{B,C,D}$. We create a table below:

\begin{center} 
   \begin{tabular}{|c|c|c|c|}
  \hline
  A & B & C & D \\
  \hline
  $a$ & $b_1$ & $c_1$ & $d$ \\
  \hline
  $a$ & $b_2$ & $c$ & $d_2$ \\
  \hline
  $a_3$ & $b$ & $c_1$ & $d$ \\
  \hline
\end{tabular} 
\end{center}

The logic is this. For first relation  $S_1 = \set{A,D}$, because $B$ and $C$ did not appear in $S_1$, they will have subscription of $1$, such as $b_1$ and $c_1$. $a$ and $b$ appears in $S_1$ so they will have no subscription. Then will calculate whether the given FD $A \rightarrow B$, $B \rightarrow C$ and $CD \rightarrow A$ is a lossless join.


For the first $FD$ $A \rightarrow B$, we will find rows that contains $a$ without subscription and make their $b$ the same. The rule of making the same $b$ is to use the minimal subscription, and use the one without subscription if there is any. So the result after applying $A \rightarrow B$ is:

\begin{center} 
   \begin{tabular}{|c|c|c|c|}
  \hline
  A & B & C & D \\
  \hline
  $a$ & $b_1$ & $c_1$ & $d$ \\
  \hline
  $a$ & $\textcolor{blue}{b_1}$ & $c$ & $d_2$ \\
  \hline
  $a_3$ & $b$ & $c$ & $d$ \\
  \hline
\end{tabular} 
\end{center}

The result of applying $B \rightarrow C$ is:

\begin{center} 
   \begin{tabular}{|c|c|c|c|}
  \hline
  A & B & C & D \\
  \hline
  $a$ & $b_1$ & $\textcolor{blue}{c}$ & $d$ \\
  \hline
  $a$ & $b_1$ & $c$ & $d_2$ \\
  \hline
  $a_3$ & $b$ & $c$ & $d$ \\
  \hline
\end{tabular} 
\end{center}

The result of applying $CD \rightarrow A$ is:
\begin{center} 
   \begin{tabular}{|c|c|c|c|}
  \hline
  A & B & C & D \\
  \hline
  $a$ & $b_1$ & $c$ & $d$ \\
  \hline
  $a$ & $b_1$ & $c$ & $d_2$ \\
  \hline
  $\textcolor{blue}{a}$ & $b$ & $c$ & $d$ \\
  \hline
\end{tabular} 
\end{center}

If there is a row without any subscription after applying all $FD$, the decomposition is a loseless join.


\begin{theorem}
    A decomposition of $R$ into $\set{R_i}$ is \cindex{dependency preserving} if and only if $\mleft (\bigcup_i F_i \mright)^{+} = F^{+}$
\end{theorem}


\begin{definition}[Partial Dependency]
    The partial dependency $X \rightarrow Y$ holds if there is $Z \subset X$ such that $Z \rightarrow Y$. In this case $Y$ is \cindex{partially dependent} on $X$.
\end{definition}

\begin{definition}[Key]
    A \cindex{key} is a superkey for which no proper subset is a super key, i.e., a minimal superkey. If a relation has more than one key, one of them is designated as \cindex{primary key}.
\end{definition}

\begin{definition}[Superkey]
    A set of attributes is called \cindex{superkey} if they contain a key.
\end{definition}

\begin{definition}[Multi-Valued Dependency]
    A relational schema is \cindex{multi-valued dependency} \cite{Fagin1977} if for any tuple $(a,b,c)$ and $(a,d,e)$ in $R$, there exists tuple $(a,b,e)$ and $(a,d,b)$. If $a$ belongs to attribute set $X$, $b$ and $d$ belong to attribute set $Y$, $c$ and $e$ belongs to $R - X - Y$, the dependency is marked as $X \twoheadrightarrow Y$.
\end{definition}


\begin{theorem}
    Some properties of multi-valued dependency \cite{Beeri1977}:
    \begin{itemize}
        \item if $X \twoheadrightarrow Y$, then  $X \twoheadrightarrow \mleft(R - (X \cup Y) \mright)$.
        \item if $X \twoheadrightarrow Y$ and $Z \subseteq W$, then  $WX \twoheadrightarrow YZ$.
        \item if $X \twoheadrightarrow Y$ and  $Y \twoheadrightarrow Z$, then  $X \twoheadrightarrow (Z - Y)$.
        \item if $X \rightarrow Y$, then $X \twoheadrightarrow Y$.
        \item if $X \twoheadrightarrow YZ$, then it is not guaranteed that $X \twoheadrightarrow Y$.
    \end{itemize}    
\end{theorem}


\begin{definition}[Join Dependency]
    A relational schema is \cindex{join dependency} $JD(R_1, R_2, \dots, R_n)$ if for every legal relation $r(R)$, we have $\mleft(\pi_{R_1} (r), \pi_{R_2}(r), \dots, \pi_{R_n}(r) \mright) = r$.
\end{definition}

Multi-valued dependency is a special case of join dependency with $n=2$.

\begin{definition}[1NF]
    A relational schema $R$ is in \cindex{1NF} $\iff$ all the domains of all attributes in $R$ are atomic.
\end{definition}

\begin{definition}[2NF]
    A relational schema $R$ is in \cindex{2NF} if each attribute $A$ of $R$ satisfies one of the following property:
    \begin{enumerate}
        \item $A$ is part of a key.
        \item $A$ is not partially dependent on a key.
    \end{enumerate}
\end{definition}

\begin{definition}[3NF]
    A relational schema $R$ is in \cindex{3NF} \cite{Codd1983a} if for every non-trivial functional dependency $X \rightarrow A$, one of the following is true:
    \begin{enumerate}
        \item $X$ is a superkey of $R$.
        \item $A$ is part of some key for $R$. (not necessarily the same key)
    \end{enumerate}
\end{definition}

\begin{definition}[Boyce-Code Normal Form (BCNF)]
    A relational schema $R$ is in \cindex{BCNF} \cite{Codd1972} if for every non-trivial functional dependency $X \rightarrow A$, $X$ is a superkey of $R$.
\end{definition}

\begin{definition}[4NF]
    A relational schema $R$ is in \cindex{4NF} \cite{Fagin1977} if for every non-trivial multi-valued dependency $X \twoheadrightarrow A$, $X$ is a superkey of $R$.
\end{definition}


\begin{definition}[5NF]
    A relational schema $R$ is in \cindex{5NF} \cite{Fagin1979} if for every non-trivial join dependency $JD(R_1, R_2, \dots, R_n)$, $R_i$ is a superkey of $R$.
\end{definition}


\begin{definition}[6NF]
    A relational schema $R$ is in \cindex{6NF} \cite{Date2003} if there is no non-trivial join dependencies.
\end{definition}


A relational schema may be converted to 2NF, 3NF, BCNF, 4NF, 5NF using a recursive algorithm: if $X \rightarrow Y$ violates the normal form, then split the $R$ into $R - Y$ and $XY$.


The relationship among all these normal forms are: $6NF \subset 5NF \subset 4NF \subset BCNF \subset 3NF \subset 2NF \subset 1NF $.

A summary of all normal forms:
\begin{center} 
   \begin{tabular}{|c|l|}
  \hline
  Normal Form & Details \\
  \hline
  1NF & domain should be atomic \\
  \hline
  2NF & no non-prime attribute is functionally dependent on a proper subset of any key \\
  \hline
  3NF & Every non-prime attribute is non-transitively dependent on every key \\
  \hline
  BCNF & every non-trivial functional dependency is a dependency on a superkey\\
  \hline
  4NF & every non-trivial multi-valued dependency is a dependency on a super key\\
  \hline
  5NF & every non-trivial join dependency is implied by the superkeys\\
  \hline 
  6NF & non non-trivial join dependency at all\\
  \hline
\end{tabular} 
\end{center}




