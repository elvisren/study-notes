\section{Determinants}

\begin{definition}
	Let $A \in M_{n \times n} (F)$. If $n =1$, let $A=(A_{11})$ and we define $\text{det}(A) = A_{11}$. For $n \geq 2$, $\text{det}(A)$ (or $\determinate{A}$) is defined as
	\begin{equation}
		\determinate{A} = \sum_{j=1}^n (-1)^{i + j} A_{ij} \times \determinate{\tilde{A}_{ij}}
	\end{equation}
	
	where $\tilde{A}_{ij}$ is obtained from $A$ by deleting row $i$ and column $j$. This is called \cindex{Laplace expansion}.
	\qed
\end{definition}


\begin{theorem}
    A function $\delta: M_{n \times n} (F) \rightarrow F$ is the same as $\determinate{A}$ if it satisfies the following 3 properties:
    \begin{enumerate}
        \item It is \cindex{$n$-linear function}: for a scalar $k$, \begin{equation}
        \determinate{\begin{bmatrix}
            a_1\\
            \vdots\\
            u + kv \\
            \vdots\\
            a_n
        \end{bmatrix}} = \determinate{\begin{bmatrix}
            a_1\\
            \vdots\\
            u \\
            \vdots\\
            a_n
        \end{bmatrix}} + k \determinate{\begin{bmatrix}
            a_1\\
            \vdots\\
            v \\
            \vdots\\
            a_n
        \end{bmatrix}}
    \end{equation}
    \item It is \cindex{alternating}: $\delta(A) = 0$ if any two adjacent rows are identical.
    \item $\delta(I) = 1$.
    \end{enumerate}
    The determinate is linear on each row when the remaining rows are held fixed.
    \qed
\end{theorem}



\begin{theorem}
    The effect of elementary row operation on the determinant of a matrix $A$ is:
\begin{enumerate}
    \item interchange any two rows: $\determinate{B} = - \determinate{A}$.
    \item multiply a row: $\determinate{B} = k \determinate{A}$.
    \item add a multiple of a row to another: $\determinate{B} = \determinate{A}$.
\end{enumerate}
\end{theorem}


\begin{theorem}
    If $\rank{A_{n \times n}} < n$, then $\determinate{A} = 0$.
\end{theorem}
\begin{proof}
    If $\rank{A_{n \times n}} < n$, one row is a linear combination of all other rows.
\end{proof}



\begin{theorem}
	\begin{equation}
		\determinate{AB} = \determinate{A} \times \determinate{B}
	\end{equation}
\end{theorem}

\begin{theorem}
	A matrix $A \in M_{n \times n}(F)$ is invertible $\Leftrightarrow$ $\determinate{A} \neq 0$. If it is invertible, $\determinate{A^{-1}} = \dfrac{1}{\determinate{A}}$.
\end{theorem}

\begin{definition}
    The \cindex{cofactor} of $A$ is defined as 
    \begin{equation}
        \cofactor{A}_{ij} = (-1)^{i+j} \determinate{\tilde{A}_{ij}}
    \end{equation}\qed
\end{definition}


If the determinate is calculated using cofactor operation, the performance is $n!$ multiplication. However if it is calculated using elementary row operation, the performance is $\dfrac{n^3 + 2n - 3}{3}$ multiplication.

\begin{definition}
    The \cindex{adjugate} of $A$ is defined as
    \begin{equation}
        \adjugate{A} = (\cofactor{A})^\top
    \end{equation}
\end{definition}

\begin{theorem}
    The inverse of invertible square matrix A is:
    \begin{equation*}
        A^{-1} = \frac{1}{\determinate{A}} \adjugate{A}
    \end{equation*}
\end{theorem}


\begin{theorem}[\cindex{Cramer's Rule}]
	Let $Ax=b$ be a system of $n$ equation with $n$ unknowns. If $\determinate{A} \neq 0$, the system has a unique solution:
	\begin{equation}
		x_k = \frac{\determinate{M_k}}{\determinate{A}}
	\end{equation}
	
	where $M_k$ is a $n\times n$ matrix obtained from $A$ by replacing column $k$ of $A$ by $b$.
\end{theorem}

\begin{proof}
	Let $a_k$ be the $k$th column of $A$ and $X_k$ denote the matrix obtained from replacing the column $k$ of identity matrix $I_n$ by $x$. Then $A X_k = M_k$:
	\begin{equation*}
	\begin{aligned}
        A X_k &= A \begin{bmatrix}
			1 &   &   & x &   \\
			 & 1 &  & x &   \\
			 && \ddots &  \vdots \\
			&&& x & \\
			&&& \vdots & \ddots \\
			&&& x&   & 1 
		\end{bmatrix} \\
		&= \begin{bmatrix}
		    Ae_1, Ae_2, \dots, Ax, \dots, Ae_n
		\end{bmatrix} \\
		& = \begin{bmatrix}
		    a_1, a_2, \dots, b, \dots, a_n
		\end{bmatrix} \\
		&= M_k
		\end{aligned}
    \end{equation*}
	
	
	Evaluate $X_k$ by cofactor expansion along row $k$ produces
	\begin{equation*}
		\determinate{X_k} = x_k \times \determinate{I_{n-1}} = x_k
	\end{equation*}
	
	Hence 
	\begin{equation*}
		\determinate{M_k} = \determinate{A X_k} = \determinate{A} \times \determinate{X_k} = \determinate{A} \times x_k
	\end{equation*}
	
	Therefore
	\begin{equation*}
		x_k = \frac{\determinate{M_k}}{\determinate{A}}
	\end{equation*}
\end{proof}

Note: Cramer's Rule is too slow for real world calculation.

\begin{theorem}
	In geometry, for a square matrix $A \in M_{n\times n}(F)$, $\absolutevalue{\determinatetext{A}}$ is the \cindex{n-dimensional volume} of the parallelepiped having vector $A_{i,\cdot}$ as adjacent sides.
\end{theorem}


