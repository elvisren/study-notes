\section{Diagonalization}


There are two questions for a linear operator $T$:
\begin{enumerate}
    \item Is there an ordered basis $\beta$ that $\coordinate{T}_\beta$ is a diagonal matrix?
    \item If such basis exists, how can it be found?
\end{enumerate}


\subsection{Eigenvalue and Eigenvectors}

\begin{definition}
    A linear operator $T$ on $V$ is \cindex{diagonalizable} if there is an ordered basis $\beta$ of $V$ that $\coordinate{T}_\beta$ is a diagonal matrix. A matrix is \cindex{diagonalizable} if $L_A$ is diagonalizable.


If an operator $T$ is diagonalizable, for $\beta = \{v_i\}$, we have
\begin{equation*}
    T(v_j) = \sum_{i=1}^n D_{ij} v_j = D_{jj} v_j = \lambda_j v_j
\end{equation*}
So to prove a  linear operator $T$ is diagnolizable is to find a basis $\beta = \{ v_i \}$ and $\{ \lambda_j \}$ that $T(v_i) = \lambda_i v_i$.
\qed
\end{definition}

\begin{definition}
    A \emph{non-zero} vector $v \in V$ is called an \cindex{eigenvector} of linear operator $T$ if $\exists \lambda : T(v) = \lambda v$. $\lambda$ is called \cindex{eigenvalue} corresponding to eigenvector $v$. Eigenvector is also called \cindex{characteristic vector}. Eigenvalue is also called \cindex{characteristic value}.

    
    
    A eigenvalue could be $0$, but eigenvector could not be $\vec{0}$. An eigenvector is an invariant subspace of dimension $1$.
\end{definition}


\begin{theorem}
    A linear operator $T$ is diagonalizable if there exists an ordered basis consisting of eigenvectors of $T$.
\end{theorem}

\begin{theorem}
    $\lambda$ is an eigenvalue of $A$ $\iff$ $\determinate{A - \lambda I_n} = 0$.
\end{theorem}

\begin{proof}
    If $\lambda$ is an eigenvalue of $A$, $\exists v \in F^n, v \neq 0$ that $A v = \lambda v$, which is $(A - \lambda I_n)(v)= 0$, which means $A - \lambda I_n$ is not invertible because $v \neq 0$, so $\determinate{A - \lambda I_n} = 0$.
\end{proof}

\begin{theorem}
    Every eigenvalue has at least one eigenvector.    
\end{theorem}
\begin{proof}
    Since $\determinate{A - \lambda I_n} = 0$, $(A - \lambda I_n) x = 0$ is a homogeneous equation with $\dimension{A - \lambda I_n} < n$.
\end{proof}



\begin{definition}
    For $A = \coordinate{T}_\beta$ the polynomial $f_A(t) = \determinate{A - t I_n}$ is called the \cindex{characteristic polynomial} of $A$ and $T$. 
\end{definition}

\begin{theorem}
    For all eigenvalues $\lambda_i$ of $A$, define 
    \begin{equation}
        S_k(A) = \sum_{1\leq j_1 \leq j_2 \leq \dots \leq j_k} \prod_{j=1}^k \lambda_{i_j}
    \end{equation}
    that is $S_k(A)$ is the sum of the product of all $k$ eigenvalues, which is the coefficient of characteristic polynomial of $f_A(t)$:
    \begin{equation}
        f_A(t) = (-1)^n t^n + (-1)^{n-1} S_{1}(\lambda)t^{n-1} + \dots + (-1)^{n-k} S_{k} t^{n-1} + \dots + S_{n}
    \end{equation}
    Define the sum of all\footnote{There are $\binom{n}{k}$ of them.} principal minor of size $k$ of $A$ as $E_k(A)$. We have
    \begin{equation}
        E_k(A) = S_k(A)
    \end{equation}
    So 
    \begin{equation}
        \text{tr} A = \sum \lambda_i
    \end{equation}
    and 
    \begin{equation}
        \determinate{A} = \prod \lambda_i
    \end{equation}
\end{theorem}
\begin{proof}
    calculate the coefficient by $\dfrac{1}{k!} \eval{\dod[k]{f_A(t)}{t}}_{t = 0}$
\end{proof}



\begin{theorem}
    The choice of basis $\beta$ did not change the eigenvalue of $T$. 
\end{theorem}
\begin{proof}
    \begin{equation*}
        \absolutevalue{\coordinate{T}_\beta - \lambda I} = \absolutevalue{Q^{-1} \mleft( \coordinate{T}_\alpha - \lambda I \mright) Q} = \absolutevalue{Q^{-1}} \times \absolutevalue{\coordinate{T}_\alpha - \lambda I } \times \absolutevalue{Q} = \absolutevalue{\coordinate{T}_\alpha - \lambda I } 
    \end{equation*}
\end{proof}

\begin{theorem}
Similar matrices have the same characteristic function.    
\end{theorem}
\begin{proof}
    Assume $A$ is similar to $B$: $A = P^{-1} B P$. We have
    \begin{equation*}
    f_A(\lambda) = \determinate{Ax - \lambda I} = \determinate{P^{-1} B P - \lambda P^{-1} P} = \determinate{P^{-1}} \times \determinate{B - \lambda I } \times \determinate{P} = \determinate{B - \lambda I } = f_B(\lambda)
    \end{equation*}
\end{proof}




\begin{theorem}
    if $Q$ is a matrix with columns of eigenvectors of $\beta$, then according to \thmref{specialchangeofcoordinates} , $Q^{-1} A Q$ is a diagonal matrix with eigenvalue.
\end{theorem}



\subsection{Diagonalizability}


\begin{theorem}
    Let ${\lambda_i}$ be distinct eigenvalue of $T$. If $\set{v_i}$ are eigenvector that corresponding to $\lambda_i$, then $\set{v_i}$ is \emph{linearly independent}.
\end{theorem}
\begin{proof}
    suppose it works for $k - 1 \geq 1$ and we have $k$ eigenvector $\{ v_i\}$. Suppose
    \begin{equation*}
        a_1 v_1 + a_2 v_2 + \dots + a_k v_k = 0
    \end{equation*}
    
    multiply $T - \lambda_k I$ to both sides, we have
    \begin{equation*}
        a_1(\lambda_1 - \lambda_k) v_1 + a_1(\lambda_2 - \lambda_k) v_2 +  \dots + a_1(\lambda_{k-1} - \lambda_k) v_{k-1} +  = 0
    \end{equation*}
    
    because $\set{v_1, v_2, \dots, v_{k-1} }$ are linearly independent, we have 
    \begin{equation*}
        a_1(\lambda_1 - \lambda_k) = a_1(\lambda_2 - \lambda_k) =  a_1(\lambda_{k-1} - \lambda_k) = 0
    \end{equation*}
    
    because $\lambda_i$ are different, we have $a_i = 0$.
\end{proof}

\begin{theorem}
    if $T$ has $n$ distinct eigenvalues, then $T$ is diagonalizable. If $T$ is diagonalizable, it may not have $n$ distinct eigenvalues, for example the identity matrix $I_V$.
\end{theorem}

\begin{definition}
    A polynomial $f(t)$ in $P(F)$ \cindex{split over} $F$ if there are scalars $c, a_1, \dots, a_n$ (not necessarily distinct) in $F$ that
    \begin{equation*}
        f(t) = c(t - a_1)(t - a_2) \dots (t-a_n)
    \end{equation*}
    
    the \cindex{multiplicity} of $\lambda$ is the largest positive integer $k$ for which $(t - \lambda)^k$ is a factor of $f(t)$.
\end{definition}

\begin{theorem}
    the characteristic polynomial of any diagonalizable linear operator splits.
\end{theorem}

\begin{proof}
    choose a basis $\beta$ of eigenvectors. $[\mathrm{T}]_\beta$ is a diagonal matrix $D$. The characteristic polynomial of $T$ is $|D - tI|$ splits.
\end{proof}

Be careful that the characteristic polynomial splits does not mean the matrix is diagonalizable. The eigenvectors need to form a basis.


\begin{definition}
    let $\lambda$ be an eigenvalue of $T$. Let $E_\lambda = \nullspace{T - \lambda I_V}$. the set $E_\lambda$ is called the \cindex{eigenspace} of $T$ corresponding to eigenvalue $\lambda$. So is it for matrix.
\end{definition}

\begin{theorem}
    let $\lambda$ be an eigenvalue of $T$ having multiplicity $m$. then $1 \leq \dimension{E_\lambda} \leq m$.
\end{theorem}
\begin{proof}
    choose ordered basis $\set{v_1, v_2, \dots, v_p}$ for $E_\lambda$, and extend it to ordered basis $\beta =\set{v_1, v_2, \dots, v_p, v_{p+1}, \dots, v_n}$ for $V$, and let $A = \coordinate{T}_\beta$. let $v_i (1 \leq i \leq q)$ be an eigenvector of $T$ corresponding to $\lambda$, we have
    \begin{equation*}
        A = \begin{pmatrix}
            \lambda I_p & B \\
            0 & C
        \end{pmatrix}
    \end{equation*}
    so \begin{equation*}
        \begin{aligned}
            f(t) &= \absolutevalue{A - t I_n} \\
            &= \absolutevalue{\begin{bmatrix}
                (\lambda - t) I_p & B \\
                0 & C - t I_{n-p}
            \end{bmatrix}} \\
            &= \absolutevalue{(\lambda - t)I_p} \times \absolutevalue{C - t I_{n-p}} \\
            &= (\lambda - t)^p g(t)
        \end{aligned}
        \end{equation*}
    So $(\lambda - t)^p$ is a factor of $f(t)$, and the multiplicity of $\lambda$ is at least $p = \text{dim}(E_\lambda)$, so $\text{dim}(E_\lambda) \leq m$ 
\end{proof}

\begin{theorem}
    let $\set{\lambda_1, \lambda_2, \dots, \lambda_k}$ be distinct eigenvalue of $T$. let $S_i$ be a finite linearly independent subset of eigenspace $E_{\lambda_i}$. then $S_1 \cup S_2 \cup \dots \cup S_k$ is a linearly independent subset of $V$.
\end{theorem}

\begin{theorem}
    let $\lambda_1, \lambda_2, \dots, \lambda_k$ be distinct eigenvalue of $T$, then
    \begin{enumerate}
        \item $T$ is diagonalizable $\iff$ the multiplicity of $\lambda_i$ is equal to $\dimension{E_{\lambda_i}}$ for all $i$.
        \item If $T$ is diagonalizable and $\beta_i$ is an ordered basis for $E_{\lambda_i}$ for each $i$, then $\beta = \beta_1 \cup \beta_2 \cup \dots \cup \beta_k$ is an ordered basis for $V$ consisting of eigenvectors of $T$.
    \end{enumerate}
\end{theorem}

\begin{theorem}
    $T$ is diagonalizable $\iff$ both of the following holds:
    \begin{enumerate}
        \item the characteristic polynomial of $T$ splits.
        \item for each eigenvalue $\lambda$ of $T$, the multiplicity of $\lambda$ equals $n - \rank{T - \lambda I}$.
    \end{enumerate}
\end{theorem}

\begin{definition}
    Let $W_i$ be subspaces of a vector space $V$. The \cindex{sum} of these subspaces is defined as:
    \begin{equation}
        \sum_{i=1}^k W_i = \set{v_1 + v_2 + \dots + v_k : v_i \in W_i \text{ for } 1 \leq i \leq k }
    \end{equation}
\end{definition}

\begin{definition}
    let $W_i$ be subspace of $V$. $V$ is the \cindex{direct sum} of subspace $\set{W_1, W_2, \dots, W_k}$, or $V = W_1 \oplus W_2 \oplus \dots \oplus W_k$ if
    \begin{equation*}
        V = \sum_{i=1}^k W_i
    \end{equation*}
    and 
    \begin{equation*}
        W_j \cap \sum_{i \neq j} W_i = \emptyset, (1 \leq j \leq k)
    \end{equation*}
\end{definition}

\begin{theorem}
    $T$ is diagonalizable $\iff$ $V$ is the direct sum of eigenspaces of $T$.
\end{theorem}


\subsection{Invariant Subspaces}

\begin{definition}
    A subspace $W$ of $V$ is $T$-\cindex{invariant subspace} of $V$ if $T(W) \subseteq W$.
    Common $T$-invariant subspaces are: $\emptyset$, $V$, $R(T)$, $N(T)$.
    \qed
\end{definition}

\begin{theorem}
    A subspace $W$ with basis $\alpha = \set{v_1, v_2, \dots, v_k}$ is $T$-invariant. Let $\beta = \alpha \cup \gamma$ as the expanded basis of $V$. Then
    \begin{equation}
        \coordinate{T}_\beta = \begin{bmatrix}
            A_{k \times k} & B \\
            0 & C
        \end{bmatrix}
    \end{equation}
    The reverse is true. If $\coordinate{T}_\beta$  has such representation, the first $k$ basis of $\beta$ is $T$-invariant. 
\end{theorem}



\begin{definition}
    A $T$-\cindex{cyclic subspace} of $V$ generated by $x$ is defined as $W=\text{span} \mleft(  \set{x, T(x), T^2(x), \dots} \mright)$.
\end{definition}

\begin{theorem}
    Let $T$ be a linear operator on finite-dimensional vector space $V$, and let $W$ be a $T$-invariant subspace of $V$. Then the characteristic polynomial of $T_W$ divides the characteristic polynomial of $T$.
\end{theorem}

\begin{proof}
    Choose ordered basis $\gamma$ for $W$ and expand it to $\beta$ for $V$. Calculate $\coordinate{T}_\beta$ and $\coordinate{T}_\gamma$.
\end{proof}


\begin{theorem}
    Let $T$ be a linear operator on finiate-dimensional vector space $V$, and let $W$ be a $T$-cyclic subspace of $V$ generated by nonzero vector $v \in V$. Let $k = \dimension{W}$. Then:
    \begin{enumerate}
        \item $\set{v, T(v), T^2(v), \dots, T^{k-1}(v)}$ is a basis for $W$.
        \item If $a_0 v + a_1 T(v) + a_2 T^2(v) + \dots + a_{k-1} T^{k-1}(v) + T^k(v) = 0$, then the characteristic polynomial of $T_W$ is $f(t) = (-1)^k \mleft( a_0 + a_1 t + \dots + a_{k-1} t^{k-1} + t^k \mright)$.
    \end{enumerate}
\end{theorem}

\begin{proof}
    Let $\beta = \set{v, T(v), T^2(v), \dots, T^{k-1}(v)}$, and let $a_i$ be the scalars that 
    \begin{equation*}
            a_0 v + a_1 T(v) + a_2 T^2(v) + \dots + a_{k-1} T^{k-1}(v) + T^k(v) = 0
    \end{equation*}
    
    Fors basis $\set{v, T(v), T^2(v), \dots, T^{k-1}(v)}$, $\coordinate{T(v)}_\beta = \coordinate{0,1,\dots, 0}$, $T\mleft( T(v)\mright)_\beta = \coordinate{0,0,1,\dots, 0}$, etc, we have:
    \begin{equation*}
        [T_W]_\beta = \begin{bmatrix}
            0 & 0 & \dots & 0 & - a_0 \\
            1 & 0 & \dots & 0 & -a_1\\
            \vdots & \vdots  & & \vdots & \vdots \\
            0 & 0 & \dots & 1 & - a_{k-1}
        \end{bmatrix}
    \end{equation*}
    which has characteristic polynomial
    \begin{equation*}
        f(t) = (-1)^k (a_0 + a_1 t + \dots + a_{k-1} t^{k-1} + t^k)
    \end{equation*}
\end{proof}

\begin{theorem}[\cindex{Cayley-Hamilton}]
    Let $T$ be linear operator on a finite-dimensional vector space $V$, and let $f(t)$ be the characteristic polynomial of $T$. Then $f(T) = 0$.
\end{theorem}

\begin{proof}
    Suppose $v \neq 0$. Let $W$ be the $T$-cyclic subspace generated by $v$, and suppose the $\dimension{W} =k$. So there exists scalars $\set{a_i}$ that 
    \begin{equation*}
        a_0 v + a_1 T(v) + a_2 T^2(v) + \dots + a_{k-1} T^{k-1}(v) + T^k(v) = 0
    \end{equation*}
    which implies the characteristic polynomial of $T_W$ is
    \begin{equation*}
        g(t) = (-1)^k \mleft(a_0 + a_1 t + \dots + a_{k-1} t^{k-1} + t^k \mright)
    \end{equation*}
    We have
    \begin{equation*}
        g(T)(v) = (-1)^k \mleft(a_0 I + a_1 T + \dots + a_{k-1} T^{k-1} + T^k \mright)(v) = 0
    \end{equation*}
    Because $g(t)$ divides $f(t)$, $\exists q(t)$ that $f(t) = g(t) q(t)$. So
    \begin{equation*}
        f(T)(v) = q(T)g(T)(v) = q(T) \mleft(g(T)(v)\mright) = q(T)(0) = 0
    \end{equation*}
\end{proof}



\begin{definition}
    Let $B_1 \in M_{m \times m}(F)$, and $B_2 \in M_{n \times n} (F)$. The \cindex{direct sum} of $B_1$ and $B_2$, denoted as $B_1 \oplus B_2$, as the $(m+n) \times (m+n)$ matrix $A$ that
    \begin{equation*}
        A = \begin{bmatrix}
            B_1 & 0 \\
            0 & B_2
        \end{bmatrix}
    \end{equation*}
\end{definition}


\begin{theorem}
    Suppose $V=W_1 \oplus W_2 \oplus \dots \oplus W_k$, where $W_i$ is a $T$-invariant subspace of $V$. Suppose $f_i(t)$ is the characteristic polynomial of $T_{W_i}$, Then $\displaystyle \prod_{i=1}^k f_i$ is the characteristic polynomial of $T$. Let $\beta_i$ be an ordered basis for $W_i$, and let $\displaystyle \beta = \bigcup_{i=1}^k \beta_i$. Let $A=\coordinate{T}_\beta$, and $B_i=[T_{W_i}]_\beta$. Then $A = B_1 \oplus B_2 \oplus \dots \oplus B_k$.
\end{theorem}



\subsection{Limit of Markov Chain Matrix}

\begin{definition}
    A sequence $\set{A_1, A_2, \dots}$ \cindex{converge} to \cindex{limit} $L$ if $\displaystyle \lim_{m \rightarrow \infty} (A_m)_{ij} = L_{ij}$.
\end{definition}

\begin{theorem}
    If $A_i \rightarrow L$, them for any $P$ and $Q$, $\displaystyle \lim_{m \rightarrow \infty} P A_m = PL$ and $\displaystyle \lim_{m \rightarrow \infty} A_m Q = LQ$.
\end{theorem}

\begin{theorem}
    Let $Q$ be invertible and $A_i \rightarrow L$. Then $\displaystyle \lim_{m \rightarrow \infty} (Q A Q^{-1})^m = Q A Q^{-1}$.
\end{theorem}

\begin{definition}
    Define a set $S$ which consists of the interior of unit disk and $1$:
    \begin{equation}
        S = \set{\lambda \in C :  \absolutevalue{\lambda} < 1   \vee \lambda = 1}
    \end{equation}
\end{definition}


\begin{theorem}
    Let $A$ be square matrix in $C$. $\displaystyle \lim_{m \rightarrow \infty} A^m$ exists if and only if:
    \begin{enumerate}
        \item Every eigenvalue of $A$ is in $S$.
        \item If $1$ is an eigenvalue of $A$, then the dimension of its eigenspace equals its multiplicity.
    \end{enumerate}
\end{theorem}
\begin{proof}
    use Jordan canonical form.
\end{proof}


\begin{theorem}
    For square matrix $A$ in $C$, if
    \begin{enumerate}
        \item Every eigenvalue of $A$ is in $S$.
        \item $A$ is diagonalizable.
    \end{enumerate}    
    Then $\displaystyle \lim_{m \rightarrow \infty} A^m$ exists.
\end{theorem}
\begin{proof}
    Since $A$ is diagonalizable, $\exists Q: A = Q D Q^{-1}$. So $A^m = Q D^m Q^{-1}$. This is used to calculate $A^m$.
\end{proof}

\begin{definition}
    \cindex{transition matrix} or \cindex{stochastic matrix} is a square matrix $A$ that $A_{ij} \geq 0 \wedge \forall j \mleft(\sum_{i} A_{ij} = 1 \mright)$.
\end{definition}

\begin{definition}
    $P$ is a \cindex{probability vector} if its entries are all non-negative and sum to $1$.
\end{definition}

\begin{definition}
    \cindex{$\vec{1_n}$} is a column vector that each coordinate is $1$.
\end{definition}

\begin{theorem}
    Let $M$ be a square matrix with non-negative real entries, and $v$ a column vector with real non-negative coordinates. Then
    \begin{enumerate}
        \item $M$ is a transition matrix if and only if $M^\top \vec{1_n} = \vec{1_n}$.
        \item $v$ is a probability vector if and only if $\vec{1_n}^\top v = 1$.
        \item The product of two transition matrix is transition matrix.
        \item The product of a transition matrix and probability vector is a probability vector.
    \end{enumerate}    
\end{theorem}

\begin{definition}
    A transition matrix is \cindex{regular} if some power of the matrix contains only positive entries. It may contain zero entries.
\end{definition}

\begin{definition}
    For square matrix $A$, define $\displaystyle \rho_i (A) = \sum_j \absolutevalue{A_{ij}}$ and $\displaystyle v_j(A) = \sum_i \absolutevalue{A_{ij}}$. The \cindex{row sum} $\rho (A) = \max \rho_i$ and \cindex{column sum} $v(A) = \max v_j$.
\end{definition}

\begin{definition}
    For square matrix $A_{n \times n}$, the \cindex{Gerschgorin disk} $C_i$ is defined as:
    \begin{equation}
        C_i = \set{z \in C: \absolutevalue{z - A_{ii}} < \rho_i (A) - \absolutevalue{A_{ii}}}
    \end{equation}
    So the disk center is the diagonal entry, and the radius is the sum of the absolute values of all rest row entries.
\end{definition}

\begin{theorem}
    Every eigenvalue of $A$ is contained in a Gerschgorin disk.    
\end{theorem}
\begin{proof}
    Let $\lambda$ be a eigenvalue with eigenvector $v$. So $\displaystyle \sum_{j=1}^n A_{ij} v_j = \lambda v_i$. Assume $v_k$ is the coordinate of $v$ that has the largest absolute value. Then $v_k \neq 0$ because $v \neq 0$. We have
    \begin{equation*}
        \begin{aligned}
            \absolutevalue{\lambda v_k - A_{kk} v_k} = \absolutevalue{\sum_{j=1}^n A_kj v_j - A_{kk} v_k} = \absolutevalue{\sum_{j \neq k} A_{kj} v_j} \leq \sum_{j \neq k} \absolutevalue{A_{kj}} \absolutevalue{v_j} \leq \sum_{j \neq k} \absolutevalue{A_{kj}} \absolutevalue{v_k} = \absolutevalue{v_k} \mleft(\rho_i (A) - \absolutevalue{A_{kk}} \mright)
        \end{aligned}
    \end{equation*}
    So $\absolutevalue{v_k} \times \absolutevalue{\lambda - A_{kk} } \leq \absolutevalue{v_k} \mleft(\rho_i (A) - \absolutevalue{A_{kk}} \mright)$ and $\absolutevalue{\lambda - A_{kk} } \leq \mleft(\rho_i (A) - \absolutevalue{A_{kk}} \mright)$.
\end{proof}

\begin{theorem}
    Let $\lambda$ be any eigenvalue of $A$. Then $\absolutevalue{\lambda} \leq \rho(A)$.
\end{theorem}
\begin{proof}
    $\displaystyle \absolutevalue{\lambda} = \absolutevalue{(\lambda - A_{kk}) + A_{kk}} \leq \absolutevalue{\lambda - A_{kk}} + \absolutevalue{A_{kk}} \leq \rho_i (A) - \absolutevalue{A_{kk}}  + \absolutevalue{A_{kk}}  = \rho_i (A)$
\end{proof}

\begin{theorem}
    Let $\lambda$ be any eigenvalue of $A$. Then $\absolutevalue{\lambda} \leq \min \set{\rho(A), v(A)}$.
\end{theorem}
\begin{proof}
    $\lambda$ is an eigenvalue of $A^\top$.
\end{proof}

\begin{theorem}
    If $\lambda$ is an eigenvalue of transition matrix, then $\absolutevalue{\lambda} \leq 1$.
\end{theorem}

\begin{theorem}
    Every transition matrix has $1$ as eigenvalue.    
\end{theorem}
\begin{proof}
    $A^\top \times  \vec{1_n} = \vec{1_n}$.
\end{proof}

\begin{theorem}
    Let $A$ be a matrix with positive entries, and let $\lambda$ be an eigenvalue of $A$ that $\absolutevalue{\lambda} = \rho(A)$. Then $\lambda = \rho(A)$ and $\vec{1_n}$ is a basis for $E_\lambda$.
\end{theorem}
\begin{proof}
    Let $v$ be an eigenvector for $\lambda$, and $v_k$ is the coordinate that has the largest absolute value $b = \absolutevalue{v_k}$. Then
    \begin{equation*}
        \absolutevalue{\lambda} b = \absolutevalue{\lambda v_k} = \absolutevalue{\sum_{j=1}^n A_{kj} v_j} \leq \sum_{j=1}^n \absolutevalue{A_{kj} v_j} = \sum_{j=1}^n \absolutevalue{A_{kj}} \absolutevalue{v_j} \leq \sum_{j=1}^n \absolutevalue{A_{kj}} b = \rho_k(A) b \leq \rho(A) b
    \end{equation*}
    Since $\absolutevalue{\lambda} = \rho(A)$, all inequalities are equalities, so
    \begin{enumerate}
        \item \label{transitionmatrixproperty1}$\displaystyle \absolutevalue{\sum_{j=1}^n A_{kj} v_j} = \sum_{j=1}^n \absolutevalue{A_{kj} v_j}$
        \item \label{transitionmatrixproperty2}$\displaystyle \absolutevalue{A_{kj}} \absolutevalue{v_j} = \sum_{j=1}^n \absolutevalue{A_{kj}} b$
        \item \label{transitionmatrixproperty3}$\rho_k(A) \leq \rho(A)$
    \end{enumerate}
    
    For Item \ref{transitionmatrixproperty1} to hold, $A_{kj} v_j$ are non-negative multiplies of a common complex number $z$. Assume $\absolutevalue{z}=1$. Then $\mleft(\exists \set{c_j} \subset R^+ \mright) (A_{kj} v_j = c_j z)$.
    
    For item \ref{transitionmatrixproperty2}, since $b = \max \absolutevalue{v_j}$, $\absolutevalue{v_j} = b$. So $\displaystyle b = \absolutevalue{v_j} = \absolutevalue{\frac{c_j}{A_{kj}} z} = \frac{c_j}{A_{kj}}$, and $\displaystyle v_j = \frac{c_j}{A_{kj}} z = bz$, and $v = bz \vec{1_n}$.
    
    Since $A$ and $\vec{1_n}$ are all positive, $A \vec{1_n} = \lambda \vec{1_n}$, so $\lambda > 0$.
\end{proof}

\begin{theorem}
    Let $A$ be a transition matrix that each entry is positive, and let $\lambda$ be any eigenvalue of $A$ other than $1$. Then $\absolutevalue{\lambda} < 1$. Moreover, the eigenspace of eigenvalue $1$ has dimension $1$.
\end{theorem}

\begin{theorem}
    Let $A$ be a regular transition matrix, and $\lambda$ be one of its eigenvalue, then
    \begin{enumerate}
        \item $\absolutevalue{\lambda} \leq 1$.
        \item If $\absolutevalue{\lambda} = 1$, then $\lambda = 1$ and $\dimension{E_\lambda} = 1$.
    \end{enumerate}    
\end{theorem}

\begin{theorem}
    Let $A$ be a disagonalizable regular transition matrix, then $\displaystyle \lim_{m \rightarrow \infty} A^m$ exists.
\end{theorem}

\begin{theorem}
    Let $A$ be a regular transition matrix, then
    \begin{enumerate}
        \item the multiplicity of eigenvalue $1$ is $1$.
        \item $\displaystyle \lim_{m \rightarrow \infty} A^m$ exists.
        \item $L = \displaystyle \lim_{m \rightarrow \infty} A^m$ is a transition matrix.
        \item $AL = LA = L$.
        \item The column of $L$ are identical vector $v$ which is the probability vector in $E_1$.
        \item For any probability vector $w$, $\displaystyle \lim_{m \rightarrow \infty} A^m w = v$.
    \end{enumerate}    
\end{theorem}
\begin{proof}
    Since $AL = L$, $L$ are columns of eigenvector for eigenvalue $1$. Let $y = \displaystyle \lim_{m \rightarrow \infty} A^m w = Lw$, $Ay = ALw = Lw = y$. So $y$ is an eigenvector for eigenvalue $1$, and $y = v$.
\end{proof}












































