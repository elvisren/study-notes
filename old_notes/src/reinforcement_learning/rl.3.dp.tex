


\section{Dynamic Programming}

DP are effective for medium size problems (million of states).

\subsection{Policy Evaluation (Prediction)}

If $P$ and $\pi$ are known, Bellman equation (\ref{bellman:v}) could be converted to iterative solution. All $V$ are randomly initialized and updated using \cindex{iterative policy evaluation}:

\begin{equation}
	V_{k+1} (s) = \sum_a \pi (a | s) \mleft ( \sum_{s',r} p(s',r | s,a) \Big ( r + \gamma V_{k}(s') \Big ) \mright )
\end{equation}

Here $V_{t+1}(s)$ means the value of $V(s)$ in $(t+1)$ round.

There are two ways to update $V_{t+1}(s)$:
\begin{itemize}
	\item copy $V_{t+1}(s)$ to a new array and update original array when sweeping is done
	\item \cindex{in-place} update: update $V(s)$ on the fly. updated value may be used immediately so it is faster than two array solution.
\end{itemize}


The problem is that iterative algorithm \cindex{sweep} through all state space, which might not be practical.

See Algorithm (\ref{algo:itepole}) for detail.

\begin{algorithm}
	\caption{Iterative policy evaluation, estimate $V_\pi$}\label{algo:itepole}
	
	\begin{algorithmic}[1]
		\Procedure{}{$\pi, p, \theta$}
			\State $\forall s \in S, V(s) \gets random$
			\State $V(terminal) \gets 0$
			\Repeat
				\State $\Delta \gets 0$
				\For{$s \in S$}
					\State $v \gets V(s)$
					\State \Comment{in-place update}
					\State $V(s) \gets \sum\limits_a \pi (a | s) \mleft ( \sum\limits_{s',r} p(s',r | s,a) \big(r + \gamma V(s') \big ) \mright )$ 
					\State $\Delta \gets \max{}(v, |v - V(s)|)$
				\EndFor
			\Until{$\Delta < \theta$}
		\EndProcedure
	\end{algorithmic}

\end{algorithm}


\subsection{Policy Improvement}

The reason for calculating value function is to help find a better policy. A new greedy policy $\pi^{\prime}$ could be calculated using:

\begin{equation}
	\begin{aligned}
		\pi^{\prime}(s) &= \underset{a \in \mathcal{A}(s)}{\text{argmax}} \ q_{\pi}(s,a)\\
		&=\underset{a \in \mathcal{A}(s)}{\text{argmax}}\ \sum_{s',r} p(s',r|s,a)\Big ( r+\gamma V(s') \Big )
	\end{aligned}
\end{equation}

A series of policy evaluation and improvement will converge to optimal result, and its conversion is very fast. 

The drawback is that every iteration may trigger evaluation, which involves multiple sweep through all state space.

See Algorithm (\ref{algo:polite}) for detail.


\begin{algorithm}
	\caption{Policy Iteration, estimate $V_*$ and $\pi_*$}\label{algo:polite}
	
	\begin{algorithmic}[1]
		\State $\forall s \in \mathcal{S}, V(s) \gets random$
		\State $V(terminal) \gets 0$
		\State $\pi(s) \gets \text{random}(\mathcal{A}(s))$
		
		\Statex
		
		\Procedure{PolicyEvaluation}{$\varepsilon$}
			\Repeat
				\State $\Delta \gets 0$
				\For{$s \in \mathcal{S}$}
					\State $v \gets V(s)$
					\State $V(s) \gets \sum\limits_{s',r} p(s',r|s,\pi(s))\big (r+\gamma V(s') \big )$ \Comment{$\pi(s)$ : use optimal policy}
					\State $\Delta \gets \max (\Delta, |v - V(s)|)$
				\EndFor
			\Until{$\Delta < \varepsilon$}
		\EndProcedure
		
		\Statex
		
		\Procedure{PolicyImprovement}{}
			\State $\text{stable} \gets \text{TRUE}$
			\For{$s \in \mathcal{S}$}
				\State $old \gets \pi(s)$
				\State $\pi(s) \gets \underset{a \in \mathcal{A}(s)}{\text{argmax}}\ \sum\limits_{s',r}P(s',r|s,a) \big (r+\gamma V(s') \big )$
				\If{$old \neq \pi(s)$}
					\State $\text{stable} \gets \text{FALSE}$
				\EndIf
			\EndFor
			\If{$\text{stable}$}
				\Return $(V_*,\pi_*)$
			\Else
				\State \Call{PolicyEvaluation}{$\varepsilon$} \Comment{update $V$ if optimal policy has changed}
			\EndIf
		\EndProcedure
	\end{algorithmic}
\end{algorithm}

\subsection{Value Iteration}

In policy iteration, multiple sweep can be reduce to one by taking the best action, and calculate optimal policy using (\ref{optimal:policy}):

\begin{equation}
	\begin{aligned}
		V_{k+1}^{old} (s) &= \sum_{s',r} p(s',r|s,\pi(s))\big (r+\gamma V_k(s') \big ) \\
		V_{k+1}^{new} (s) &= \underset{a}{\max} \sum_{s',r} p(s',r|s,a) \big (r+\gamma V_k(s') \big )
	\end{aligned}
\end{equation}

See Algorithm (\ref{algo:valite}) for detail.

\begin{algorithm}
	\caption{Value Iteration, estimate $\pi_*$}\label{algo:valite}
	
	\begin{algorithmic}[1]
		\State $\forall s \in S, V(s) \gets random$
		\State $V(terminal) \gets 0$
		
		\Statex
		
		\Repeat
			\State $\Delta \gets 0$
			\For{$s \in S$}
				\State $v \gets V(s)$
				\State \Comment policy evaluation and improvement
				\State $V(s) \gets \underset{a}{\max}\ \sum\limits_{s',r} p(s',r|s,a) \big ( r+\gamma V(s') \big )$ 
				\State $\Delta \gets \max (\Delta, |v - V(s)|)$
			\EndFor
		\Until{$\Delta < \theta$}
		
		\State \Return $\pi_*(s)= \underset{a}{\text{argmax}} \ \sum\limits_{s',r} p(s',r|s,a) \big (r+\gamma V(s') \big )$
	\end{algorithmic}
\end{algorithm}

\subsection{Generalized Policy Iteration}

\cindex{Generalized Policy Iteration} (GPI) is a series of evaluation and improvement process. Almost all reinforcement learning methods are GPI.


\subsection{Performance}

DP is exponentially faster than direct \cindex{policy space search}.

DP is better than \cindex{linear programming methods} for large problem, but worse for small problem.


The \cindex{curse of dimension} is not the problem of algorithm but the problem itself.

The time complexity for $v$ is $O(mn^2)$ and for $q$ is $O(m^2 n^2)$, where $m$ is the number of action and $n$ is the number of state. So DP is effective for medium size problems (million of states).

\subsection{Extension to Sweeping}

\subsubsection{Prioritized Sweeping}

backup the state with the maximum \cindex{Bellman error}:
\begin{equation}
	\mleft| \max_{ a \in \mathcal{A}} \mleft( \mathcal{R}
	+ \gamma \sum_{s' \in S} \mathcal{P}_{ss'}^a v(s') \mright) - v(s) \mright|
\end{equation}

\subsubsection{Realtime Dynamic Programming}

Choose the state that are relevant to agent.

