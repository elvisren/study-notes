
%
% section for MDP
%

\section{Markov Decision Process}

\subsection{Definition}

A \cindex{Markov decision process} (MDP) is a Markov reward process with decisions. It is an environment in which all stated are Markov. It is a tuple $\langle \mathcal{S}, \mathcal{A}, \mathcal{R}, \mathcal{P}, \gamma \rangle$:

\begin{itemize}
	\item $\mathcal{S}$ is a finite set of states.
	\item $\mathcal{A}$ is a finite set of actions.
	\item $\mathcal{R}$ is a reward function.
	\item $\mathcal{P}$ is a state transition probability matrix.
	\item $\gamma \in [0,1]$ is a discount factor.
\end{itemize}

\subsection{Goals and Equations}

\subsubsection{Environment}

\cindex{environment} has \cindex{state} $S_t \in \mathcal{S}$ and generate \cindex{reward} $R_{t+1} \in \mathbb{R}$. Anything that cannot be changed arbitrarily by the agent is part of environment. The goal of reinforcement learning is to maximize expected value of cumulative sum of received scalar reward.

The reward signal should be chosen so it will not affect how agent act.

\subsubsection{Agent}

\cindex{agent} has \cindex{action} $A_t \in \mathcal{A}(S_t)$ and \cindex{observation} $O_t$ of state $S_t$. Agent may know everything about how the environment works but still unable to solve problem, such as the Rubik cube puzzle.

\subsubsection{State and Reward}

The probability of next state and reward is :
\begin{equation}
	p(s^\prime,r|s,a) = \mathbb{P}\{S_t=s',R_t=r|S_{t-1}=s,A_{t-1}=a\}
\end{equation}

The \cindex{state-transition probabilities} is :
\begin{equation}
	p(s'|s,a)=P\{S_t=s'|S_{t-1}=s,A_{t-1} = a \}=\sum_{r \in \mathbb{R}} p(s',r|s,a)
\end{equation}





Usually the state and reward probability will be treated as independent, so their formulas are:

\begin{equation}
	\mathcal{P}_{s,s'}^a = \mathbb{P}[S_{t+1}=s' | S_t = s, A_t = a]
\end{equation}


\begin{equation}
	\mathcal{P}_{s,s'}^{\pi} = \sum_{a \in A(s)} \pi(a|s) \mathcal{P}_{s,s'}^a
\end{equation}


\begin{equation}
	\mathcal{R}_s^a = \mathbb{E}[R_{t+1}|S_t =s, A_t=a]
\end{equation}


\begin{equation}
	\mathcal{R}_s^\pi = \sum_{a \in A(s)} \pi(a|s) \mathcal{R}_s^a
\end{equation}



The expected reward of state-action pair is:
\begin{equation}
	r(s,a) = \mathbb{E}[R_t|S_{t-1}=s,A_{t-1}=a]= \sum_{r \in \mathbb{R}} \sum_{s' \in \mathbf{S}} p(s',r|s,a)
\end{equation}

\subsubsection{Episode}
An \cindex{episode} is an order sequence of $S_0, A_0,R_1,S_1,A_1, \dots, R_n, S_n$ in MDP. So an \cindex{episode} will always end.

\cindex{continuing task} is a list of actions that never terminate.

Episode task can be converted to continuing task by appending infinite \cindex{absorbing state}.


\subsubsection{Goal}
The \cindex{expected return} is the sum of rewards in the episode:
\begin{equation}
	G_t=R_{t+1}+R_{t+2} + R_{t+3} +\dots + R_T
\end{equation}

\begin{itemize}
	\item \cindex{terminal state}: $R_T$ 
	\item $S$: All non-terminal states
	\item $S^+$: all terminal and non-terminal states
\end{itemize}


\cindex{discounted return} for continuing task is defined as:
\begin{equation}
	\begin{aligned}
		G_t&=R_{t+1}+\gamma R_{t+2} + \gamma^2 R_{t+3} + \dots  \\
		&= \sum_{k=0}^\infty \gamma^k R_{t+k_1} \\
		&= R_{t+1} + \gamma G_{t+1}
	\end{aligned}
\end{equation}

where $0 \leq \gamma \leq 1$.


\subsubsection{Policy}

A stochastic policy is a probability of selecting next possible action:

\begin{equation}
	\pi(a|s)=\mathbb{P}[A_t = a | S_t = s]
\end{equation}

\subsubsection{Value Function}

The \cindex{state-value} function $V_\pi$ for policy $\pi$ is:

\begin{equation}
	\begin{aligned}
		V_{\pi}(s) &= \mathbb{E}_{\pi}[G_t|S_t=s] \\
		&=\mathbb{E}[R_{t+1} + \gamma V_{\pi}(S_{t+1})| S_t = s]
	\end{aligned}
\end{equation}

The \cindex{action-value} function $q_\pi$ for policy $\pi$ is:

\begin{equation}
	\begin{aligned}
		q_{\pi}(s,a) &= \mathbb{E}_{\pi}[G_t|S_t=s,A_t=a] \\
					&= \mathbb{E}[R_{t+1} + \gamma q_{\pi}(S_{t+1}, A_{t+1})| S_t = s, A_t = a]
	\end{aligned}
\end{equation}

\subsubsection{Bellman Equation}


Bellman equation (\cindex{backup} process) is an expansion of value function:

\begin{equation}\label{bellman:v}
	\begin{aligned}
		V_{\pi}(s) &= \sum_{a \in \mathcal{A}(s)} \pi(a|s) q_{\pi}(s,a) \\
		&= \sum_{a \in \mathcal{A}(s)} \pi(a|s) \mleft( \mathcal{R}_s^a + \gamma \sum_{s' \in \mathcal{S}} \mathcal{P}_{s,s'}^a V(S') \mright)
	\end{aligned}
\end{equation}

\begin{equation}
	\begin{aligned}
		q_{\pi}(s,a) &= R_s^a + \gamma \sum_{s' \in \mathcal{S}} \mathcal{P}_{s,s'}^a V_{\pi}(s') \\
		&= R_s^a + \gamma \sum_{s' \in \mathcal{S}} \mathcal{P}_{s,s'}^a \mleft( \sum_{a', s'} \pi(a'|s') q_{\pi}(s',a') \mright )
	\end{aligned}
\end{equation}



\subsubsection{Bellman Equation Solution}

Formula (\ref{bellman:v}) is for a single state. Let 
\begin{equation}
	V_{\pi} = \mleft [\begin{matrix}
 	V_{\pi}(s_1) \\
 	\vdots \\
 	V_{\pi}(s_n)
 \end{matrix} \mright ]
\end{equation}

Formula (\ref{bellman:v}) now becomes:

\begin{equation}
	\mathbf{V}_{\pi}=\mathbf{R}^{\pi} + \gamma \mathbf{P}^{\pi} \mathbf{V}_{\pi} 
\end{equation}

So the solution to \cindex{Bellman Equation} is:

\begin{equation}
	\mathbf{V}_{\pi}=(\mathbf{I} - \gamma \mathbf{P}^{\pi})^{-1}\mathbf{R}^{\pi}
\end{equation}

It is a fixed point solution to formula (\ref{bellman:v}).

\subsection{Optimal Policy}

\subsubsection{Policy Partial Order}

$\pi \geq \pi^{\prime}$ if $\forall s \in \mathcal{S}, V_{\pi} (s) \geq V_{\pi^{\prime}} (s)$ . 


For MDP all optimal solution share the same value function $V_*$ and include at least one deterministic policy $\pi_*$. 


\subsubsection{Optimal State-value Function}

The optimal state-value function $V_*$ is defined as:


\begin{equation}
	\begin{aligned}
		V_*(s) &= \underset{\pi}{\max} \ V_{\pi}(s) \\
		&= \underset{a}{\max}\ q_* (s, a) \\
		&= \underset{a}{\max}\ \mleft( \mathcal{R}_s^a + \gamma \sum_{s' \in \mathcal{S}} \mathcal{P}_{s,s'}^a V_*(s') \mright )
	\end{aligned}
\end{equation}

\subsubsection{Optimal Action-value Function}


The optimal action-value function $q_*$ is defined as:

\begin{equation}
	\begin{aligned}
		q_*(s,a)&=\underset{\pi}{\max}\ q_{\pi}(s,a)\\
		&=\mathbb{E}[R_{t+1}+\gamma V_*(S_{t+1})|S_t=s, A_t=a]\\		
		&=\mathcal{R}_s^a+\gamma \sum_{s' \in \mathcal{S}} \mathcal{P}_{s,s'}^a V_*(s')\\
		&=\mathcal{R}_s^a+\gamma \sum_{s' \in \mathcal{S}} \mathcal{P}_{s,s'}^a \ \underset{a'}{\max}\  q_*(s',a')
	\end{aligned}
\end{equation}

Here $\mathcal{R}_s^a$ is an expectation. In model-free learning and controlling it is the sample return from environment.

\subsubsection{Optimal Policy from Action Value Function} Once the optimal action value function is known, we can calculate the optimal policy by:

\begin{equation}\label{optimal:policy}
	\pi_*(a|s) =
	\begin{cases}
		1& \text{, if $a = \underset{a \in \mathcal{A}(s)}{\text{argmax}}\ q_*(s,a)$}\\
		0& \text{, else}
	\end{cases}
\end{equation}

So the optimal policy is a greedy algorithm.

Compared with $V_*$, $q_*$ is better because it does not need to do one step lookahead.

\subsubsection{Reason for Complex Algorithms}

There is no closed form for optimal Bellman policy equation, so many iterative solution exists:
\begin{itemize}
	\item value iteration
	\item policy iteration
	\item $Q$-learning
	\item Sarsa
\end{itemize}

