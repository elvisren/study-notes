\section{Static Games of Complete Information}

\subsection{Definition}

\begin{definition}[\cindex{common knowledge}]
    An event $E$ is common knowledge if:
    \begin{enumerate}
        \item Everyone knows $E$
        \item Everyone knows that everyone knows $E$, and so on ad infinitum
    \end{enumerate}
\end{definition}

The assumptions we will have are:
\begin{enumerate}
    \item Players are rational, which means they will maximize their payoff consistently.
    \item Players are intelligent, which means they know everything about the game
    \item Common knowledge, which means it is common knowledge that all players are rational and intelligent.
    \item Self-enforcement, which means they will act independently.
\end{enumerate}

\begin{definition}[\cindex{strategy}]
    A strategy is a plan of action intended to accomplish a specific goal. For example, "if $A$ do something then I will respond with ...". The set of all strategies for player $i$ is denoted as $S_i$.
\end{definition}

\begin{definition}[\cindex{strategy profile}]
    The strategy profile is the set of all strategies of all players.
\end{definition}

\begin{definition}[\cindex{pure strategy}]
    A pure strategy for a player is a deterministic plan of actions.
\end{definition}

\begin{definition}[\cindex{normal-form game}]
    A normal-form game consistes of three features:
    \begin{enumerate}
        \item A finite set of players. $N = \set{1, 2, ... , n}$.
        \item A collection of sets of pure strategies, $\set{S_1, S_2, ... , S_n}$. The set of all strategy profiles is denoted by $\displaystyle S = \prod_{j \in N} S_j$. A strategy profile is $s \in S$.
        \item A set of payoff functions $\set{v_1, v_2, ..., v_n}$, each assigning a payoff value to each combination of chosen strategies, that is, $\displaystyle v_i : S \rightarrow \real $.
    \end{enumerate}
    
    A normal-form game is a tuple $\Gamma = \mleft\langle N, \set{S_i}_{i \in N}, \set{v_i}_{i \in N} \mright\rangle$. Sometimes we use actions $A$ to stand for strategies $S$ and $\succsim$ for $v_i$. So the tuple becomes $\Gamma = \mleft\langle N, A_i, \succsim_i \mright\rangle$.
\end{definition}

\subsection{Dominance}

\begin{definition}
    A strategy profile $s \in S$ \cindex{pareto dominate}s strategy profile $m \in S$ if $\forall i \in N, v_i(s) \geq v_i(m)$ and $\exists i \in N, v_i(s) > v_i(m)$. In this case $m$ is \cindex{pareto dominated} by $s$. A strategy profile is \cindex{pareto optimal} if it is not pareto dominated by any other strategy profiles.
\end{definition}

\begin{definition}[\cindex{belief}]
    Define $\displaystyle S_{-i} = \prod_{j \neq i} S_j$ and strategy profile $s_{-i} \in S_{-i}$. $s_{-i}$ is a belief of player $i$. The payoff of player $i$ for strategy profile $s$ is now $v_i(s_i, s_{-i})$ where $s = (s_i, s_{-i})$.
\end{definition}

\begin{definition}
    For player $i$ with strategy $n_i \in S_i$ and $m_i \in S_i$, $m_i$ is \cindex{weakly dominated} by $n_i$ , or $n_i$ \cindex{$\succeq$} $m_i$ if 
    \begin{equation*}
       \forall s_{-i} \in S_{-i}, v_i (n_i, s_{-i}) \geq v_i (m_i, s_{-i})
    \end{equation*}
    
    $m_i$ is \cindex{strictly dominated} by $n_i$, or $n_i$ \cindex{$\succ$} $m_i$, if 
    \begin{equation*}
       \forall s_{-i} \in S_{-i}, v_i (n_i, s_{-i}) > v_i (m_i, s_{-i})
    \end{equation*}
\end{definition}

\begin{theorem}
    A rational player will never play a strictly dominated strategy.    
\end{theorem}

\begin{definition}[\cindex{strictly dominant strategy}]
    $m_i \in S_i$ is a strictly dominant strategy if $\forall n_i \in S_i, n_i \neq m_i, m_i \succ n_i$.
\end{definition}

\begin{definition}[\cindex{strictly dominant strategy equilibrium}]
    The strategy profile $s^D \in S$ is a strictly dominant strategy equilibrium if $s_i^D \in S_i$ is a strictly dominant strategy for all $i \in N$.
\end{definition}

\begin{theorem}
    If a game has a strictly dominant strategy equilibrium $s^D$,  $s^D$ is unique.
\end{theorem}

\begin{definition}[\cindex{IESDS}]
    The \cindex{iterated elimination of strictly dominated strategies} is defined as:
    \begin{enumerate}
        \item Define $\forall i, S_i^0 = S_i$, and $S^0 = \prod S_i^0$
        \item For all players $i$, find its strictely dominated strategy $s_i \in S_i$ (if any).
        \item For all players $i$, define $S_i^{k+1} = S_i^k \backslash \set{s_i}$, and $S^{k+1} = \prod S_i^{k+1}$
        \item If $S^{k+1} = S^k$, define $S^{ES} = S^{k+1}$ and terminate.
    \end{enumerate}
    
    A strategy profile $s^{ES} \in S^{ES}$ is called \cindex{iterated-elimination equilibrium}.
\end{definition}

\begin{theorem}
    If $s^*$ is a strictly dominant strategy equilibrium for a game $\Gamma$, it is unique.
\end{theorem}


\subsection{Nash Equilibrium}

\begin{definition}[\cindex{best reponse}]
    The strategy $m_i \in S_i$ is player $i$'s best response to his opponents' strategy $s_{-i} \in S_{-i}$ if
    \begin{equation*}
        m_i \in \underset{s_i \in S_i}{\text{argmax}}\ v_i (s_i, s_{-i})
    \end{equation*}
\end{definition}

\begin{definition}[\cindex{best-reponse correspondence}]
    The best-response correspondence of player $i$ is a set value function $B_i: S_{-i} \rightarrow 2^{S_i}$ that $B_i(s_{-i})$ is the best reponse of player $i$ to $s_{-i}$.
\end{definition}


\begin{theorem}
    A rational player who believes his opponents are playing $s_{-i}$ will always choose the best response to $s_{-i}$.
\end{theorem}

\begin{theorem}
    A strictly dominated strategy cannot be the best response.    
\end{theorem}

\begin{theorem}
    If $s^*$ is the strictly dominant strategy equilibrium, or if $s^*$ is the unique strategy profile in $S^{ES}$, then $s_i^*$ is the best response to $s_{-i}^*$ for all $i \in N$.
\end{theorem}


\begin{definition}[\cindex{Nash equilibrium}]
    The pure strategy profile $s^*$ is a Nash equilibrium of $s_i^*$ is a best response to $s_{-i}^*$ for all $i$, that is:
    \begin{equation*}
        \forall i \in N, s_i^* \in B_i (s_{-i}^*)
    \end{equation*}
\end{definition}

The easiest way to find a Nash equilibrium is first to calculate the best response function of each player, then find a profile $s^*$ of actions for which $s_i^* \in B_i(s_{-1}^*)$ for all $i$.


Nash equilibrium is a steady-state so no player wants to change its choice.

\begin{theorem}
    If $s^*$ is either:
    \begin{enumerate}
        \item the strictly dominant strategy equilibrium,
        \item the unique survivor of IESDS
        \item the unique rationalizable strategy profile
    \end{enumerate}    
    Then $s^*$ is the unique Nash equilibrium.
\end{theorem}

\begin{theorem}
    Nash equilibrium does not guarantee Pareto optimality.    
\end{theorem}





\subsection{Mixed Strategy}

\begin{definition}[\cindex{mixed strategy}]
    Define $\bigtriangleup S_i$ as the \cindex{simplex} of $S_i$, which is the set of all probability distributions over $S_i$. A mixed strategy for player $i$ is an element $\sigma_i \in \bigtriangleup S_i$, so that $\sigma_i$ is a probability distribution over $S_i$. The same definition works for continuous probability case.
\end{definition}

\begin{definition}[\cindex{support}]
    For a mixed strategy $\sigma_i$ for player $i$, a pure strategy $s_i$ is in the support of $\sigma_i$ if and only if $\sigma_i (s_i) > 0$. The same definition works for continuous probability case.
\end{definition}

\begin{definition}[\cindex{belief}]
    A belief for player $i$ is a probability distribution $\pi_i \in \bigtriangleup S_{-i}$ over the strategies of his opponents. It is denoted by $\pi_i(s_{-i})$ for $s_{-i}$.
\end{definition}

\begin{definition}[\cindex{expected payoff}]
    The expected payoff of player $i$ when he chooses the pure strategy $s_i \in S_i$ and his opponents play the mixed strategy $\sigma_{-i} \in \bigtriangleup S_{-i}$ is:
    \begin{equation*}
        v_i(s_i, \sigma_{-i}) = \sum_{s_{-i} \in S_{-i}} \sigma_{-i}(s_{-i}) \times v_i (s_i, s_{-i})
    \end{equation*}
    
    When player $i$ choose a mixed strategy $\sigma_i \in \bigtriangleup S_i$ and his opponents play the mixed strategy $\sigma_{-i} \in \bigtriangleup S_{-i}$, the expected payoff is:
    \begin{equation*}
    \begin{aligned}
        v_i(\sigma_i, \sigma_{-i}) &= \sum_{s_i \in S_i} \sigma_i (s_i) \times v_i(s_i, \sigma_{-i}) \\
        &= \sum_{s_i \in S_i} \sigma_i (s_i)  \times \mleft(\sum_{s_{-i} \in S_{-i}} \sigma_{-i} (s_{-i}) \times v_i (s_i, s_{-i}) \mright) \\
        &= \sum_{s \in S} v_i(s) \times \prod_{i \in N} \sigma_i (s_i)
    \end{aligned}
            \end{equation*}
\end{definition}









\subsection{Nash Equilibrium for Mixed Strategy}

\begin{definition}
    The mixed strategy $\sigma^*$ is a Nash equilibrium if for each player $\sigma^*$ is the best response to $\sigma^*_{-1}$, that is:
    \begin{equation*}
        \forall i \in N, \sigma_i \in \bigtriangleup S_i, v_i(\sigma_i^*, \sigma_{-i}^*) \geq v_i(\sigma_i, \sigma_{-i}^*)
    \end{equation*}
\end{definition}

\begin{theorem}
    If $\sigma^*$ is a Nash equilibrium, and $m_i$ and $n_i$ are in the support of $\sigma_i^*$, then
    \begin{equation*}
        v_i (m_i, \sigma_{-i}^*) = v_i (n_i, \sigma_{-i}^*) = v_i (\sigma_i^*, \sigma_{-i}^*)
    \end{equation*}
\end{theorem}
\begin{proof}
    Assume on the contrary that $v_i (m_i, \sigma_{-i}^*) > v_i (n_i, \sigma_{-i}^*)$. We could assign higher probability to $m_i$ to increase the expected payoff. The end result of adjustment is the probability of $n_i$ would be $0$ and $n_i$ will not be in the support of $\sigma_i$.
\end{proof}

\begin{theorem}
    For a two players game, there can be no Nash equilibrium in which one plays a pure strategy and the other a mixed strategy.    
\end{theorem}
\begin{proof}
    If player $i$ plays a pure strategy. Player $j$ will face different payoffs when he plays randomly, so he will choose the highest payoff and become a pure strategy.
\end{proof}

\begin{definition}[\cindex{strictly dominated}]
    Let $\sigma_i \in \bigtriangleup S_i$ and $s_i \in S_i$ be possible strategies for player $i$. $s_i$ is strictly dominated by $\sigma_i$ if 
    \begin{equation*}
        \forall s_{-1} \in S_{-1}, v_i (\sigma_i, s_{-i}) > v_i (s_i, s_{-i})
    \end{equation*}
\end{definition}

\begin{theorem}[\cindex{Brouwer's Fixed-point Theorem}]
    If $f(x): [0,1] \rightarrow [0,1]$ is a continuous function, then there exists a $x^*$ that $f(x^*) = x^*$.
\end{theorem}

\begin{theorem}[\cindex{Kakutani's Fixed-point Theorem}]
    A function $f: X \rightarrow X$ has a fixed point $x$ if four conditions are met:
    \begin{enumerate}
        \item $X$ is a non-empty, compact and convex non-empty subset of $\real^n$.
        \item $\forall x, f(x) \neq \emptyset$
        \item $\forall x, f(x)$ is convex.
        \item $f$ has a closed graph.
    \end{enumerate}
\end{theorem}

\begin{definition}[\cindex{collection of best-response correspondences}]
    The collection of best-response correpsondences is a function $BR: \bigtriangleup S \rightarrow \bigtriangleup S$ that for every element $\sigma \in \bigtriangleup S$, $\mleft(BR(\sigma)\mright)_i$ is the best response to $\sigma_{-i}$.
\end{definition}

\begin{theorem}
    A mixed-strategy profile $\sigma^* \in \bigtriangleup S$ is a Nash equilibrium if and only if it is a fixed point of the collection of best-response correspondences, $\sigma^* \in BR(\sigma^*)$.
\end{theorem}

\begin{theorem}
    Every finite mixed strategy has a Nash equilibrium.    
\end{theorem}
\begin{proof}(use Kakutani's fixed-point theorem)
    If $\sigma_i$ and $\sigma_i^{'}$ are in $BR(\sigma_{-i})$. Their linear combinarion is also in $BR(\sigma_{-i})$.
\end{proof}













