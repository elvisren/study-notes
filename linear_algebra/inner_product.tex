\chapter{Inner Product}

This section analyzes the length property of vector space. All vector spaces here are over real or complex field.


Inner product defines the length operator $\norm{\cdot}$.
The inner product defines an algorithmic representation of linear transformation (to $F$) and linear operator. It also defines orthogonal relationship which leads us to the investigation of orthonormal eigenvectors.  


A linear operator on complex field is diagonalizable if it is normal. Normal operator has several properties:
\begin{itemize}
    \item $\norm{T} = \norm{T^*}$: normal
    \item $T - c  I$ is normal
    \item $T$ and $T^*$ has the same eigenvectors but conjugate eigenvalues
    \item Eigenvectors are orthogonal
\end{itemize}

A linear operator on real field is diagonalizable. It does not have any special property, except that all eigenvalues are real, which is obvious.

Unitary and orthogonal operators preserve length, so $\norm{T(x)} = \norm{x}$. But it does not have any other special property.

 The list of length relations are:
\begin{enumerate}
    \item $\norm{T} = \norm{T^*}$: normal
    \item $\norm{T(x)} = \norm{x}$: unitary (orthogonal)
    \item $\innerproduct{T(x)}{T(y)} = \innerproduct{x}{y}$: unitary (orthogonal)
\end{enumerate}

The relations between $T$ and $T^*$ are:
\begin{enumerate}
    \item $TT^* = T^*T$: normal
    \item $TT^* = T^*T = I$: unitary (orthogonal)
    \item $T^* = \inverse{T}$: unitary (orthogonal)
    \item $T^* = T$: self-adjoint (Hermitian)
\end{enumerate}



\section{Definitions}

\begin{definition}[\cindex{inner product}]
    An inner product is a function $\innerproduct{x}{y}: V \rightarrow V \rightarrow F$ that:
\begin{itemize}
    \item $\innerproduct{x + y}{z} = \innerproduct{x}{z} + \innerproduct{y}{z}$    
    \item $\innerproduct{x}{y+z} = \innerproduct{x}{y} + \innerproduct{x}{z}$
    \item $\innerproduct{cx}{y} = c \innerproduct{x}{y}$
    \item $\innerproduct{x}{cy} = \conjugate{c} \innerproduct{x}{y}$
    \item $\conjugate{\innerproduct{x}{y}} = \innerproduct{y}{x}$
    \item $\innerproduct{x}{x} \geq 0$
    \item $\innerproduct{x}{x} = 0$ if and only if $x = 0$
    \item $\innerproduct{x}{0} = \innerproduct{0}{x} = 0$
    \item If $\innerproduct{x}{y} = \innerproduct{x}{z}$ for all $x$, then $y = z$
\end{itemize}

A \cindex{inner product space} is a vector space with inner product.
\end{definition}



From the definition, we can find that
\begin{itemize}
    \item Inner product is linear in the first component
    \item Inner product is \cindex{conjugate linear} in the second component
\end{itemize}

Be careful that \emph{inner product could be negative}. So it is different from metric which has to be non-negative.


\begin{definition}[\cindex{conjugate transpose}]
    The conjugate transpose (\cindex{adjoint}) of $A_{m \times n}$ is 
    \begin{equation}
        A^* = \conjugate{\transpose{A}}
    \end{equation}
\end{definition}

\begin{example}
    Here are the common inner products:
\begin{itemize}
    \item For $x,y \in F^n$, $\displaystyle \innerproduct{x}{y} = \sum_{i=1}^n x_i \conjugate{y_i}$ (standard inner product)
    \item For $A,B \in M_{n \times n}$, $\innerproduct{A}{B} = \trace{B^* A}$ (\cindex{Forbenius Inner Product})
    \item For $f,g \in C([0,2 \pi])$, $\displaystyle \innerproduct{f}{g} = \frac{1}{2 \pi} \int_{0}^{2\pi} f(x) \conjugate{g(x)} \dif x$ (\cindex{inner product space $H$})\label{inner_product_space_h}
\end{itemize}
\end{example}


\begin{definition}[\cindex{norm}]
    The norm or \cindex{length} is defined as
    \begin{equation}
        \norm{x} = \sqrt{\innerproduct{x}{x}}
    \end{equation}
\end{definition}


\begin{theorem}
    The property of norm:
    \begin{itemize}
    \item $\norm{cx} = \absolutevalue{c} \norm{x}$
    \item $\norm{x} \geq 0$
    \item $\absolutevalue{\innerproduct{x}{y}} \leq \norm{x} \times \norm{y}$ (area of rectangle, \cindex{Cauchy-Schwarz Inequality})
    \item $\norm{x+y} \leq \norm{x} + \norm{y}$ (diagonal and edges)
    \end{itemize}
\end{theorem}

\begin{proof}
    The proof relies on $\norm{x} \geq 0$.
\end{proof}




\begin{theorem}
    For a parallelogram, the sum of the squares of the lengths of the diagonals equals the sum of the squares of the lengths of the four sides:
    \begin{equation}
        \norm{x + y}^2 + \norm{x-y}^2 = 2 \mleft(\norm{x}^2 + \norm{y}^2\mright)
    \end{equation}    
\end{theorem}


\begin{definition}
    For a inner product space $V$:
    \begin{enumerate}
        \item unit vector: $\norm{x} =1$
        \item orthogonal: $\innerproduct{x}{y} = 0$
        \item orthonormal: a subset $S$ of $V$ is orthonormal if $S$ consists of orthogonal unit vectors
    \end{enumerate}
\end{definition}

% orthogonal
\section{Orthogonal}

\begin{definition}[\cindex{orthogonal}]
    $x$ and $y$ are orthogonal if $\innerproduct{x}{y} = 0$. A subset $S$ of $V$ is orthogonal if any two vectors in $S$ are orthogonal. 
    
    A subset $S$ of $V$ is \cindex{orthonormal} if $S$ is orthogonal and consists entirely of unit vectors. So $S = \set{e_j}$ is orthonormal if and only if $\innerproduct{e_i}{e_j} = \delta_{ij}$\footnote{$\delta_{ij}$ is the Kronecker delta defined in \defiref{kronecker_delta}.} .
    
    A \cindex{orthonormal basis} for $V$ is an ordered basis that is orthonormal.
\end{definition}


\begin{definition}
    The $\cos\theta$ between two non-zero vector $x,y$ is defined as:
    \begin{equation}
        \innerproduct{x}{y} = \norm{x} \norm{y} \cos\theta
    \end{equation}    
\end{definition}


\begin{definition}[\cindex{unit vector}]
	A vector is unit vector if $\norm{x} = 1$. A \cindex{normalizing} to non-zero $x$ is $\dfrac{1}{\norm{x}} x$.
\end{definition}

\begin{theorem}
    A orthogonal set is linearly independent.
\end{theorem}


\begin{theorem}
    Let $f_n (t) = e^{i nt}$ where $0 \leq t \leq 2 \pi$. All $f_i$ are orthogonal.
\end{theorem}
\begin{proof}
    \begin{equation}
        \begin{aligned}
            \innerproduct{f_m}{f_n} &= \frac{1}{2 \pi} \int_0^{2 \pi} e^{imt} \overline{e^{int}} \dif{t} \\
            &= \frac{1}{2\pi} \int_0^{2\pi} e^{i (m-n) t} \dif{t} \\
            &= \eval{\frac{1}{2\pi (m-n)} e^{i(m-n)t}}_0^{2\pi} \\
            &= 0
        \end{aligned}
    \end{equation}
\end{proof}


\begin{theorem}
    If $x \perp y$, we have $\norm{x+y}^2 = \norm{x}^2 + \norm{y}^2$    
\end{theorem}




% projection
\section{Orthogonal Projection}



\begin{definition}[\cindex{orthogonal complement}]
    For a subset $S$ of $V$, the orthogonal complement $S^\perp$ is defined as $S^\perp=\set{x \in V: \forall y \in S, \innerproduct{x}{y} = 0}$.
\end{definition}

Here \emph{$S$ does not have to be a vector space, but $S^\perp$ is a vector space}. It could be any set and $(S^\perp)^\perp$ is usually not the same as $S$. 

If $S$ is a subspace, then we have
\begin{equation}
    V = S \oplus S^\perp
\end{equation}

\begin{definition}[\cindex{projection}]
    Let $V = W_1 \oplus W_2$. $\forall x = a + b$ where $a \in W_1$ and $b \in W_2$, if $T(x) = a$, then $T$ is a projection on $W_2$ along $W_1$.
\end{definition}


\begin{definition}[\cindex{orthogonal projection}]
    Orthogonal projection put extra requirement on projection that $W_1 \bot W_2$. 
    
    
    So if $T$ is an orthogonal projection on $U$, for any $v \in V$, write $v = u + w$ where $u \in U$ and $w \in U^\perp$, we have $T(v) = u$. Define it as:
    \begin{equation}
        T(v) = \projection{v}{U} = u
    \end{equation}
    
    Note that $u$ and $w$ are unique.
\end{definition}

Be careful that orthogonal projection operator is not an orthogonal operator.

\begin{definition}[\cindex{projection on vector}]
    For an orthogonal subset $S=\set{v_1, v_2, ..., v_k}$ of $V$. Define the \emph{projection of $y$ on $v$} as
    \begin{equation}
        \projection{y}{v} = \frac{\innerproduct{y}{v}}{\innerproduct{v}{v}} v
    \end{equation}
\end{definition}

\begin{definition}[\cindex{projection on subspace}]
    If $U$ is a subspace of $V$ with orthogonal basis $\set{v_1, \cdots, v_m}$, the \emph{projection of $y$ on $U$} is
    \begin{equation}
        \projection{y}{U} = \sum_{i=1}^{m} \projection{y}{v_i}
    \end{equation}
    
    If the \emph{basis $\set{e_i}$ is orthonormal}, the projection of $y$ on $U$ is
    \begin{equation}
        \projection{y}{U} = \sum_{i=1}^{m} \innerproduct{y}{e_i} e_i
    \end{equation}
    
    If $y \in U$, then $\projection{y}{U} = y$.
\end{definition}

\begin{theorem}
    All projections have a property that
    \begin{equation}
        \mleft(v - \projection{v}{U}\mright) \bot \mleft(\projection{v}{U}\mright)
    \end{equation}
\end{theorem}
\begin{proof}
    $\innerproduct{y - \frac{\innerproduct{y}{v}}{\innerproduct{v}{v}} v}{v} = \innerproduct{y}{v} - \frac{\innerproduct{y}{v}}{\innerproduct{v}{v}} \innerproduct{v}{v} = 0$.
\end{proof}


\begin{theorem}[\cindex{Gram-Schmidt process}]\label{gram_schmidt_process}
    The steps of converting a linearly independent subset $S=\set{w_1, w_2, ..., w_k}$ into orthonormal set $S^\prime=\set{v_1, v_2, ..., v_k}$. The idea is to 
    \begin{enumerate}
        \item Project $w_{j}$ onto $\set{v_1, v_2, ..., v_{j-1}}$.
        \item Remove the projection from $w_j$, and we have unnormalized $v_j$.
        \item Normalize $v_i$: $v_i \rightarrow \displaystyle \frac{v_i}{\norm{v_i}}$.
    \end{enumerate}
    
    The first two steps can be summarized as:
    \begin{equation}
        v_j = w_j - \sum_{i=1}^{j-1} \frac{\innerproduct{w_j}{v_i}}{\innerproduct{v_i}{v_i}} v_i = w_j - \sum_{i=1}^{j-1} \projection{w_j}{v_i}
    \end{equation}
\end{theorem}

Once we have orthonormal basis $\beta = \set{v_1, v_2, ..., v_n}$,  we could directly represent vectors and linear operators using orthonormal basis:
\begin{itemize}
    \item $\displaystyle x = \sum_{i=1}^n \innerproduct{x}{v_i} v_i$\footnote{$\innerproduct{x}{v_i}$ are the \cindex{Fourier coefficients}.}
    \item $\mleft(\coordinate{T}_\beta\mright)_{ij} = \innerproduct{T(v_j)}{v_i}$\footnote{extremely useful. The order of $i$ and $j$ are swapped}
    \item For linear functional\footnote{Defined in \defiref{linear_functional}.} $g: V \rightarrow F$, $g(x) = \innerproduct{x}{\sum_{i=1}^n \conjugate{g(v_i)} v_i}$\footnote{\cindex{Riesz representation theorem}, proved in \theoref{eiesz_representation_theorem}.}
\end{itemize}


\theoref{split_means_upper_trangular} did not give requirement on the basis $\beta$. In fact we could ask $\beta$ to be orthonormal according to the theorem below.

\begin{theorem}[\cindex{Schur}]\label{schur_theorem}
    If the characteristic polynomial of $T$ splits, then there exists an orthonormal basis $\gamma$ such that $\coordinate{T}_\gamma$ is upper triangular.
\end{theorem}
\begin{proof}
    If $T$ splits, \theoref{split_means_upper_trangular} means there is a basis $\beta$ that $\coordinate{T}_\beta$ is upper triangular. Then use the Gram-Schmidt process (\theoref{gram_schmidt_process}) to normalize $\beta$ and we have $\gamma$.
\end{proof}



\begin{theorem}
    Given $v \in V$, $\forall u \in U \subset V$, we have:
    \begin{equation}
        \norm{v - \projection{v}{U}} \leq \norm{v - u}
    \end{equation}
    
    The equality holds if and only if $u = \projection{v}{U}$.
\end{theorem}
\begin{proof}
    $\norm{v - u}^2 = \norm{v - \projection{v}{U}}^2 + \norm{\projection{v}{U} - u}^2 \geq \norm{\projection{v}{U} - u}^2$.
\end{proof}




\begin{example}
    We want to find which $g \in U$ could minimize the expression
    \begin{equation}
        \argmin{g} \frac{1}{2\pi}\int_{0}^{2\pi} \absolutevalue{f(x) - g(x)}^2 \dif x
    \end{equation}
    
     This is an expression of $\norm{f - g}^2$ on inner product space $H$\footnote{Defined in Example \ref{inner_product_space_h}} , so we would like to find a $g$ that $g \perp f$. We could project $f$ on $U$, which is to project $f$ on basis of $U$ and add up the projections.
     
     For example, $f(x) = \sin{x}$ and $g \in \mathcal{P}_5$. We could find the minimal $g$ by:
     \begin{enumerate}
         \item Convert basis $1, x, x^2, x^3, x^4, x^5$ to orthonormal basis $e_1, e_2, e_3, e_4, e_5$ in $H$
         \item Project $\sin{x}$ to basis $e_i$ as $\displaystyle g(x) = \sum_{i=1}^5 \innerproduct{\sin{x}}{e_i}e_i$
     \end{enumerate}
\end{example}



\begin{theorem}
    $T$ is a projection if and only if $T^2 = T$.    
\end{theorem}
\begin{proof}
    Assume $T^2 = T$. Then $T=I$ in subspace $\rangespace{T}$ and $T = 0$ in $\rangespace{T}^\perp$.
\end{proof}

\begin{theorem}
    $T$ is an orthogonal projection if and only if $T^2 = T = T^*$.
\end{theorem}
\begin{proof}
    $T$ is a projection, so $T^2 = T$. For $a_i \in \rangespace{T}$ and $b_i \in \nullspace{T}$, we have 
    \begin{equation*}
        \innerproduct{a_1 + b_1}{T(a_2 + b_2)} = \innerproduct{a_1}{a_2} = \innerproduct{T(a_1 + b_1)}{a_2 + b_2}
    \end{equation*}
    So $T = T^*$.
    
    For the reverse. Let $x \in \rangespace{T}$ and $y \in \nullspace{T}$. We have $x = T(x) = T^*(x)$. So
    \begin{equation*}
        \innerproduct{x}{y} = \innerproduct{T^*(x)}{y} = \innerproduct{x}{T(y)} = \innerproduct{x}{0} = 0
    \end{equation*}
    So $x \in \nullspace{T}^\perp$, and $\rangespace{T} \subseteq \nullspace{T}^\perp$.
    
    Now let $y \in \nullspace{T}^\perp$ and check $\norm{y - T(y)}^2 = \innerproduct{y}{y - T(y)} - \innerproduct{T(y)}{y - T(y)}$. Since $T(y - T(y)) = 0$, we have $y - T(y) \in \nullspace{T}$ and $y \in \nullspace{T}^\perp$, so $y \perp y - T(y)$ and $\innerproduct{y}{y - T(y)} = 0$. Also $\innerproduct{T(y)}{y - T(y)} = \innerproduct{y}{T(y) - TT(y)} = 0$. So $\norm{y - T(y)}^2 = 0$ and $T(y) = y$. So $y \in \rangespace{T}$ and $\nullspace{T}^\perp \subseteq \rangespace{T}$, and $\rangespace{T} = \nullspace{T}^\perp$.
\end{proof}


% adjoint
\section{Adjoint}

\begin{theorem}[\cindex{Riesz representation theorem}]\label{eiesz_representation_theorem}
Let $V$ be a inner product space over $F$, and let $g: V \rightarrow F$ be a linear transformation. Then there exists a unique $y \in V$ that $g(x) = \innerproduct{x}{y}$.
\end{theorem}
\begin{proof}
    Let $\beta = \set{v_1, v_2, \dots, v_n}$ be an orthonormal basis for $V$. Because $\displaystyle x = \sum_i \innerproduct{x}{v_i}v_i$, we have
    \begin{equation}
        g(x) = g\mleft(\sum_i \innerproduct{x}{v_i}v_i \mright) = \sum_i \innerproduct{x}{v_i}g(v_i) = \sum_i \innerproduct{x}{\conjugate{g(v_i)}v_i}
    \end{equation}
    
    So $\displaystyle y = \sum_{i=1}^n \conjugate{g(v_i)} v_i$.
\end{proof}


\begin{theorem}[\cindex{adjoint}]\label{adjoint_operator_definition}
    Let $T$ be linear operator on a finite-dimension inner product space $V$. There is a unique linear operator $T^*$ that
    
    \begin{equation}
        \begin{aligned}
            \innerproduct{T(x)}{y} &= \innerproduct{x}{T^* (y)} \\
            \innerproduct{x}{T(y)} &= \innerproduct{T^* (x)}{y}
        \end{aligned}
    \end{equation}
    
    
    $T^*$ is the \cindex{adjoint} of $T$. So we could move $T$ into the left or right side of the inner product.
\end{theorem}
\begin{proof}
    Linearity is easy to prove. To find $T^*$, for each $y$, there is a function $g_y (x) = \innerproduct{T(x)}{y} = \innerproduct{x}{g^\prime (y)}$. This $g^\prime$ is the $T^*$.
\end{proof}


For matrix $A_{m \times n}$, it has a nice property:
\begin{equation}
    \innerproduct{A x_n}{y_m}_m = \innerproduct{x_n}{A^* y_m}_n
\end{equation}

\begin{theorem}\label{operator_conjugate_transpose_requirement}
    Let $\beta$ be an orthonormal basis of $V$, for a linear operator $T$, we have
    \begin{equation}
        \coordinate{T^*}_\beta = \mleft(\coordinate{T}_\beta\mright)^*
    \end{equation}
    Be very careful that \emph{$\beta$ must be orthonormal.}
\end{theorem}
\begin{proof}
    Because $\beta$ is orthonormal, we have
    \begin{equation*}
        \mleft(\coordinate{T^*}_\beta\mright)_{ij} = \innerproduct{T^* (v_j)}{v_i} = \innerproduct{v_j}{T(v_i)} = \conjugate{\innerproduct{T(v_i)}{v_j)}} = \conjugate{(\coordinate{T}_\beta))_{ji}}
    \end{equation*}
\end{proof}



\begin{theorem}\label{congjugate_transpose_has_eigenvalue}
    If $T$ has eigenvalue, then $T^*$ has eigenvalue.    
\end{theorem}
\begin{proof}
    Let $v$ be an eigenvector for eigenvalue $\lambda$. Then for any $x$:
    \begin{equation*}
        0 = \innerproduct{(T-\lambda I)v}{x} = \innerproduct{v}{(T-\lambda I)^* x} = \innerproduct{v}{(T^* - \conjugate{\lambda}I) x}
    \end{equation*}
    
    So $v \perp \spanset{T^* - \conjugate{\lambda}I}$, which means $T^* - \conjugate{\lambda}I$ is not everything, so there is a nonzero null space.
\end{proof}


$T$ above are all linear operators. The adjoin could also be defined on linear transformation:

\begin{definition}
    Let $T : V \rightarrow W$ be a linear transformation where $V$ and $W$ are finite dimensional inner product space with inner product $\innerproduct{\cdot{}}{\cdot{}}_V$ and $\innerproduct{\cdot{}}{\cdot{}}_W$. A function $T^* : W \rightarrow V$ is called \cindex{adjoint} of $T$ if $\innerproduct{T(x)}{y}_W = \innerproduct{x}{T^*(y)}_V$.
\end{definition}

The extended definition of adjoint is used in SVD \theoref{svd_theorem}.

\begin{theorem}
    Let $T^*$ be an adjoint of $T: V \rightarrow W$. If $\beta$ and $\gamma$ are orthonormal basis for $V$ and $W$, then
    \begin{equation}
        [T^*]_\beta^\alpha = \mleft( [T]_\alpha^\beta \mright)^*
    \end{equation}
\end{theorem}



\begin{example}[Least Square]
    For a list of points $(x_i, y_i)$, we want to find $x_0$ that $\norm{y - Ax_0}$ is minimum. $Ax$ is a subspace, so we want to project $y$ on it and find the projection. If $Ax_0$ is the projection, then $y - Ax_0 \perp Ax$, so $\innerproduct{Ax}{y - Ax_0} = 0$. so $\innerproduct{x}{A^* (Ax_0 - y)} = 0$ for all $x$, which means $A^* (Ax_0 - y) = 0$.
\end{example}

\begin{example}[Minimal Solution]
    For all solutions to $Ax=b$, we want to find the solution with minimal norm $\norm{s}$. Define $W = \rangespace{L_{A^*}}$ and $W^\perp = \nullspace{L_A}$. Because $W \perp W^\perp$, we have $x = s + y$ where $s \in W$ and $y \in W^\perp$. So $b = Ax = As + Ay = As$ and $s$ is a solution.
    
    To prove $s$ is minimal, use the property that if $s \perp y$, then $\norm{s+y}^2 = \norm{s}^2 + \norm{y}^2$. So $s$ is the only solution to $Ax=b$ that lies in $\rangespace{L_{A^*}}$.
    
    What's more, if $(AA^*)u = b$, then $s = A^* u$. This is because $s = A^* u \in \rangespace{L_{A^*}}$ and $As = AA^*u = b$. $s$ is the only vector with such property.
\end{example}

Take away from least square and minimal solution example: we try to find solution to $Ax = b$. If there is no solution, we will try to minimize $\norm{y - A x_0}$. If it has solution, it may have many solutions. So we want to find a solution with minimum norm. And this solution is the only solution in $\rangespace{L_{A^*}}$.




% normal
\section{Normal Operators}

Diagonalizable matrix has many good properties so we like to find them. We want to find the answer to these questions:
\begin{enumerate}
    \item When will a matrix be diagonalizable?
    \item Could its eigenvector be an orthonormal basis?
\end{enumerate}

The answer is yes. It obviously puts requirement on both the $V$ and $T$. It turns out that the $T$ must be normal (in $\complexnumber$) or self-adjoint (in $\realnumber$). So we prefer complex normal and real symmetric matrices. The operators here are complex normal or real symmetric.

\begin{definition}[Normal]\label{normal_definition}
    $T$ is \cindex{normal} if $TT^* = T^* T$. $A$ is normal if $AA^* = A^* A$.
\end{definition}

A common mistake is to think that $T$ is normal if and only if $\coordinate{T}_\beta$ is normal. \emph{$\beta$ must be orthonormal} according to \theoref{operator_conjugate_transpose_requirement}.

\begin{theorem}
    $T$ will have many nice properties if it is normal:
\begin{itemize}
    \item $\norm{T(x)} = \norm{T^* (x)}$
    \item $T - cI$ is normal
    \item If $(T - \lambda I)x = 0$, then $(T^* - \conjugate{\lambda}I)x = 0$. So $T$ and $T^*$ share the same eigenvector but the eigenvalues are conjugate
    \item If $a$ and $b$ are eigenvector for eigenvalue $\lambda_1$ and $\lambda_2$, then $a \perp b$\label{normal_eigenvectors_are_orthogonal}
\end{itemize}
\end{theorem}
\begin{proof}
    \begin{equation*}
        \norm{T(x)}^2 = \innerproduct{T(x)}{T(x)} = \innerproduct{T^*T(x)}{x} = \innerproduct{TT^*(x)}{x} = \innerproduct{T^*(x)}{T^*(x)} = \norm{T^*(x)}^2
    \end{equation*}
    \begin{equation*}
        (T-cI)(T-cI)^* = (T-cI)(T^* - \conjugate{c}I) = ... = (T-cI)^*(T-cI)
    \end{equation*}
    Because $T-cI$ is normal when $T$ is normal, so
    \begin{equation*}
        0 = \norm{(T-cI)x} = \norm{(T-cI)^* x} =  \norm{(T^*- \conjugate{\lambda} I)x}
    \end{equation*}
    which means $(T^*- \conjugate{\lambda} I)x = 0$. So $\conjugate{\lambda}$ is an eigenvalue of $T^*$ and $x$ is its eigenvector.
    \begin{equation*}
        \lambda_1 \innerproduct{a}{b} = \innerproduct{\lambda_1 a}{b} = \innerproduct{T(a)}{b} = \innerproduct{a}{T^*(b)} = \innerproduct{a}{\conjugate{\lambda_2}b} = \lambda_2 \innerproduct{a}{b} \Rightarrow (\lambda_1 - \lambda_2)\innerproduct{a}{b} = 0
    \end{equation*}
    Since $\lambda_1 \neq \lambda_2$, we have $\innerproduct{a}{b} = 0$ and $a \perp b$.
\end{proof}


\begin{theorem}
    Let $T$ be a linear operator. We have
    \begin{equation}
        \begin{aligned}
            \rangespace{T^*}^\perp &= \nullspace{T} \\
            \rangespace{T}^\perp &= \nullspace{T^*} \\
            \rangespace{T^*} &= \nullspace{T}^\perp \\
            \rangespace{T} &= \nullspace{T^*}^\perp \
        \end{aligned}        
    \end{equation}
    It means $\rangespace{T^*}$ and $\nullspace{T}$ are orthogonal to each other and they are the direct sum of $V$. This formula is \emph{very useful} because it find two perpendicular subspaces from $A$.
    
    Be careful that in general $\rangespace{T}^\perp \neq \nullspace{T}$. They are equal when $T$ is normal. So
    \begin{equation}
        T \text{ is normal } \Rightarrow \rangespace{T}^\perp = \nullspace{T}
    \end{equation}    
\end{theorem}
\begin{proof}
    For $x \in \nullspace{T}$ and $y \in V$, we have $T^*y \in \rangespace{T^*}$, we have $\innerproduct{T^*y}{x} = \innerproduct{y}{Tx} = \innerproduct{y}{0} = 0$. So $x \perp T^* y$.
    
    
    If $T$ is normal, for $x \in \nullspace{T}$, $\norm{T(x)} = \norm{T^*} = 0$, so $T^*(x) = 0$ and $x \in \nullspace{T^*}$.
\end{proof}


Now we come the most important theorem of this section.

\begin{theorem}\label{normal_orthonormal_with_complex_space}
    For finite-dimension inner product space $V$ on $F=\complexnumber$,
    \begin{equation}
        T \text{ is normal } \Leftrightarrow \text{there exists orthonormal basis that are eigenvectors of } T
    \end{equation}
    
    Note that it does not work for infinite dimensional complex inner product space.
\end{theorem}
\begin{proof}
    If such basis exists, the coordinates of $T$ and $T^*$ are diagonal matrix, so they commute. We do not need the complex feature here. The difficult is on the other side of proof and we need the complex property.
    
    Because complex polynomials always split, by \theoref{schur_theorem}, there is orthonormal $\beta$ that $A=\coordinate{T}_\beta$ is an upper triangular matrix. Assume the first $k-1$ columns are eigenvectors, let's see what happens to $v_{k}$. Since $A$ is upper triangular, we have
    \begin{equation*}
        T(v_k) = A_{1k} v_1 + A_{2k} v_2 + ... + A_{kk} v_k
    \end{equation*}
    Because $\beta$ is orthonormal, we could directly calculate $A_{jk}$:
    \begin{equation*}
        A_{jk} = \innerproduct{T(v_k)}{v_j} = \innerproduct{v_k}{T^*(v_j)} = \innerproduct{v_k}{\conjugate{\lambda_j} v_j} = \lambda \innerproduct{v_k}{v_j} = 0
    \end{equation*}
    
    The normal property that $v_j$ is the eigenvector with eigenvalue $\conjugate{\lambda_j}$ is used above.
    
    Because $T$ is normal, $\transpose{\lambda}$ is an eigenvalue of $T^*$. Although $T^*$ always have eigenvalue if $T$ has eigenvalue, the eigenvalue $v_j$ of $T$ may not be the eigenvalue of $T^*$.
    
    So the result is a diagonal matrix and $\beta$ are all eigenvectors.
\end{proof}

\begin{theorem}
    Normal matrix in $\complexnumber$ is always diagonalizable.
\end{theorem}


Being both orthonormal and eigenvector means there is a diagonal of eigenvalues $D$ that $D = Q^* A Q$ where $Q^* = \inverse{Q}$.


\section{Self-adjoint Operators}

\theoref{normal_orthonormal_with_complex_space} did not work for $F=\realnumber$ because polynomials do not always split. So we need stronger restriction to make it split in $\realnumber$, which is self-adjoint.

\begin{definition}[\cindex{Hermitian},\cindex{self-adjoint}]
    $T$ is Hermitian (self-adjoint) if $T = T^*$. So is it for $A$.
\end{definition}

\begin{theorem}
    If $T$ is self-adjoint, it has even better properties:
\begin{enumerate}
    \item Every eigenvalue of $T$ is real, regardless of $F$ is $\complexnumber$ or $\realnumber$
    \item If $V$ is on $F=\realnumber$, $T$ still splits
\end{enumerate}
\end{theorem}
\begin{proof}
    \begin{equation*}
        \lambda x = Tx = T^* x = \conjugate{\lambda} x
    \end{equation*}
    So $\lambda = \conjugate{\lambda}$ and $\lambda$ is real.
    
    For second property, take $F=\complexnumber$ and let $T$ splits in $\complexnumber$. But all eigenvalues are real, so $T$ splits in $\realnumber$.
\end{proof}

\begin{theorem}
    For finite-dimension inner product space $V$ on $F=\realnumber$, 
    \begin{equation}
        T \text{ is self-adjoint} \Leftrightarrow \text{there exists orthonormal basis that are eigenvectors of } T
    \end{equation}
\end{theorem}
\begin{proof}
    If $T$ is self-adjoint, it splits. So we apply \theoref{schur_theorem} to find orthonormal basis $\beta$ that $A = \coordinate{T}_\beta$ is upper triangular. But
    \begin{equation*}
        A^* = \mleft(\coordinate{T}_\beta\mright)^* = \coordinate{T^*}_\beta = \coordinate{T}_\beta = A
    \end{equation*}
    
    So $A$ is a diagonal matrix and $\beta$ are eigenvectors.
\end{proof}

A quick summary of important theorems:
\begin{itemize}
    \item If $T$ splits, there is $\beta$ that $\coordinate{T}_\beta$ is upper triangular
    \item $\beta$ could be orthonormal using Gram-Schmidt process
    \item $\beta$ could be orthonormal and eigenvectors iff $T$ in $\complexnumber$ is normal ($TT^* = T^*T$)
    \item $\beta$ could be orthonormal and eigenvectors iff $T$ in $\realnumber$ is self-adjoint ($T = T^*$)
    \item normal means diagonalizable in $\complexnumber$
    \item symmetric means diagonalizable in $\realnumber$
\end{itemize}


So the conclusion is, if the matrix $A$ is complex normal or real symmetric, there is orthonormal basis of eigenvectors of $A$.

% unitary and orthogonal operators
\section{Unitary and Orthogonal Operators}

Adjoint operation ($T^*$ and $T$) works pretty like conjugate in $\complexnumber$. There is a special $i \in C$  that $i \conjugate{i} = 1$, so we would like to examine what happens when $TT^* = T^* T = I_n$. It turns out that these $T$ could preserve length.

\begin{definition}\label{unitary}
    If $\norm{T(x)} = \norm{x}$, $T$ is called \cindex{unitary} if $F=C$ and \cindex{orthogonal} if $F=R$.
\end{definition}


\begin{theorem}\label{normal_zero_operator}
    If $T$ is normal and $\innerproduct{x}{Tx} = 0$ for all $x$, then $T=0$.
\end{theorem}
\begin{proof}
    The trick is in using $TT$:
    \begin{equation*}
        0 = \innerproduct{x + T(x)}{T(x + T(x))} = 2 \innerproduct{T(x)}{T(x)}
    \end{equation*}
    So $\innerproduct{Tx}{Tx} = 0$ for all $x$, and $T = 0$
\end{proof}


\begin{theorem}    
If $T$ is unitary, it has many good properties. All the followings are equivalent:
\begin{enumerate}
    \item $T^*T = I$
    \item $TT^* = I$
    \item $\innerproduct{T(x)}{T(y)} = \innerproduct{x}{y}$
    \item If $\beta$ is an orthonormal basis, then $T(\beta)$ is also an orthonormal basis
    \item There exists an orthonormal basis $\beta$ that $T(\beta)$ is also an orthonormal basis
    \item $\norm{T(x)} = \norm{x}$
\end{enumerate}
\end{theorem}
\begin{proof}
    Assume $TT^* = I$
    \begin{equation*}
        \innerproduct{T(x)}{T(y)} = \innerproduct{x}{T^*Ty} = \innerproduct{x}{y}
    \end{equation*}
    
    If $\beta = \set{v_i}$ are orthonormal basis, we have $\innerproduct{T(v_i)}{T(v_j)} = \innerproduct{v_i}{v_j} = \delta_{ij}$, so $T(\beta)$ is also an orthonormal basis.
    
    If $\norm{T(x)} = \norm{x}$, we have 
    \begin{equation*}
        \innerproduct{x}{x} = \innerproduct{T(x)}{T(x)} = \innerproduct{x}{T^*Tx}
    \end{equation*}
    which means $\innerproduct{x}{(T^*T - I)x} = 0$. Because $T^*T - I$ is normal, Theorem~\ref{normal_zero_operator} shows $T^*T = I$
\end{proof}

So unitary (or orthogonal) means $\inverse{T} = T^*$.

These properties shows that if $T$ is unitary, it preserves inner product ($\innerproduct{T(x)}{T(y)} = \innerproduct{x}{y}$) and length ($\norm{T(x)} = \norm{x}$).


If $AA^* = I$, the rows of $A$ are orthonormal basis for $F^n$. Also if $A^*A = I$, then the columns are orthonormal basis.


\begin{theorem}
    If $V$ is on $\realnumber$, it has an orthonormal basis of eigenvectors with $\absolutevalue{\lambda_i} = 1$ if and only if $T$ is both self-adjoint and orthogonal. So $T = T^*$ and $TT^* = I$.
    
    If $V$ is on $\complexnumber$, $T$ needs to be unitary.
    
    In another term, if $T$ is in $\complexnumber$, then $T^* = \inverse{T}$. But if $T$ is in $\realnumber$, then $T = T^* = \inverse{T}$.
\end{theorem}
\begin{proof}
    If $V$ has orthonormal basis with eigenvectors of $T$, $T$ is self-adjoint (or normal). So $T(T^* v_i) = T (\lambda_i v_i) = \lambda_i (T v_i) = \lambda_i^2 v_i$. Because $\absolutevalue{\lambda_i} = 1$, we have $TT^* v_i = v_i$, so $TT^* = I$ and $T$ is orthogonal.
    
    Then for the $\realnumber$ case. If $T$ is self-adjoint, there exists orthonormal basis of eigenvectors of $T$. Now if $T$ is also orthogonal, we have $\absolutevalue{\lambda_i} \norm{v_i} = \norm{\lambda_i v_i} = \norm{T (v_i)} = \norm{v_i}$, so $\absolutevalue{\lambda_i} = 1$. 
    
    For the $\complexnumber$ case, we have the same proof for $\complexnumber$.
\end{proof}


\begin{definition}[\cindex{unitarily equivalent}]
    If $D$ and $A$ are similar, we have $D = \inverse{Q} A Q$. If, in addition, $\inverse{Q} = Q^*$, $D$ and $A$ are called unitarily equivalent (or \cindex{orthogonally equivalent}). Here we did not require $A$ to be unitary or orthogonal.
\end{definition}


\begin{theorem}
    The matrix with orthonormal basis is unitary.
\end{theorem}


With the definition of unitarily equivalent, the necessary and sufficient condition of being normal could be rephrased:
\begin{theorem}
    Let $A$ in $\complexnumber$. $A$ is normal if and only if $A$ is unitarily equivalent to a diagonal matrix. If $A$ is in $\realnumber$, then the condition is orthogonally equivalent.
\end{theorem}
\begin{proof}
    If $A$ is normal, there is an orthonormal basis $Q$ that $A = Q^{-1} D Q$. Because $Q$ is unitary, we have $Q^{-1} = Q^*$. 
    For the reverse, let $A=Q^* D Q$, we have $AA^* = P^* D D^* P = A^* A$.
\end{proof}

The relationship among normal, unitary, unitarily equivalent are complex:
\begin{enumerate}
    \item If $A$ is normal, there is diagonal $D$ that $D = \inverse{Q} A Q$. $Q$ here is unitary
    \item If $A$ is normal and unitary, (unitary means normal), there is still such $D$ and $Q$, with $\absolutevalue{\lambda_i} = 1$
    \item If $D = Q^* A Q$ and $Q$ is unitary, $A$ and $D$ are unitarily equivalent
\end{enumerate}

So the unitary is sometimes on $A$ and sometimes on $Q$.


% rigid motion
\section{Rigid Motion}

Rigid motion analyze how to move a thing without changing its shape.

\begin{definition}[\cindex{rigid motion}]
    Let $V$ be real inner product space. A function $f$ is called rigid motion if 
    \begin{equation}
        \norm{f(x) - f(y)} = \norm{x - y}
    \end{equation}
\end{definition}

One rigid motion example is \cindex{rotation}:
\begin{equation}
    \begin{pmatrix}
        \cos \theta & - \sin \theta \\
        \sin \theta & \cos \theta
    \end{pmatrix}
\end{equation}

Another example is orthogonal operator because $\norm{T(x-y)} = \norm{x - y}$.


We have two more example of rigid motion: reflection and translation.

\begin{definition}[\cindex{reflection}]
    Let $L$ be a line in $R^2$ that passes through the original. $T$ is called reflection about $L$ if $T(x) = x$ for $x \in L$ and $T(x) = -x$ for $x \in L^\perp$. 
    
    If $L$ has angle $\theta$, the reflection $T$ has matrix representation:
\begin{equation}
    \begin{pmatrix}
        \cos 2\theta & \sin 2\theta \\
        \sin 2\theta & -\cos 2\theta
    \end{pmatrix}
\end{equation}
\end{definition}

Reflection and rotation are unitary operator because $\norm{T(x)} = \norm{x}$.

\begin{definition}[\cindex{translation}]
    For a vector $v$, a translation $T$ is $T(x) = x + v$ for all $x$. Translation is a rigid motion.
\end{definition}

A orthogonal operator is a rigid motion, and a translation is a rigid motion. So what happens if we combine the two?

\begin{theorem}
    Let $f$ be a rigid motion. Then there exists a unique orthogonal operator $T$ and a unique translation $g$ that $f = g  \circ T$.
\end{theorem}
\begin{proof}
    Define $T(x) = f(x) - f(0)$ and $g(x) = x + f(0)$. Then $\norm{T(x)} = \norm{f(x) - f(0)} = \norm{x - 0} = \norm{x}$.
    
    The next step is to prove $T$ is a linear transformation. Prove it by proving 
    \begin{equation*}
        \norm{T(x + a y) - T(x) - aT(y)}^2 = 0
    \end{equation*}
    
    If $T$ is linear and preserves inner product, it is orthogonal.
\end{proof}

\begin{theorem}
    It turns out there are only two orthogonal operators in $R^2$. Assume $\beta$ is the standard basis of $R^2$ and let $A = \coordinate{T}_\beta$:
\begin{enumerate}
    \item rotation, with $\determinate{A} = 1$
    \item reflection, with $\determinate{A} = -1$
\end{enumerate}

So in $R^2$, a rigid motion is either a rotation or reflection followed by a translation.
\end{theorem}
\begin{proof}
    Since $T$ is orthogonal, $T(\beta)$ is orthonormal. Since $T(e_1)$ is an unit vector, assume $T(e_1) = (\cos \theta, \sin \theta)$, the only choice for the other vector is $T(e_2) = (- \sin \theta, \cos \theta)$ or $T(e_2) = (\sin \theta, - \cos \theta)$.
    
    If $T(e_2) = (- \sin \theta, \cos \theta)$, $A = \begin{pmatrix}
                \cos \theta & - \sin \theta \\
        \sin \theta & \cos \theta
    \end{pmatrix}$, so it is a rotation by $\theta$.
    
    If $T(e_2) = (\sin \theta, - \cos \theta)$, $A = \begin{pmatrix}
        \cos \theta & \sin \theta \\
        \sin \theta & -\cos \theta
    \end{pmatrix}$, so it is a reflection along a line $L$ with $\displaystyle \frac{\theta}{2}$.
\end{proof}


\begin{example}[\cindex{quadratic form}]
    Now consider the quadratic form $ax^2 + 2bxy + cy^2$. If we let 
\begin{equation}
    A = \begin{pmatrix}
        a & b \\
        b & c
    \end{pmatrix} \text{ and } X = \begin{pmatrix}
        x \\
        y
    \end{pmatrix}
\end{equation}

The quadratic form becomes $X^t A X = \innerproduct{AX}{X}$. Because $A$ is symmetric, there exists an orthogonal matrix $P$ that $D = P^t A P$. Now let $X^\prime = P^t X$, we have $PX^\prime = PP^tX = X$. So $X^t A X = (PX^\prime)^t A (PX^\prime) = {X^\prime}^t (P^t A P) X^\prime = {X^\prime}^t D X^\prime = \lambda_1 (x_1^\prime)^2 + \lambda_2 (x_2^\prime)^2$. Now the quadratic form is simplified.

$P$ may not be the final result because it is possible that $\determinate{P} = -1$. In this case, we could exchange the column of $P$. Now $\lambda_1$ and $\lambda_2$ need to switch order.

So the algorithm to simplify a quadratic form is:
\begin{enumerate}
    \item Change from quadratic form to $X^t A X$
    \item Find the eigenvalues $\lambda_i$ and eigenvectors $x_1$ and $x_2$
    \item Make all eigenvectors unit vector and construct unitary matrix $P$
    \item Define $X = P X'$ and replace $X$ in $X^t A X$, the result would be $X'^t D X'$
    \item If $\determinate{P} = -1$, exchange $x_1$ and $x_2$
    \item The result is a rotation with angle $\arccos P_{11}$
\end{enumerate}
\end{example}

\begin{example}
    If $A$ is an self-adjoint matrix, there is an orthogonal matrix $P$ that $\transpose{P} A P = \Sigma$. So $A = P \Sigma \transpose{P}$. Therefore the effect of $A \times B$ is to
    \begin{enumerate}
        \item Rotate $B$ by $\inverse{P}$
        \item Scale $B$ by $\Sigma$
        \item Rotate $B$ back by $P$
    \end{enumerate}
\end{example}



% spectral theorem
\section{Spectral Theorem}


If $T$ is an orthogonal projection on $W$, we may choose orthonormal basis $\set{v_k}$ of $W$ and expand it to $\beta = \set{v_i}$ of $V$. Then we have
\begin{equation}
    \coordinate{T}_\beta = \begin{pmatrix}
        I_k & 0 \\
        0 & 0
    \end{pmatrix}
\end{equation}

So $T$ is self-adjoint.

The result above could be expanded further:

\begin{theorem}
    Let $T$ be normal or self-adjoint, and with distinct eigenvalue $\set{\lambda_k}$ ($k \leq n$). Let $W_k$ be eigenspace and $T_k$ be the orthogonal projection on $W_k$. We have
    \begin{enumerate}
        \item $V = W_1 \oplus W_2 + ... + W_k$
        \item $W_i^\perp = W_i^\prime$ where $W_i^\prime$ is the direct sum of all the rest eigenspace
        \item $T_i T_j = \delta_{ij}T_i$
        \item $\displaystyle I = \sum_{i=1}^k T_i$
        \item $\displaystyle T = \sum_{i=1}^k \lambda_i T_i$
    \end{enumerate}
    
    $\lambda_i$ is called the spectrum of $T$, $\sum_{i=1}^k T_i$ is called the resolution of the identity operator and $\sum_{i=1}^k \lambda_i T_i$ is called spectral decomposition of $T$.
    
    
    Another view of the theorem is that, since $T$ is normal or self-adjoint, there is an orthonormal basis $\beta$ that contains the eigenvectors of $T$, which means $\coordinate{T}_\beta$ is
    \begin{equation}
        \begin{aligned}
        \coordinate{T}_\beta = \begin{pmatrix}
            \lambda_1 I_{m_1} & 0 & \cdots & 0 \\
            0 & \lambda_2 I_{m_2} & \cdots & 0 \\
            \vdots & \vdots & \ddots & \vdots \\
            0 & 0 & \cdots & \lambda_k I_{m_k}
        \end{pmatrix} &= \lambda_1 \begin{pmatrix}
        I_{m_1} & 0 \\
        0 & 0
    \end{pmatrix} + \lambda_2 \begin{pmatrix}
        0 & 0 & 0\\
        0 & I_{m_2} & 0 \\
        0 & 0 & 0
    \end{pmatrix} + \cdots \\
    &= \lambda_1 \coordinate{T}_{\beta_1} + \lambda_2 \coordinate{T}_{\beta_2} + \cdots
    \end{aligned}
    \end{equation}
\end{theorem}
\begin{proof}
    Let $x = x_1 + x_2 + ... + x_k$ where $x_i \in W_i$. We have $T_i (x) = x_i$. So we have
    \begin{equation*}
        I(x) = T_1 (x) + T_2 (x) + ... + T_k (x)
    \end{equation*}
   
    \begin{equation*}
        \begin{aligned}
            T(x) &= T(x_1) + T(x_2) + ... + T(x_k) \\
            &= \lambda_1 x_1 + \lambda_2 x_2 + ... + \lambda_k x_k \\
            &= \lambda_1 T_1 (x) + \lambda_2 T_2 (x) + ... + \lambda_k T_k (x)
        \end{aligned}        
    \end{equation*}
\end{proof}


Some conclusions are:
\begin{theorem}
    Let $F = \complexnumber$. We have:
    \begin{enumerate}
        \item $T$ is unitary if and only if $T$ is normal and $\absolutevalue{\lambda_i} = 1$
        \item $T$ is self-adjoint if and only if $T$ is normal and $\lambda \in \realnumber$
    \end{enumerate}    
\end{theorem}
\begin{proof}
    If $T$ is normal and $\absolutevalue{\lambda_i} = 1$, we have 
    \begin{equation*}
        \begin{aligned}
            TT^* &= (\lambda_1 T_1 + \lambda_2 T_2 + ... + \lambda_k T_k) (\conjugate{\lambda_1} T_1 + \conjugate{\lambda_2} T_2 + ... + \conjugate{\lambda_k} T_k) \\
            &= \absolutevalue{\lambda_1}^2 T_1 + \absolutevalue{\lambda_2}^2 T_2 + ... + \absolutevalue{\lambda_k}^2 T_k \\
            &= T_1 + T_2 + ... + T_k \\
            &= I
        \end{aligned}        
    \end{equation*}
    
    If $\lambda \in R$, we have $T^* = \conjugate{\lambda_1} T_1 + \conjugate{\lambda_2} T_2 + ... + \conjugate{\lambda_k} T_k = T$.
\end{proof}



% positive definite
\section{Positive Definite}

\begin{definition}[\cindex{positive definite}]
    A linear operator $T$ is positive definite if $T$ is self-adjoint and for $x \in V$,
    \begin{equation}
        \innerproduct{T(x)}{x} > 0
    \end{equation}
    
    It is called \cindex{positive semidefinite} if $\innerproduct{T(x)}{x} \geq 0$.
\end{definition}

\begin{theorem}
    $T$ is positive definite (semidefinite) if and only if all eigenvalues are positive (nonnegative)
\end{theorem}
\begin{proof}
    For eigenvector $v_i$ with eigenvalue $\lambda_i$, we have 
    \begin{equation*}
        \innerproduct{Tv_i}{v_i} = \innerproduct{\lambda_i v_i}{v_i} = \lambda_i \norm{v_i} \geq 0
    \end{equation*}
\end{proof}

\begin{theorem}
$T$ is positive definite (semidefinite) if and only if $A$ is positive definite (semidefinite). Here $A = \coordinate{T}_\beta $ where $\beta$ is an orthonormal basis for $V$.
\end{theorem}


\begin{theorem}[\cindex{$\sqrt{T}$}]
   A positive semidefinite $T$ could have many square root, that is $BB = T$, but it only have one \cindex{positive semidefinite square root}. This square root is called \cindex{$\sqrt{T}$}. $\sqrt{T}$ has a property that
   \begin{equation}
       \mleft(\sqrt{T}\mright)^* = \sqrt{T}
   \end{equation}   
\end{theorem}
\begin{proof}
    Since $T$ is self-adjoint, we have orthonormal eigenvectors $\set{v_i}$ of $T$. If $T$ is positive semidefinite, all eigenvalues are non-negative, so we define a function $B$ as
    \begin{equation*}
        B(v_j) = \sqrt{\lambda_j} v_j
    \end{equation*}
    
    We have $BB(v_j) = B\mleft(\sqrt{\lambda_j} v_j\mright) = \lambda_j v_j = T(v_j)$. So $B^2= T$ and $B$ is positive semidefinite.
    
    If we have a such $B$, $B=B^*$ and we have
    \begin{equation*}
        \innerproduct{Tx}{x} = \innerproduct{BB(x)}{x} = \innerproduct{B(x)}{B^*(x)} = \innerproduct{B(x)}{B(x)} \geq 0
    \end{equation*}
    
\end{proof}



\begin{theorem}
    $T$ is positive semidefinite if and only if $T = B^*B$ for some $B$.    
\end{theorem}
\begin{proof}
    If $T$ is positive semidefinite, there is $B$ that $B = B^*$ and $T = BB = BB^*$.
\end{proof}


\begin{theorem}[\cindex{Gram matrix}]\label{tt_positive_semidefinite}
    Let $T:V \rightarrow W$ be a linear transformation, The Gram matrix $TT^*$ and $T^*T$ are positive semidefinite.
\end{theorem}
\begin{proof}
    $\innerproduct{TT^*(x)}{x} = \innerproduct{T^*(x)}{T^*(x)} \geq 0$.
\end{proof}



% SVD
\section{Singular Value Decomposition}

Previous sections examine the property of linear operator. SVD is concerned with linear transformation. So it associated two inner product spaces and two orthonormal bases. The invariants here are singular values which are non-negative, which does not happen for normal operators.

\begin{definition}[\cindex{adjoint of linear transformation}]
    \defiref{adjoint_operator_definition} defines the adjoint of linear operator. Here is the extension to linear transformation.
    
    Let $T:V \rightarrow W$ be a linear transformation, with inner product $\innerproduct{\cdot}{\cdot}_V$ and $\innerproduct{\cdot}{\cdot}_W$. $T^*: W \rightarrow V$ is called adjoint if 
    \begin{equation}
        \innerproduct{v}{T^*(w)}_V = \innerproduct{T(v)}{w}_W
    \end{equation}
\end{definition}

\begin{theorem}
    The adjoint of linear transformation has similar property:
\begin{enumerate}
    \item $\innerproduct{T^*(w)}{v}_V = \innerproduct{w}{T(v)}_W$ (definition is $T^*$, here is $T$)
    \item If $\alpha$ and $\beta$ are orthonormal basis for $V$ and $W$, we have \begin{equation}
        \coordinate{T^*}_\beta^\alpha = \mleft(\coordinate{T}_\alpha^\beta\mright)^*
    \end{equation}
\end{enumerate}
\end{theorem}



\begin{theorem}[\cindex{SVD}]\label{svd_theorem}
    Let $T: V \rightarrow W$ be a linear transformation with rank $r$. We have:
    \begin{enumerate}
        \item Orthonormal eigenvectors $\set{v_1, v_2, ..., v_n}$ for $T^*T$ for each eigenvalue $\lambda_{i\leq n}$
        \item Orthonormal eigenvectors $\set{w_1, w_2, ..., w_m}$ for $TT^*$
        \item Eigenvalue $\lambda_i$ for $T^*T$ (or $TT^*$) that
                \begin{itemize}
                    \item $\lambda_1 \geq \lambda_2 \geq ... \geq \lambda_r > 0$
                    \item $\lambda_{r+1} = ... = \lambda_n = 0$
                \end{itemize}
        \item $\sigma_i = \sqrt{\lambda_i}$
        \item Orthonormal basis $\set{w_i}$ for $W$ that $\displaystyle w_i = \frac{1}{\sigma_i} T(v_i)$ for $i \leq r$
    \end{enumerate} that 
    \begin{equation}
        T(v_i) = \begin{cases}
        \begin{aligned}
            &\sigma_i w_i & \text { , if } & 1 \leq i \leq r \\
            & 0 & \text{ , if } & i > r
        \end{aligned}            
        \end{cases}
    \end{equation}
\end{theorem}
\begin{proof}
    \theoref{tt_positive_semidefinite} says $T^*T$ is positive semidefinite, so there exists such $\lambda_i$. Because it is semidefinite, some $\lambda_i$ are zero. The number of positive $\lambda_i$ is the rank $r$.
    
    For $w_{i\leq r}$, we have
    \begin{equation}
        \begin{aligned}
            \innerproduct{w_i}{w_j}_{i,j \leq r} = \innerproduct{\frac{1}{\sigma_i} T(v_i)}{\frac{1}{\sigma_j} T(v_j)} &= \frac{1}{\sigma_i \sigma_j} \innerproduct{T(v_i)}{T(v_j)} \\
             &= \frac{1}{\sigma_i \sigma_j} \innerproduct{T^*T(v_i)}{v_j} \\
             &= \frac{1}{\sigma_i \sigma_j} \innerproduct{\lambda_i v_i}{v_j} \\
             &= \frac{\sigma_i^2}{\sigma_i \sigma_j} \innerproduct{v_i}{v_j} = \delta_{ij}
        \end{aligned}        
    \end{equation}
    
    So $\set{w_i}_{i \leq r}$ are orthonormal (prove the orthonormal property of $w_i$), and we can extent it to $\set{w_m}$.
\end{proof}


\begin{definition}[\cindex{singular value}]
    $\set{v_i}$ might be different because there are multiple choices of orthonormal, but $\sigma_i$ are always the same. $\sigma_i$ are called the singular value of $T$.
\end{definition}


The singular value of $T^*$ and $T$ are the same, with reversed $\set{v_i}$ and $\set{w_i}$. This is because $\innerproduct{T^*(u_i)}{v_j} = \innerproduct{u_i}{T(v_j)}$.


The eigenvalue of $TT^*$ and $T^*T$ are the same, so the order does not matter.

\begin{theorem}
    For $A_{m \times n}$ matrix with rank $r$, there are positive singular value $\sigma_1 \geq \sigma_2 \geq ... \geq \sigma_r$. Define $\Sigma$ as
    \begin{equation}
        \Sigma = \begin{pmatrix}
            \sigma_1 & & & & 0 \\
             & \sigma_2 \\
             &  & \ddots &  \\
             &  &  &\sigma_r \\
            0 & & &  & 0
        \end{pmatrix}_{m \times n}
    \end{equation}
    
    There exists unitary matrix $W_{m \times m}$ and unitary matrix $V_{n \times n}$ that
    \begin{equation}
        A = W \Sigma V^*
    \end{equation}
\end{theorem}
\begin{proof}
    Use SVD on $L_A$, and we have orthonormal basis $v_i$ and $w_j$. Build $V$ using $v_i$ and $W$ using $w_j$ and prove $AV = W \Sigma$ by proving their columns are the same.
\end{proof}

Steps of calculating the SVD for $T:V \rightarrow W$:
\begin{enumerate}
    \item Find orthonormal basis $\alpha$ of $V$ and $\beta$ of $W$
    \item Find $A = \coordinate{T}_\alpha^\beta$
    \item Find the eigenvalue $\lambda_i$ of $A^*A$, and its corresponding eigenvectors in $F^n$. Find $V$
    \item Calculate the $\sigma_i = \sqrt{\lambda_i}$. Find $\Sigma$
    \item Convert from eigenvectors in $F^n$ representation into vectors $v_i$ in $V$
    \item Calculate $\displaystyle w_i = \frac{1}{\sigma_i} T(v_i)$ for $i \leq r$. Find remaining $W$ by adding orthonormal vectors $w_{i > r}$
\end{enumerate}

\begin{theorem}
    If $A$ is real symmetric and positive definite, then its singular values are the same as eigenvalue, and the left and right singular vectors are the eigenvectors.    
\end{theorem}


For linear transformation $T: V \rightarrow W$, it does not have inverse if it is not invertible. The psudoinverse catches the part that is invertible.

\begin{definition}[\cindex{pseudoinverse}]
    Let $T: V \rightarrow W$ be a linear transformation, and $L: \nullspace{T}^\perp \rightarrow \rangespace{T}$. The pseudoinverse (\cindex{Moore-Penrose generalized inverse}) $\pseudoinverse{T}$ is defined as the unique linear transformation $\pseudoinverse{T}: W \rightarrow V$ that
    \begin{equation}
        \pseudoinverse{T}(y) = \begin{aligned}
            \begin{cases}
                \inverse{L}(y) & \text{ ,for } y \in \rangespace{T} \\
                0 & \text{ ,for } y \in \rangespace{T}^\perp
            \end{cases}
        \end{aligned}
    \end{equation}
    
    
    For $A_{m \times n}$, we have $A = W \Sigma V^*$ and $\sigma_1 \geq \sigma_2 \geq ... \geq \sigma_r$. Define $\pseudoinverse{\Sigma}$ as 
    \begin{equation}
        \pseudoinverse{\Sigma} = \begin{aligned}
            \begin{cases}
                \displaystyle \frac{1}{\sigma_i} & \text{ ,if } i = j \leq r \\
                0 & \text{ ,otherwise}
            \end{cases}
        \end{aligned}
    \end{equation}
    
    We have $\pseudoinverse{A} = V \pseudoinverse{\Sigma} W^*$.
\end{definition}
\begin{proof}
    Use the SVD. We have orthonormal basis $\set{v_i}$ for $V$ and $\set{w_i}$ for $W$. What's more,
    \begin{itemize}
        \item $\set{v_i, v_2, ..., v_r}$ is the basis for $\nullspace{T}^\perp$
        \item $\set{v_{r+1}, ..., v_m}$ is the basis for $\nullspace{T}$
        \item $\set{w_1, w_2, ..., w_r}$ is the basis for $\rangespace{T}$
        \item $\set{w_{r+1}, ..., w_n}$ is the basis for $\rangespace{T}^\perp$
    \end{itemize}
    
    $L$ is the invertible linear transformation from $\set{v_i, v_2, ..., v_r}$ to $\set{w_1, w_r, ..., w_r}$. Because $L(v_i) = \sigma_i w_i$, we have $\displaystyle w_i = \frac{1}{\sigma_i}L(v_i)$ and $\displaystyle \inverse{L}(w_i) = \frac{1}{\sigma_i} v_i$.
\end{proof}

If $T$ is invertible, $\pseudoinverse{T} = \inverse{T}$.


For a system of linear equations $Ax=b$, we would like to know what is the relation between $\pseudoinverse{A}b$ and the solution.

\begin{theorem}
    Define $T:V \rightarrow W$, we have
    \begin{enumerate}
        \item $\pseudoinverse{T}T$ is the orthogonal projection of $V$ on $\nullspace{T}^\perp$
        \item $T \pseudoinverse{T}$ is the orthogonal projection of $W$ on $\rangespace{T}$
    \end{enumerate}
\end{theorem}
\begin{proof}
    Use the $L$ defined above. For $x \in \nullspace{T}^\perp$, $\pseudoinverse{T}T(x) = \inverse{L}L(x) = x$. For $x \in \nullspace{T}$, $\pseudoinverse{T}T(x) = \inverse{L} 0 = 0$. So $\pseudoinverse{T}T$ is an orthogonal projection.
\end{proof}

\begin{theorem}
    For a system of linear equation $A_{m \times n}x=b$. If $z = \pseudoinverse{A}b$, then $z$ has the following property:
    \begin{enumerate}
        \item If $Ax=b$ has solution, $z$ is the solution with minimum norm
        \item If $Ax=b$ has no solution, $z$ is the best approximation. For any $y$, $\norm{Az-b} \leq \norm{Ay-b}$, with equality holds if and only if $Az = Ay$. And if $Az = Ay$, we have $\norm{z} \leq \norm{y}$
    \end{enumerate}
\end{theorem}

\begin{theorem}
    $\pseudoinverse{A}$ is the unique matrix that satisfies the following 4 properties:
    \begin{enumerate}
        \item $A \pseudoinverse{A}A = A$
        \item $\pseudoinverse{A}A\pseudoinverse{A} = \pseudoinverse{A}$
        \item $\transpose{(A \pseudoinverse{A})} = A \pseudoinverse{A}$
        \item $\transpose{(\pseudoinverse{A}A)} = \pseudoinverse{A}A$
    \end{enumerate}
\end{theorem}


%
%
% conditioning
%
%

\section{Conditioning}


\begin{definition}
    For $Ax=b$, if a small change to $A$ and $b$ cause small change to $x$, the property is called \cindex{well-conditioned}. Otherwise the system is \cindex{ill-conditioned}.
\end{definition}

\begin{definition}
    The \cindex{relative change} in $b$ is $\dfrac{\norm{\dif{b}}}{\norm{b}}$ with $\norm{\cdot}$ be the standard norm on $\mathcal{C}^n$.
\end{definition}

\begin{definition}
    The \cindex{Euclidean norm} of square matrix $A$ is 
    \begin{equation}
        \norm{A} = \max_{x \neq 0} \frac{\norm{Ax}}{\norm{x}}
    \end{equation}
\end{definition}


\begin{definition}
    Let $B$ be a self-adjoint matrix. The \cindex{Rayleigh quotient} for $x \neq 0$ is $R(x) = \dfrac{\innerproduct{Bx}{x}}{\norm{x}^2}$
\end{definition}


\begin{theorem}
For a self-adjoint matrix $B$, the $\displaystyle \max_{x \neq 0} R(x)$ is the largest eigenvalue of $B$ and $\displaystyle \min_{x \neq 0} R(x)$ is the smallest eigenvalue of $B$.
\end{theorem}
\begin{proof}
    Choose the orthonormal basis $v_i$ of $B$ such that $Bv_i = \lambda_i v_i$ where $\lambda_1 \geq \lambda_2 \geq \lambda_n$. $\forall x \in F^n$, $\exists a_i$ that $\displaystyle x = \sum_{i=1}^n a_i v_i$. So
    \begin{equation*}
        R(x) = \frac{\innerproduct{Bx}{x}}{\norm{x}^2} = \frac{\innerproduct{\sum_{i=1}^n a_i \lambda_i v_i}{\sum_{j=1}^n a_j v_j}}{\norm{x}^2} = \frac{\sum_{i=1}^n \lambda_i \absolutevalue{a_i}^2}{\norm{x}^2} \leq \frac{\lambda_1 \sum_{i=1}^n \absolutevalue{a_i}^2}{\norm{x}^2} = \frac{\lambda_1 \norm{x}^2}{\norm{x}^2} = \lambda_1
    \end{equation*}
\end{proof}

\begin{theorem}
    $\norm{A} = \sqrt{\lambda}$ where $\lambda$ is the largest eigenvalue of $A^* A$.
\end{theorem}

\begin{theorem}
    $\lambda$ is an eigenvalue of $A^* A$ if and only if $\lambda$ is an eigenvalue of $AA^*$.
\end{theorem}

\begin{theorem}
    Let $A$ be invertible matrix. Then $\norm{A^{-1}} = \dfrac{1}{\sqrt{\lambda}}$ where $\lambda$ is the smallest eigenvalue of $A^*A$.
\end{theorem}

\begin{definition}
    $\norm{A} \times \norm{A^{-1}}$ is the \cindex{condition number} of $A$ and denoted as $\text{cond}(A)$.
\end{definition}

\begin{theorem}
    For system $Ax=b$ where $A$ is invertible and $b \neq 0$, we have:
    \begin{enumerate}
        \item For any norm $\norm{\cdot}$, we have $\dfrac{1}{\text{cond}(A)} \dfrac{\norm{\dif b}}{\norm{b}} \leq \dfrac{\norm{\dif x}}{\norm{x}} \leq \text{cond}(A) \dfrac{\norm{\dif b}}{\norm{b}}$.
        \item If $\norm{\cdot}$ is the Euclidean norm, then $\text{cond}(A) = \sqrt{\dfrac{\lambda_1}{\lambda_n}}$ where $\lambda_1$ and $\lambda_n$ are the largest and smallest eigenvalue of $A^*A$.
    \end{enumerate}
    
    So when $\text{cond} (b) \geq 1$. If $\text{cond}(b)$ is close to $1$, the relative error in $x$ is small when relative error of $b$ is small. However when $\text{cond}(b)$ is large, the relative error in $x$ could be large or small. 
    
    $\text{cond}(x)$ is seldom calculated because when calculating $A^{-1}$ in computer, there are rounding errors which is related to $\text{cond}(A)$.
\end{theorem}



%
%
% matrix decomposition
%
%

\section{Matrix Decomposition}

The list of famous decomposition:

\begin{table}[H]
\centering
\begin{tabular}[t]{ccc}
\hline
Name & Result & Requirement \\ \hline
Polar & unitary + positive semidefinite & \\
QR & orthogonal + upper triangular & rank $n$ \\
LU & lower triangular + upper triangular & \\
Cholesky & lower triangular & positive semidefinite \\ \hline
\end{tabular}
\end{table}

\begin{theorem}[\cindex{Polar Decomposition}]
    A square matrix $A$ could be expressed as 
    \begin{equation}
        A = WP = W \sqrt{A^*A}
    \end{equation}
    
    Where $W$ is unitary and $P$ is positive semidefinite. If $A$ is invertible, the representation is unique. 
    
    A similar representation in $\complexnumber$ is that for $x \in C$, $\displaystyle x = \frac{x}{\norm{x}} \cdot \norm{x}$. The first part is unitary and the second part is positive.
\end{theorem}
\begin{proof}
    $A = U\Sigma V^* = U V^* V \Sigma V^* = (U V^*) \cdot (V \Sigma V^*)$. Let $W = U V^*$ and $P = V \Sigma V^*$.
\end{proof}


\begin{theorem}[\cindex{QR Decomposition}]
    Let $A_{m \times n} = \rowvector{a_1, a_2, \dots, a_n}$ with $\rank{A} = n$, so $\set{a_i}$ is linearly independent. Use Gram-Schmidt process to form $n$ orthonomal basis:
    \begin{equation*}
        \begin{aligned}
            u_1 &= a_1 & \text{ , } e_1 = \frac{u_1}{\norm{u_1}} \\
            u_2 &= a_2 - \projection{a_2}{u_1} & \text{ , } e_2 = \frac{u_2}{\norm{u_2}} \\
            \dots \\
            u_n &= a_n - \sum_{j=1}^{n-1} \projection{a_n}{u_j} & \text{ , } e_n = \frac{u_n}{\norm{u_n}} 
        \end{aligned}
    \end{equation*}
    Then $\forall k$, $\displaystyle a_k = \sum_{j=1}^k \innerproduct{a_k}{e_k} e_k$. So
    \begin{equation}
        A = QR = [e_1, e_2, \dots, e_n] \times \begin{bmatrix}
            \innerproduct{a_1}{e_1} & \innerproduct{a_2}{e_1} & \innerproduct{a_3}{e_1} & \cdots & \innerproduct{a_n}{e_1}\\
            0 & \innerproduct{a_2}{e_2} & \innerproduct{a_3}{e_2} & \cdots & \innerproduct{a_n}{e_2}\\
            0 & 0 & \innerproduct{a_3}{e_3} & \cdots & \innerproduct{a_n}{e_3} \\            
            \vdots & \vdots & \vdots & \ddots & \vdots \\
            0 & 0 & 0 & 0 & \innerproduct{a_n}{e_n}
        \end{bmatrix}
    \end{equation}
    
    The $Q$ is an orthonormal matrix. $R$ could be calculated by:
    \begin{equation}
        R = Q^\top Q R = Q^\top A
    \end{equation}
\end{theorem}


\begin{theorem}[\cindex{LU Decomposition}]
    For square matrix $A$, it could be composed into two factors: a lower triangular matrix $L$ and an upper triangular matrix $U$ that
    \begin{equation}
        A = LU
    \end{equation}
    
    Here $L$ stands for lower, and $U$ stands for upper.
    
    We could find $L$ and $U$ using the Gaussian elimination.
    
    Sometimes we need to permutate the rows of $A$ so the leading value is nonzero. So we need to multiply a permutation matrix $P$ to $A$ to have $PA = LU$. This is called partial pivoting. If we permutate the columns as well, it is called full pivoting $PAQ = LU$.
\end{theorem}

\begin{theorem}[\cindex{Cholesky Decomposition}]
    For a positive semidefinite matrix $A$, it could be decomposed into 
    \begin{equation}
        A = L L^*
    \end{equation}
    
    where $L$ is a lower triangular matrix.
    
    Cholesky decomposition is roughly twice as efficient as LU decomposition.
\end{theorem}
\begin{proof}
    We do the Gaussian elimination on both rows and columns of $A$. So 
    \begin{equation}
        \Sigma = \transpose{\sqrt{\Sigma}} \sqrt{\Sigma} = L_i ... L_2 L_1 A \transpose{L_1} \transpose{L_2} ... \transpose{L_i}
    \end{equation}
    
    So $L = \transpose{\mleft(\sqrt{\Sigma}L_i ... L_2 L_1\mright)}$.
\end{proof}

























