\chapter{Determinant}


\section{Definition}

\begin{definition}[determinate]
    Define the \cindex{cofactor} of the entry of $A$ in row $i$ and column $j$ as
    \begin{equation}
        \cofactor{A} = (-1)^{i+j}\determinate{\tilde{A}_{ij}}
    \end{equation}
    
    where $\tilde{A}_{ij}$ is a $(n-1)\times (n-1)$ matrix obtained by removing row $i$ and column $j$ of $A$. The \cindex{determinant} of $A$ along row $i$ is recursively defined using \cindex{Laplace expansion}:
    \begin{equation}
    \determinate{A} = \sum_{j=1}^n A_{ij} \times \cofactor{A_{ij}} = \sum_{j=1}^n (-1)^{i+j} A_{ij}  \determinate{\tilde{A}_{ij}}
    \end{equation}
\end{definition}




\section{Properties}

\begin{theorem}
    The volume of $n$ vectors $\set{v_i}$ in $F^n$ is :
    \begin{equation}
      \absolutevaluetext{\begin{vmatrix}
            v_1, v_2, \cdots,  v_n
        \end{vmatrix}}
    \end{equation}
\end{theorem}

\begin{theorem}
    For a linear transformation $T: \realnumber^n \rightarrow \realnumber^n$ with matrix representation $M$. If $\determinate{M} < 1$, the transformation shrink the area. If $\determinate{M} > 1$, it enlarge the areas. if $\determinate{M} = 1$, it preserve the area.
\end{theorem}


\begin{theorem}
    If a matrix has two identical rows, then $\determinate{A} = 0$.
\end{theorem}


\begin{theorem}
    \begin{equation}
        \determinate{AB} = \determinate{A} \times \determinate{B}
    \end{equation}    
\end{theorem}


\begin{theorem}[\cindex{Cramer's Rule}]
    For $A_{n \times n}x=b$, if $\determinate{A} \neq 0$, the solution is:
    \begin{equation}
        x_k = \frac{\determinate{M_k}}{\determinate{A}}
    \end{equation}
    where $M_k$ is obtained by replacing column $k$ of $A$ by $b$.
\end{theorem}
\begin{proof}
    Define $X_k$ as 
    \begin{equation*}
        \begin{pmatrix}
        1 &   &   & x_1 &  \\
          & 1 &   & x_2 &  \\
          &   & \ddots & \vdots \\
          &   &        &  x_k & \\
          &&& \vdots & \ddots \\
          &&& x_n & & 1          
        \end{pmatrix}
    \end{equation*}
    
    We have $A X_k = M_k$, so $\determinate{A X_k} = \determinate{A} \times \determinate{X_k} = \determinate{A} \times x_k = \determinate{M_k}$
\end{proof}

\begin{theorem}
    If $A$ is invertible, then let \cindex{adjugate matrix} be
    \begin{equation}
        \adjugate{A} = (\cofactor{A})^t
    \end{equation}
    
    Then we have
    \begin{equation}
        \inverse{A} = \frac{\adjugate{A}}{\determinate{A}}
    \end{equation}    
\end{theorem}


\begin{theorem}
  For elementary operations:
\begin{itemize}
    \item interchange two rows: $\determinate{B} = - \determinate{A}$ 
    \item multiply a row by $k$: $\determinate{B} = k \determinate{A}$
    \item add a row to another $\determinate{B} = \determinate{A}$
\end{itemize}  
\end{theorem}

\begin{theorem}
    \begin{equation}
        \determinate{A^t} = \determinate{A}
    \end{equation}    
\end{theorem}

\begin{theorem}
    If $A$ is invertible, then:
    \begin{equation}
        \determinate{\inverse{A}} = \frac{1}{\determinate{A}}
    \end{equation}    
\end{theorem}




\begin{definition}
    If $\beta$ and $\gamma$ are two ordered basis for $R^n$, they have the same orientation of and only if $\determinate{Q} > 0$, where $Q$ is change-of-coordinate matrix from $\beta$ to $\gamma$. If $\beta$ is the standard basis, then $\gamma$ is a right-handed coordinate system if $\determinate{Q} > 0$.
\end{definition}


% theory
\section{Theory}

The determinant could be viewed as the only result of special property. First we have to define two things.

\begin{definition}[n-linear function]
    $\delta$ is a n-linear function over $A$ if:
    \begin{equation}
        \delta \begin{pmatrix}
            \vdots \\
            u + kv \\
            \vdots \\
        \end{pmatrix} = \delta \begin{pmatrix}
            \vdots \\
            u \\
            \vdots \\
        \end{pmatrix} + k \times \delta \begin{pmatrix}
            \vdots \\
            v \\
            \vdots \\
        \end{pmatrix} 
    \end{equation}
\end{definition}


\begin{definition}[alternating]
    A n-linear function is alternating if $\delta (A) = 0$ when two adjacent rows of $A$ are identical.
\end{definition}

Then the determinant is the only function $\gamma$ that :
\begin{enumerate}
    \item n-linear
    \item alternating
    \item $\gamma(I_n) = 1$
\end{enumerate}


\begin{theorem}
    Assume $A \in F^{m \times m}$, $B \in F^{m \times n}$, $C \in F^{n \times m}$, $D \in F^{n \times n}$, We have:
    \begin{itemize}
        \item If $A$ is invertible, \begin{equation}
            \begin{vmatrix}
                A & B \\
                C & D
                \end{vmatrix} = \determinate{A} \times \determinate{D - C \inverse{A} B}
        \end{equation}
    \item If $D$ is invertible, \begin{equation}
            \begin{vmatrix}
                A & B \\
                C & D
                \end{vmatrix} = \determinate{D} \times \determinate{A - B \inverse{D} C}
        \end{equation}
    \item If both $A$ and $D$ are invertible, \begin{equation}
        \determinate{D - C \inverse{A} B} = \frac{\determinate{A}}{\determinate{D}} \times \determinate{A - B \inverse{D} C}
        \end{equation}
    \item $\determinate{\lambda I_m - BC} =\lambda^{m-n} \determinate{\lambda I_n - CB}$ ($m \geq n$)
    \end{itemize}
\end{theorem}


\begin{theorem}[Vandermonde]
    The Vandermonde matrix is:
    \begin{equation}
        \begin{vmatrix}
        1 & 1  & \cdots  & 1  \\
        x_1  & x_2 & \cdots & x_n \\
        x_1^2 & x_2^2 & \cdots & x_n^2 \\
        \vdots & \vdots & & \vdots \\
        x_1^{n-1} & x_2^{n-1} & \cdots & x_n^{n-1} \\
        \end{vmatrix} = \prod_{1 \leq j < i \leq n}(x_i - x_j)
    \end{equation}
\end{theorem}
































































































