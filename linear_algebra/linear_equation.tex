\chapter{Linear Equations}

\section{Elementary Operations}

There are 3 \cindex{elementary row operations} on a $m \times n$ matrix $A$:
\begin{enumerate}
    \item interchange any two rows
    \item multiply any row by a nonzero scalar
    \item add any scalar multiple of a row to another row
\end{enumerate}

Each operation has an elementary operation $E$ associated with it.


The \cindex{rank} of a matrix $A$ is the rank of $L_A$. So we are defining the rank of a matrix using the rank of a linear transformation. This is sometimes useful because we could use matrix to calculate the rank of a linear transformation.

By applying a series of elementary row and column operations on $A_{m \times n}$, we could find invertible matrix $B_{m \times m}$ (multiplication of row operations) and $C_{n \times n}$ (multiplication of column operations) that
\begin{equation*}
    D = B A C = \begin{pmatrix}
        I_r & 0 \\
        0 & 0
    \end{pmatrix}
\end{equation*}

We could find the inverse of a matrix using elementary operations. Define an augmented matrix $(A|I_n)$. If we could apply a series of row operation on $A$ and convert it to $I$, then the same operations could convert from $I$ to $\inverse{A}$. So the logic is:
\begin{equation}
    E_p E_{p-1} \hdots E_2 E_1 (A|I_n) = (I_n | \inverse{A} )
\end{equation}


\section{Systems of Linear Equations}

A \cindex{system of linear equations} could be written as 
\begin{equation}
    A x = b
\end{equation}

$A$ is called the \cindex{coefficient matrix}. The equation is \cindex{consistent} if the solution set is non-empty; otherwise it is called \cindex{inconsistent}. If $b=0$, the equation is called \cindex{homogeneous}; otherwise it is \cindex{nonhomogeneous}.

For homogeneous equation $Ax=0$, the solution is $\nullspace{A}$. There are many properties of the solution:
\begin{itemize}
    \item If $m < n$, $A_{m \times n} x = 0$ always has nonzero solution
    \item If $s$ is one solution to $Ax=b$, and $N$ is the solution space of $Ax=0$, then all the solutions are $\set{s} + N$
    \item If $A$ is invertible, then there is only one solution $\inverse{A} b$
\end{itemize}




\begin{definition}
	A matrix is in \cindex{reduced row echelon form} if:
	\begin{enumerate}
		\item any row containing a nonzero entry precedes any row in which all the entries are zero.
		\item the first nonzero entry in each row is the only nonzero entry in its column.
		\item the first nonzero entry in each row is $1$ and it occurs in a column to the right of the first nonzero entry in the preceding row.
	\end{enumerate}
\end{definition}


\begin{theorem}
    $Ax=b$ is consistent if and only if $\rank{A} = \rank{A|b}$
\end{theorem}

If $C$ is invertible, then the solution of $Ax=b$ is the same as $CAx=Cb$. So we could apply a series of elementary row operations to $Ax=b$ until $A$ is in reduced row echelon form, which greatly reduce the calculation effort. 





\begin{theorem}\label{rankoftwomatrix}
    For $A_{m \times n}$ and $B_{n \times p}$, we have:
    \begin{equation}
        \rank{AB} = \rank{B} - \dimension{\nullspace{A} \cap \rangespace{B}}
    \end{equation}
\end{theorem}
\begin{proof}
    Let $\beta_i$ be the basis of $\nullspace{A} \cap \rangespace{B}$, expand to the basis $\beta \cup \alpha$ of $B$. Prove $\alpha$ is a basis of $\rangespace{AB}$.
\end{proof}

\begin{theorem}\label{rankofadjoint}
    For $A_{m \times n}$, we have
    \begin{enumerate}
        \item $\rank{A^\top A} = \rank{A} = \rank{A A^\top}$.
        \item $\rangespace{A^\top A} = \rangespace{A^\top}$.
        \item $\nullspace{A^\top A} = \nullspace{A}$.
    \end{enumerate}
    $A^\top$ could be replaced by $A^*$ in $C$.
\end{theorem}
\begin{proof}
    If $\exists x \neq 0 \left(x \in \nullspace{A^\top} \cap \rangespace{A} \right)$. Then $(A^\top x = 0) \wedge \left(\exists y(x = A y) \right)$. So $x^\top x = y^\top A^\top x = y^\top ( A^\top x) = 0 $ and then $x =0$. According to \thmref{rankoftwomatrix}, $\rank{A^\top A} = \rank{A^\top} - \dimension{\nullspace{A^\top} \cap \rangespace{A}} = \rank{A}$.
\end{proof}









































