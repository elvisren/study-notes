\chapter{Geometric Aspect}

The vectors here are all in $\realnumber^2$ or $\realnumber^3$.

\section{Vectors}

A vector could be expressed as 
\begin{equation}
    \vec{v} = \length{\vec{v}} \angle \theta
\end{equation}

Vector multiplication:
\begin{equation}
    \begin{aligned}
        \vec{u} \cdot \vec{v} &= x_1 y_1 + x_2 y_2 = \length{\vec{u}} \length{\vec{v}} \cos \theta \\
        \vec{u} \times \vec{v} &= \begin{vmatrix}
            i & j & k \\
            u_x & u_y & u_z \\
            v_x & v_y & v_z
        \end{vmatrix}
    \end{aligned}    
\end{equation}

we have
\begin{equation}
    \length{\vec{u} \times \vec{v}} = \length{\vec{u}} \length{\vec{v}} \sin \theta
\end{equation}

For 3 vectors $u, v, w$ in $R^3$, its volume is
\begin{equation}
    \begin{vmatrix}
            u_x & u_y & u_z \\
            v_x & v_y & v_z \\
            w_x & w_y & w_z
    \end{vmatrix} = \vec{u} \cdot (\vec{v} \times \vec{w})
\end{equation}

The direction of $\vec{u} \times \vec{v}$ is perpendicular to both $\vec{u}$ and ${v}$, and it follows the right-hand rule.

In $\realnumber^3$, we have the following cross product:

\begin{table}[H]
\centering
\begin{tabular}[t]{ccc}
    $i \times j = k$ & $j \times k = i$ & $k \times i = j$ \\
    $j \times i = -k$ & $k \times j = -i$ & $i \times k = -j$
\end{tabular}
\end{table}

The projection $\Pi$ of $\vec{v}$ along $\vec{d}$ is
\begin{equation}
    \Pi_{\vec{d}} (\vec{v}) = \left(\frac{\vec{v} \cdot \vec{d}}{\length{\vec{d}^2}}\right) \vec{d}
\end{equation}

\section{Complex Number}

The polar expression of a complex number $z$ is
\begin{equation}
    z = \absolutevalue{z} \angle \theta
\end{equation}

For two complex number $p \angle \theta$ and $q \angle \phi$, their product formula is
\begin{equation}
    p \angle \theta \cdot q \angle \phi = pq \angle (\theta + \phi)
\end{equation}

\begin{theorem}[Euler's formula] The Euler's formula is
\begin{equation}
    e^{i \theta} = \cos \theta + i \sin \theta
\end{equation}

And we have
\begin{equation}
    (\cos \theta + i \sin \theta)^n = \cos n \theta + \sin n \theta
\end{equation}
\end{theorem}


\section{Lines}

\begin{definition}[parametric equation]
    Given a direction vector $\vec{v}$ and some point $p$, the line is defined as 
    \begin{equation}
        l = p + t \vec{v}
    \end{equation}
\end{definition}

\begin{definition}[symmetric equation]
    Another expression is to project $\vec{v}$ on each coordinate:
\begin{equation}
    \begin{aligned}
        x &= p_x + t v_x \\
        y &= p_y + t v_y \\
        z &= p_z + t v_z
    \end{aligned} 
\end{equation}

Or it could be expressed as
\begin{equation}
    \frac{x - p_x}{v_x} = \frac{y - p_y}{v_y}  = \frac{z - p_z}{v_z}
\end{equation}
\end{definition}

\begin{definition}
    A third way to express a line is to treat is as the intersection of two planes:
    \begin{equation}
        \begin{aligned}
            A_1 x + B_1 y + C_1 z &= D_1 \\
            A_2 x + B_2 y + C_2 z &= D_2 \\
        \end{aligned}
    \end{equation}
\end{definition}


%
% planes
%
\section{Planes}
      
\begin{definition}
    A plane could be expressed as
    \begin{equation}
        A x + B y + C z = D
    \end{equation}
\end{definition}

\begin{definition}
    A plane passes a point $p$ and is orthogonal to a vector $\vec{n}$. It could be expressed as 
    \begin{equation}
        \vec{n} \cdot (x - p) = 0
    \end{equation}
\end{definition}

\begin{definition}
    A plane passes a point $p$ and there are two linearly independent vector $\vec{u}$ and $\vec{v}$ that are parallel to the plane. It could be expressed as
    \begin{equation}
        x = p + s \vec{u} + t \vec{v}
    \end{equation}
\end{definition}


%
% distance
%

\section{Distance}

\begin{theorem}
    The distance between two point $p$ and $q$ is $\length{p - q}$.
\end{theorem}

\begin{theorem}
    The distance between the origin and the line $p + t \vec{v}$ is 
    \begin{equation}
        \lengthsmall{p - \frac{p \cdot \vec{v}}{\length{\vec{v}}^2} \vec{v}}
    \end{equation}
\end{theorem}

\begin{theorem}
    The distance between the origin and the plane $\vec{n} \cdot (x - p) = 0$ is    
    \begin{equation}
        \frac{\absolutevalue{\vec{n} \cdot p}}{\length{\vec{n}}}
    \end{equation}
\end{theorem}
































































































































