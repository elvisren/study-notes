\documentclass[reqno, 11pt]{book}


% do not use it. conflict with mtp2font
%\usepackage{amssymb}


\usepackage{latexsym}



% make the alignment of text much better
\usepackage[tracking=true]{microtype}



% draw complex cover page
\usepackage{tikz}
\usetikzlibrary{matrix, positioning, arrows.meta}


% set section level depth and re-number it from 1
%\renewcommand{\thesection}{\arabic{section}}
%\setcounter{secnumdepth}{3}


% math theorem, lemma, proof
\usepackage{amsthm}
\usepackage{thmtools}
\usepackage{mdframed}

% specify color background for theorem, definition
\definecolor{lightblue}{rgb}{0.8, 0.9, 1.0}
\definecolor{lightyellow}{rgb}{1.0, 1.0, 0.8}
\definecolor{lightgray}{rgb}{0.95, 0.95, 0.95}
\definecolor{titlecolor}{rgb}{0.2, 0.2, 0.7}

\declaretheoremstyle[
    spaceabove=2pt,
    spacebelow=2pt,
    headfont=\normalfont\bfseries\color{titlecolor},
    notefont=\normalfont\bfseries,
    notebraces={(}{)},
    bodyfont=\normalfont,
    postheadspace=0.5em,
    mdframed={
        backgroundcolor=lightblue,
        linecolor=lightblue,
        outerlinewidth=2pt,
        topline=false,
        bottomline=false,
        rightline=false,
        innerleftmargin=3pt,
        innerrightmargin=3pt,
        innertopmargin=3pt,
        innerbottommargin=3pt,
        roundcorner=5pt
    }
]{theoremstyle}

\declaretheoremstyle[
    spaceabove=2pt,
    spacebelow=2pt,
    headfont=\normalfont\bfseries\color{titlecolor},
    notefont=\normalfont\bfseries,
    notebraces={(}{)},
    bodyfont=\normalfont,
    postheadspace=0.5em,
    mdframed={
        backgroundcolor=lightyellow,
        linecolor=lightblue,
        outerlinewidth=2pt,
        topline=false,
        bottomline=false,
        rightline=false,
        innerleftmargin=3pt,
        innerrightmargin=3pt,
        innertopmargin=3pt,
        innerbottommargin=3pt,
        roundcorner=5pt
    }
]{definitionstyle}


\declaretheoremstyle[
    spaceabove=2pt,
    spacebelow=2pt,
    headfont=\normalfont\bfseries\color{titlecolor},
    notefont=\normalfont\bfseries,
    notebraces={(}{)},
    bodyfont=\normalfont,
    postheadspace=0.5em,
    mdframed={
        backgroundcolor=lightgray,
        linecolor=lightblue,
        outerlinewidth=2pt,
        topline=false,
        bottomline=false,
        rightline=false,
        innerleftmargin=3pt,
        innerrightmargin=3pt,
        innertopmargin=3pt,
        innerbottommargin=3pt,
        roundcorner=5pt
    }
]{examplestyle}

\declaretheorem[style=theoremstyle]{theorem}
\declaretheorem[style=definitionstyle]{definition}
\declaretheorem[style=examplestyle]{example}
\declaretheorem[style=theoremstyle]{axiom}



% for mathematical integratoin
\usepackage{commath}


% create index
\usepackage{mathtools}
\usepackage{makeidx}
\makeindex


% make the "[" and "]" just as high as the content inside it
\usepackage{mleftright}
\mleftright % make \left work like \mleft


\usepackage{listings}


% no space in itemize
\usepackage{enumitem}
\setenumerate{itemsep=0pt,partopsep=0pt,parsep=\parskip,topsep=2pt}
\setitemize{itemsep=0pt,partopsep=0pt,parsep=\parskip,topsep=2pt}
\setdescription{itemsep=0pt,partopsep=0pt,parsep=\parskip,topsep=2pt}

\setlist{
    after=\vspace{0.5\baselineskip},
}


% do not float the table
\usepackage{float}


% set line space
\usepackage{setspace}

\usepackage[colorlinks,linkcolor=red,anchorcolor=blue,citecolor=green]{hyperref}

% set page size
\usepackage{geometry}
%\geometry{a4paper}
\geometry{a4paper,top=2.5cm,bottom=2.5cm,left=3cm,right=2cm}



% set programming code highlight
\usepackage[chapter]{minted}


% setup bibtex
\usepackage{cite}


% include eps file
\usepackage{epsfig}


% change the color of \emph to blue
\usepackage{xcolor}
\let\oldemph\emph
\renewcommand{\emph}[1]{\textcolor{red}{\oldemph{#1}}}



% user MathTime professional 2. so expensive...
\usepackage{newtxtext}
\usepackage[T1]{fontenc}
\usepackage[subscriptcorrection,slantedGreek,nofontinfo,mtpcal,mtpfrak,mtphrb,zswash]{mtpro2}


% make the contents are aligned on bottom and top within "[" and "]"
\usepackage{adjustbox}



% algorithm
\usepackage{algpseudocode}
\usepackage[section]{algorithm}
\usepackage{algorithmicx}
\usepackage{tikz}
\usepackage{multirow}
\usepackage{float}


% for drawing beautiful math commutative diagram
% note:
%   it treat nodes as a matrix and use "&" to separate row and "\\" for line.
%   "'" change the label from above arrow to below arrow.
%   "l", "r" and "d" means move among matrix nodes.
\usepackage{tikz-cd}



% split a page into multiple columns
\usepackage{multicol}


%
%
% define new commands
%
%

% put cite a color
\newcommand\cindex[1]{\textcolor{blue}{#1}\index{#1}}


\newcommand\mathhilight[1]{\mathop{\bf #1\/}}

\newcommand\theoref[1]{Theorem~\ref{#1} (p\pageref{#1})}
\newcommand\defiref[1]{Definition~\ref{#1} (p\pageref{#1})}
\newcommand\formularef[1]{(\ref{#1}, p\pageref{#1})}


% DTMnow can show the current ISO time
\usepackage{datetime2}

% align the content top and bottom with "[" and "]"
\usepackage{adjustbox}
\newcommand\alignedbrackets[1]{\mleft[#1\mright]}

%\NewDocumentCommand{\alignedbrackets}{m}{
%    \mleft[
%        \adjustbox{valign=c}{\ensuremath{\begin{aligned}#1\end{aligned}}}
%    \mright]
%}

% set and read the build number
\usepackage{ifthen}
\newcommand{\buildnumber}{
    \newread\buildfileread
    \newwrite\buildfilewrite
    \newcounter{buildnumber}
    
    \openin\buildfileread=buildnumber.txt
    \read\buildfileread to \buildnumbertext
    
    \ifthenelse{\equal{\buildnumbertext}{}}{
        \setcounter{buildnumber}{1}
    }{
        \setcounter{buildnumber}{\buildnumbertext}
        \addtocounter{buildnumber}{1}
    }
    
    \immediate\openout\buildfilewrite=buildnumber.txt
    \immediate\write\buildfilewrite{\thebuildnumber}
    \immediate\closeout\buildfilewrite
    
    \arabic{buildnumber}
}

\newcommand\boldmathtext[1]{\mathop{\bf #1\/}}

%
%%%%%%%%%%%%%%%%%%%%%%%%%%%%%%%%%%%%%%%%%%%%%% set theory %%%%%%%%%%%%%%%%%%%%%%%%%%%%%%%%%%%%%%%
%

% natural number, the N
\newcommand\naturalnumber[0]{\mathbb{N}}

% integer, the Z
\newcommand\integer[0]{\mathbb{Z}}

% rational, the Q
\newcommand\rational[0]{\mathbb{Q}}

% real set, the R
\newcommand\realnumber[0]{\mathbb{R}}

% position real set, the R
\newcommand\positiverealnumber[0]{\mathbb{R}^{+}}

% complex set, the C
\newcommand\complexnumber[0]{\mathbb{C}}

% the power set P(X)
\newcommand\powerset[1]{\mathcal{P}\mleft( #1 \mright)}

% open ball, B(x, r)
\newcommand\openball[2]{\mathbb{B}(#1, #2)}

% closed ball, B[x,r]
\newcommand\closedball[2]{\overline{\mathbb{B}}(#1, #2)}

% extended real, R with -inf and inf
\newcommand\extendedrealnumber[0]{\overline{\mathbb{R}}}


% Range f
\newcommand\range[1]{\mathbf{ran}(#1)}

% domain f
\newcommand\domain[1]{\mathbf{dom}(#1)}

\newcommand\allordinals[0]{\mathbf{Ord}}

\newcommand\transfinitesequence[1]{\mleft< #1 \mright>}

% sup {x}
\newcommand\supremum[1]{\mathbf{sup} \set{#1}}

% inf {x}
\newcommand\infimum[1]{\mathbf{inf} \set{#1}}



%
%%%%%%%%%%%%%%%%%%%%%%%%%%%%%%%%%%%%%%%%%%%%%% group theory %%%%%%%%%%%%%%%%%%%%%%%%%%%%%%%%%%%%%%%
%

\newcommand\kernel[1]{\displaystyle \boldmathtext{Ker}~( #1 )}


%
%%%%%%%%%%%%%%%%%%%%%%%%%%%%%%%%%%%%%%%%%%%%%% topology %%%%%%%%%%%%%%%%%%%%%%%%%%%%%%%%%%%%%%%
%

% close of a set
\newcommand\closure[1]{\overline{#1}}

% interior
\newcommand\interior[1]{\mathring{#1}}

% boundary
\newcommand\boundary[1]{\partial#1}





%
%%%%%%%%%%%%%%%%%%%%%%%%%%%%%%%%%%%%%%%%%%%%%% linear algebra %%%%%%%%%%%%%%%%%%%%%%%%%%%%%%%%%%%%%%%
%


% null space, the N(x)
\newcommand\nullspace[1]{\mathcal{N}\mleft(#1\mright)}

% range space, the R(x)
\newcommand\rangespace[1]{\mathcal{R}\mleft(#1\mright)}

% absolute value, the |x|
\newcommand\absolutevalue[1]{\abs{#1}}

% absolute value text, the "abs"
\newcommand\absolutevaluetext[1]{\boldmathtext{abs}~#1}


% determinate, the same as |x|
\newcommand\determinate[1]{\absolutevalue{#1}}

% determinate text, the "det"
\newcommand\determinatetext[1]{\displaystyle \boldmathtext{det}~\displaystyle #1}

% the coordinate, the [x]
%\newcommand\coordinate[1]{\sbr{#1}}
\newcommand\coordinate[1]{\alignedbrackets{\displaystyle #1}}

% the projection of v on S: proj_S (v)
\newcommand\projection[2]{\boldmathtext{proj}_{#2} #1}


\newcommand\rowvector[1]{\mleft[ \displaystyle #1 \mright]}

\newcommand\vectorsymbol[1]{\overrightarrow{#1}}

% inverse, the x^-1
\newcommand\inverse[1]{ {#1}^{-1} }

% dimension text, the "dim x"
\newcommand\dimension[1]{\displaystyle \boldmathtext{dim}~\mleft( #1 \mright)}

% rank text, the "rank x"
\newcommand\rank[1]{\displaystyle \boldmathtext{rank}~\mleft( #1 \mright)}

% adjugate matrix, the "adj x"
\newcommand\adjugate[1]{\displaystyle \boldmathtext{adj}~\mleft( #1 \mright)}

% inner product, the <x,y>
\newcommand\innerproduct[2]{\mleft\langle \displaystyle #1, #2 \mright\rangle}

% trace, the "tr x"
\newcommand\trace[1]{\displaystyle \boldmathtext{tr}~( #1 )}

% span, the "span x"
\newcommand\spanset[1]{\displaystyle \boldmathtext{span}~\mleft( #1 \mright)}

% pseudoinverse, the dagger
\newcommand\pseudoinverse[1]{ {#1}^\dagger }

% the length ||
\newcommand\lengthsmall[1]{ \Vert {#1} \Vert }
\newcommand\length[1]{\enVert{#1}}

% transpose, the T
\newcommand\transpose[1]{ {#1}^\top }

%\newcommand\adjugate[1]{\displaystyle \mathrm{adj}~\displaystyle #1 }
\newcommand\cofactor[1]{\displaystyle \boldmathtext{cof}~\displaystyle #1 }

% column vector
\newcommand\columnvector[1]{\boldsymbol{#1}}

% norm
\newcommand\vectornorm[1]{\displaystyle \mleft\lVert #1 \mright\rVert}

% conjugate, put a line above x
\newcommand\conjugate[1]{\overline{#1}}

% select a row from matrix
\newcommand\selectmatrixrow[2]{{#1}_{{#2},\cdot}}


% select a column from matrix
\newcommand\selectmatrixcolumn[2]{{#1}_{\cdot, {#2}}}


% function restriction T|_subspace
\newcommand\functionrestriction[2]{{#1}|_{#2}}


%
% ----------------------------------------- analysis -----------------------------------------
%

\newcommand\sequence[1]{\intoo{#1}}

\newcommand\realpartofcomplexnumber[1]{\boldmathtext{Re}\text{(}#1\text{)}}

\newcommand\imaginarypartofcomplexnumber[1]{\boldmathtext{Im}\text{(}#1\text{)}}



%
%%%%%%%%%%%%%%%%%%%%%%%%%%%%%%%%%%%%%%%%%%%%%% probability %%%%%%%%%%%%%%%%%%%%%%%%%%%%%%%%%%%%%%%
%

% P[x]
\newcommand\probability[1]{\boldmathtext{P}\mleft[\displaystyle #1 \mright]}
\newcommand\probabilitywithsmallbottom[1]{\boldmathtext{P}\alignedbrackets{\displaystyle #1}}


% E[x]
\newcommand\expect[1]{\boldmathtext{E}\alignedbrackets{\displaystyle #1}}

% Var[x]
\newcommand\variance[1]{\boldmathtext{Var}\mleft[\displaystyle #1 \mright]}
\newcommand\variancewithsmallbottom[1]{\boldmathtext{Var}\alignedbrackets{\displaystyle #1}}


% H[x]: entropy
\newcommand\entropy[1]{\boldmathtext{H}\mleft[\displaystyle #1 \mright]}

% KL divergence
\newcommand\kldivergence[2]{\boldmathtext{HL}\mleft(\displaystyle #1 \middle| \middle| \displaystyle #2 \mright)}


% Cov(a,b)
\newcommand\covariance[2]{\boldmathtext{Cov}\displaystyle \mleft(#1,#2 \mright)}

% exp { xxx }
\newcommand\exponential[1]{\boldmathtext{exp}~\set{#1}}

% N(\mu, \sig^2)
\newcommand\normaldistribution[2]{\mathcal{N}\mleft(#1,#2\mright)}
\newcommand\normaldistributionwithparameter[3]{\mathcal{N}\mleft(#1 \middle| #2,#3\mright)}


% Gamma(\alpha, \beta)
\newcommand\gammadistribution[2]{\boldmathtext{Gamma}~(#1,#2)}

% Beta(\alpha, \beta)
\newcommand\betadistribution[2]{\boldmathtext{Beta}~(#1,#2)}

% Chi-square (k)
\newcommand\chisquaredistribution[1]{\chi^2_{#1}}






%
%%%%%%%%%%%%%%%%%%%%%%%%%%%%%%%%%%%%%%%%%%%%%% machine learning %%%%%%%%%%%%%%%%%%%%%%%%%%%%%%%%%%%%%%%
%

\newcommand\subscription[2]{\boldsymbol{#1}^{(#2)}}

% argmin and argmax
\newcommand\argmin[1]{\mathop{\arg\min}\limits_{#1}}
\newcommand\argmax[1]{\mathop{\arg\max}\limits_{#1}}


% select one sample
\newcommand\selectonesample[2]{{#1}^{(#2)}}

% select one dimension from one sample
\newcommand\selectonedimension[2]{{#1}_{#2}}


% estimation of density
\newcommand\estimation[1]{\hat{#1}}


% select row and column
\newcommand\selectonedimensionfromonesample[3]{{#1}^{(#2)}_{#3}}

% expectation
\newcommand\expectation[1]{\mathbb{E}\mleft[{#1}\mright]}

% expectation on a specific variable
\newcommand\expectationonanvariable[2]{\mathbb{E}_{#1}\mleft[{#2}\mright]}

% bold x
\newcommand\feature[1]{\mathbf{#1}}









































































































































































































































































































































