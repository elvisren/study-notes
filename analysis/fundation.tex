\chapter{Fundation}


\begin{enumerate}
    \item Theorem: an important proposition
    \item Lemma: a proposition before a theorem, and is needed for its proof
    \item Corollary: a proposition follows directly from a theorem
\end{enumerate}


%
% group theory 
%

\section{Group Theory}

\begin{definition}[\cindex{magma}]
    A magma $(M, \cdot)$ has the property that
    \begin{equation}
        a, b \in M \Rightarrow a \odot b \in M
    \end{equation}
\end{definition}

\begin{definition}[\cindex{semigroup}]
    A semigroup is a magma with associativity:
    \begin{equation}
        (a \odot b) \odot c = a \odot (b \odot c)
    \end{equation}
\end{definition}

\begin{definition}[\cindex{monoid}]
    A monoid is a semigroup with identity:
    \begin{equation}
        e \odot a = a
    \end{equation}
\end{definition}

\begin{definition}[\cindex{group}]
    A group is monoid with inverse:
    \begin{equation}
        \forall a, \exists b \Rightarrow a \odot b = e
    \end{equation}
\end{definition}

\begin{definition}[\cindex{Abelian group}]
    An Abelian group is a group with commutativity on $\cdot$:
    \begin{equation}
        a \odot b = b \odot a
    \end{equation}
\end{definition}

\begin{definition}[\cindex{ring}]
    A ring $(R, +, \odot)$ is defined as:
    \begin{itemize}
        \item $+$ is an Abelian group
        \item $\odot$ is a monoid
        \item Distributivity: $a \odot (b + c) = a \odot b + a \odot c$, and $(a+b) \odot c = a \odot c + b \odot c$
    \end{itemize}
    
\end{definition}

\begin{definition}[\cindex{field}]
    A field is a ring that $\odot$ is a group too.
\end{definition}

\begin{definition}[ordered field]
    An ordered field is a field with order $\leq$:
    \begin{itemize}
        \item $(R, \leq)$ is totally ordered
        \item $x < y \Rightarrow x + z < y + z$
        \item $x,y > 0 \Rightarrow xy > 0$
    \end{itemize}
    
    $x \in R$ is positive if $x > 0$.
\end{definition}

\begin{definition}[\cindex{algebra}]
    Let $K$ be a field, and let $A$ be a vector space over $K$ with operation $\cdot : A \times A \rightarrow A$. Then $A$ is an algebra over $K$ if the following identities hold for all elements $x,y,z \in A$, and all n $a,b \in K$:
    \begin{itemize}
        \item Right distributivity: $(x+y) \cdot z = x \cdot z + y \cdot z$
        \item Left distributivity: $z \cdot (x+y) = z \cdot x + z \cdot y$
        \item Compatibility with scalars: $(ax) \cdot (by) = (ab) (x \cdot y)$
    \end{itemize}

    These three axioms are another way of saying that the binary operation is \cindex{bilinear}. An algebra over $K$ is sometimes also called a \cindex{$K$-algebra}, and $K$ is called the \cindex{base field} of $A$.
\end{definition}

\begin{definition}[\cindex{coset}]
    Let $N$ be a subgroup of $G$ and $g \in G$. Then $g \odot N$ is the left coset and $N \odot g$ is the right coset.
    
    The left coset is an equivalent relation so it defines $G/N$ which is the set of left cosets of $G$ modulo $N$.
\end{definition}

\begin{definition}[\cindex{normal}]
    A subgroup $N$ is normal if for all $g \in G$, we have 
    \begin{equation}
        g \odot N = N \odot g
    \end{equation}    
\end{definition}

\begin{definition}[\cindex{quotient group}]
    A normal subgroup $N$ induces a quotient group $(G/N) \times (G/N) \rightarrow G/N$ that 
    \begin{equation}
        (a \odot N, b \odot N) \rightarrow (a \odot b) \odot N
    \end{equation} 
\end{definition}

\begin{definition}[\cindex{homomorphism}]
    Let $G = (G, \odot)$ and $H = (H, \otimes)$ be groups. A function $\varphi: G \rightarrow H$ is a homomorphism if 
    \begin{equation}
        \varphi(g \odot h) = \varphi(g) \otimes \varphi(h)
    \end{equation}
    
    The image of a homomorphism is a subgroup.
\end{definition}

\begin{theorem}
    If $\varphi$ is a homomorphism, then $\varphi(e) = e'$ and $\varphi(g)^{-1} = \varphi(g^{-1})$.
\end{theorem}

\begin{definition}[\cindex{kernel}]
    Let $\varphi$ be a homomorphism. The kernel of $\varphi$, $\kernel{\varphi}$ is defined as
    \begin{equation}
        \kernel{\varphi} = \varphi^{-1}(e')
    \end{equation}
    
    The kernel is a subgroup.
\end{definition}

\begin{theorem}
Let $\varphi$ be a homomorphism and $N = \kernel{\varphi}$, then
\begin{equation}
    g \odot N = \varphi^{-1}\left(\varphi(g)\right)
\end{equation}
\end{theorem}

\begin{theorem}
    Let $N$ be a normal subgroup of $G$. Then the quotient function $p: G \rightarrow G/N$ is a surjective homomorphism, the quotient homomorphism, with $\kernel{p} = N$.
\end{theorem}


\begin{definition}[\cindex{isomorphism}]
    A homomorphism is isomorphism if it is bijective. If it is from $G$ to itself, it is automorphism. Examples are:
    \begin{itemize}
        \item $g: a \odot g \odot a^{-1}$
        \item for surjective homomorphism $\varphi: G \rightarrow H$, $\varphi' : G/\kernel{\varphi} \rightarrow H$
        \item Let $(G, \odot)$ be a group, but $H$ is not. And $\varphi: G \rightarrow H$ is a bijection. Then define a function $\otimes$ that $g' \otimes h' := \varphi^{-1}(g') \odot \varphi^{-1}(h')$. $\otimes$ is the operation on $H$ induced from $\odot$ by $\varphi$
    \end{itemize}
\end{definition}



% numbers
\section{Number System}

When we extend the number system, we lose certain property:
\begin{itemize}
    \item The extension from $\integer$ to $\naturalnumber$ lose the well ordering principle
    \item The extension from $\naturalnumber$ to $\rational$ lose the single representation
    \item The extension from $\rational$ to $\realnumber$ lose the explicit representation
    \item The extension from $\realnumber$ to $\complexnumber$ lose the ordered field structure
\end{itemize}

\subsection{Integer Number}
Natural number $\naturalnumber$ did not support the inverse $-n$. Integer $\integer$ is the smallest ring that contains $(\naturalnumber, +)$. For convenience we have defined the following:
\begin{itemize}
    \item The additive identity is denoted by $0_R$
    \item The multiplicity identity is denoted by $1_R$
    \item The additive inverse of $a$ is $-a$
    \item $0 \cdot a = a \cdot 0 = 0$
\end{itemize}

\subsection{Rational Number}
Integer $\integer$ did not support the $m/n$. Rational number $\rational$ is the smallest field that contains $\integer$. $\rational$ has a limitation too. For example, there is no rational $x$ that $x^2 = 2$. So we need to extend $\rational$ too.

\begin{definition}[\cindex{order complete}]
    Let $X$ be a totally ordered set. It is order complete if every nonempty subset is bounded above has supremum. It is also called complete.
    
    Compared with well-ordering: well-ordering says any set has a minimum.
\end{definition}


\begin{theorem}
    $\rational$ is not order complete.
\end{theorem}
\begin{proof}
    Define $A:=\set{x \in \rational; x > 0 , x^2 < 2}$, and $B:=\set{x \in \rational; x > 0, x^2 > 2}$. Assume there is $c$ that $a \leq c \leq b$, define $\xi = \frac{2c+2}{c+2}$, we have
    \begin{equation}
        \begin{aligned}
            \xi^2 - 2 &= \frac{2(c^2 -2)}{(c+2)^2} \\
            \xi - c &= - \frac{c^2 -2}{c+2}    
        \end{aligned}      
    \end{equation}
    
    If $c^2 < 2$, we have $\xi^2 < 2$ and $\xi < c$, so $\xi \in A$ and $\xi > c$, contradiction.
\end{proof}


\subsection{Real Number}

Real number $\realnumber$ is the order completion of rational number $\rational$. Here we use Dedekind cut to do the completion.

\begin{definition}[\cindex{Dedekind cut}]
    Let $F$ be a field. A Dedekind cut $(L,R)$ on $F$ is a partition of $F$ into two disjoint set $L$ and $R$ that
    \begin{itemize}
        \item $L \neq \emptyset, F$
        \item $L$ is closed downwards: $(y \in L) \wedge (x < y) \Rightarrow x \in L$
        \item $L$ does not have maximum element: $(x \in L) \Rightarrow (\exists y \in L) (x < y)$
    \end{itemize}
\end{definition}
    
\begin{definition}
    For $q \in \rational$, there is a Dedekind cut $L_q = \set{x\in \rational: x < q}$.
\end{definition}

\begin{theorem}
     A Dedekind cut $L$ has a supremum in $\rational$ if and only if $L = L_q = \set{x \in \rational: x < q}$ for some $q\in \rational$.
     
     In general, a Dedekind cut may not have minimum in $\rational$.
\end{theorem}

\begin{theorem}
    A Dedekind cut $L = L_q$  if and only if $U$ has a minimal element.
\end{theorem}


\begin{definition}[\cindex{real number}]
    Let $F=\rational$, the set of all Dedekind cut $C$ on $\rational$ is real number.
    
    The real number is an ordered field with $<$, $\oplus$ and $\otimes$ operation. We first define these operations on Dedekind representation of rationals $C_\rational = \set{L_q: q \in \rational} \subseteq C$.
    \begin{itemize}
        \item $L_p < L_q$ is defined as $L_p \subset L_q$
        \item $L_p \oplus L_q = \set{x + y \in \rational: x < p, y < q}$
        \item $\otimes$ could be defined as well, but need to consider the sign of $p$ and $q$
    \end{itemize}    
    
    Now we \emph{extend the set theory based $<$, $\oplus$ and $\otimes$ operation to all Dedekind cut on $\rational$}. The result is ordered field $\realnumber = \closure{\rational}$.
\end{definition}



\begin{theorem}\label{dedekind_real_order_complete}
    The ordered field of Dedekind cut $\realnumber$ with $<$, $\oplus$ and $\otimes$  is order complete.    
\end{theorem}
\begin{proof}
    For $D \subset \realnumber$ which is bounded above, define a subset $M \subset C$ which is the union of all $d \in D$:
    \begin{equation}
        M = \set{x: x \in d \text{ for some } d \in D} =\bigcup_{d \in D} d
    \end{equation}
    
    So $M$ is also a Dedekind cut. Now prove $M$ is the least upper bound of $D$. Choose any upper bound $Y$ of $D$. for any $x \in X$, $\exists d \in D,  x \in d$. Since $Y$ is an upper bound of $D$, we have $d < Y$. So $X \subset Y$.
\end{proof}

\begin{definition}[\cindex{irrational number}]
    The set $\realnumber - \rational$ is called irrational number.
\end{definition}

\begin{theorem}
    Let $L$ be an irrational Dedekind cut. It is the least upper bound of all rational cuts smaller than itself:
    \begin{equation}
        L = \bigcup_{p \in \rational \wedge p \in L} L_p
    \end{equation}    
\end{theorem}


\begin{theorem}
Let $x \in \realnumber$. It could be expressed as the supremum of the set $x = \sup \set{q \in \rational: q \leq x}$.
\end{theorem}



\begin{theorem}
    Here are the common properties of real numbers:
    \begin{itemize}
        \item For each $x \in \realnumber_{+}$, there is a $n \in \integer$ that $n > x$.
        \item If $\displaystyle 0 \leq a \leq \frac{1}{n}$ for all $n \in \naturalnumber^{\times}$, then $a = 0$
        \item For any $a >0$, there is $n \in \naturalnumber^{\times}$ that $\displaystyle \frac{1}{n} < a$
        \item For any $a,b\in \realnumber$ that $a < b$, there is $q \in \rational$ that $a < q < b$
        \item For any $a,b\in \realnumber$ that $a < b$, there is $r \in \realnumber \textbackslash \rational $ that $a < r < b$
        \item For any $r \in \real$, there is $n \in \naturalnumber$ that $n - 1 \leq r < n$
    \end{itemize}    
\end{theorem}
\begin{proof}
    To prove $\frac{1}{n} < a$, there is $n$ that $n > \frac{1}{a}$. 
    
    To prove $a < q < b$. There is $n \in \naturalnumber$ that $n > \frac{1}{b-a}$, so $na + 1 < nb$. For $na$, there is $m \in \integer$ that $m - 1 \leq na < m$, so $na < m \leq na + 1$. So $na < m \leq na + 1 < nb$, and $a < \frac{m}{n} < b$. Let $r = \frac{m}{n} \in \rational$.
    
    To prove $a < r < b$, find $q_1, q_r \in \rational$ that $a < q_1 < q_2 < b$. Define $\xi = q_1 + \frac{q_2 - q_2}{\sqrt{2}} \in \realnumber - \rational$.
\end{proof}


\begin{example}[decimal expression of rational number]
    For an rational number $r = \frac{p}{q}$, there is $a_0$ that $p = a_0 q + b_1$ where $b_1 < q-1$. This expression is equivalent to 
    \begin{equation}
        \frac{p}{q} = a_0 + \frac{1}{10} \left(\frac{10 b_1}{q} \right)
    \end{equation}
    
    because $0 < \frac{10 b_1}{q} < 10$, so there is $b_2$ that $\frac{10 b_1}{q} = a_1 + \frac{b_2}{q}$, which means
    \begin{equation}
        \frac{p}{q} = a_0 + \frac{1}{10} \left( a_1 + \frac{b_2}{q} \right) = a_0 + \frac{a_1}{10} + \frac{1}{10^2} \left(\frac{10 b_2}{q} \right)
    \end{equation}
    
    The problem with decimal expression is that a number could have more than one representation. For example, $1.4 = 1.3 \dot 9$. The conversion is to define $0.\dot 9 = 1.\dot 0$.
    
    The irrational number has no explicit representation. We could use the floor function instead. For any $r \in \realnumber - \rational$, let $a_0 = \lfloor r \rfloor$, and then $a_1 = \lfloor 10(r-a_0) \rfloor$, etc. The result is $r = a_0 . a_1 a_2 a_3 \cdots$.
\end{example}

\begin{theorem}
    The decimal expression of a rational number $r$ terminates after finitely many terms or it is periodic.    
\end{theorem}
\begin{proof}
    Let $r = \frac{p}{q}$. Assume the decimal expression never terminate. Because all $b_i \in \set{0, \cdots, q -1}$, some $b_i$ must be equal.
\end{proof}





The real number $\realnumber$ still has limitation. It cannot solve $x^2 = -1$. The smallest extension field $\complexnumber$ to $\realnumber$ is the complex number.


% extended real

\subsection{Extended Number Line}

\begin{definition}[\cindex{extended number line}]
    The set $\closure{\realnumber} = \realnumber \cup \set{\pm \infty}$ is the extended number line. 
    
    Defined operations are: $x + \infty = \infty$, $x - \infty = -\infty$, $\displaystyle \frac{x}{\pm \infty} = 0$, $\infty + \infty = \infty$, $- \infty - \infty = - \infty$, $\infty \cdot \infty = \infty$.
    \begin{equation}
        \begin{aligned}
            x \cdot \pm \infty &= \begin{cases}
                \mp \infty \text{, } x > 0 \\
                \pm \infty \text{, } x < 0 \\
            \end{cases} \\
            \frac{x}{0} &= \begin{cases}
                \infty \text{, } x > 0 \\
                -\infty \text{, } x < 0 \\
            \end{cases} \\
        \end{aligned}
    \end{equation}
    
    Undefined operations are: $\infty - \infty$, $0 \times \pm \infty$, $\displaystyle \frac{\pm \infty}{\pm \infty}$, $\displaystyle \frac{0}{0}$, $\displaystyle \frac{\pm \infty}{0}$.
    
    It has the following property:
    \begin{itemize}
        \item $\closure{\realnumber}$ is a totally ordered set
        \item $\closure{\realnumber}$ is not a field
    \end{itemize}
\end{definition}

% complex number
\subsection{Complex Number}

The complex number $\complexnumber$ is the smallest extension field of $\realnumber$ that $x^2 = -1$ is solvable.

\begin{theorem}[\cindex{De Moivre's Identity}]
    For $z = a + b i = r(\cos \theta + i \sin \theta) = r e^{i \theta} \neq 0$ and $n \in \integer$, we have 
    \begin{equation}
        z^n = (a+bi)^n = r^n(\cos \theta + i \sin \theta)^n = r^n (\cos n\theta + i \sin n\theta) = r^n e^{in\theta}
    \end{equation}
\end{theorem}

\begin{theorem}
    For any $z \in \naturalnumber_{+}$ and $w \in \complexnumber$, we have
    \begin{equation}
        z^w = e^{w \ln z}
    \end{equation}
\end{theorem}




























































































































































































































































