\chapter{Fundation}


\begin{enumerate}
    \item Theorem: an important proposition
    \item Lemma: a proposition before a theorem, and is needed for its proof
    \item Corollary: a proposition follows directly from a theorem
\end{enumerate}






% logic and set theory

\section{Logic and Set Theory}


\begin{definition}    
The implication $A \Rightarrow B$ is defined as:
\begin{equation}
    A \Rightarrow B := \neg A \vee B
\end{equation}

Here $A$ is the sufficient condition for $B$, and $B$ is the necessary condition for $A$.

Another observation is:
\begin{equation}
    A \Rightarrow B \Leftrightarrow \neg B \Rightarrow \neg A
\end{equation}

$\neg B \Rightarrow \neg A$ is the contrapositive of the statement $A \Rightarrow B$
\end{definition}

\begin{definition}
    A function $f : X \rightarrow Y$ is a rule that for each element of $X$, there is only one element of $Y$.
\end{definition}

\begin{definition}
    The fiber of $f$ at $y$ is $f^{-1}$ that:
    \begin{equation}
        f^{-1}(y) := \set{x \in X; f(x) = y}
    \end{equation}
\end{definition}

\begin{definition}
    A relation on $X$ which is reflective, transitive and symmetric is called an equivalent relation on $X$, and is denoted as $\sim$.
    
    It defines a surjection $X \rightarrow X / \sim$ which is called quotient function.
\end{definition}



\begin{theorem}
    The natural number $\naturalnumber$ are well ordered, that is, each nonempty subset of $\naturalnumber$ has a minimum.
\end{theorem}


\begin{definition}
    A set $X$ is finite if it is empty, or if there are $n \in \naturalnumber^{\times}$ and a bijection from $\set{1, ..., n}$ to $X$.
    
    If a set is not finite, it is infinite.
    
    A set $X$ is countably infinite if there is a bijection from $X$ to $\naturalnumber$. A set is countable if it is countably infinite or finite. Or it is uncountable.
\end{definition}

\begin{theorem}
    There is no surjection from $X$ to $\powerset{X}$. So $\powerset{\naturalnumber}$ is uncountable.
\end{theorem}
\begin{proof}
    Define a function $\varphi: X \rightarrow \powerset{X}$, and consider the set $A = \set{x\in X, x \notin \varphi(x)}$.
\end{proof}


\begin{theorem}
A countable union of countable sets is countable.    
\end{theorem}

\begin{theorem}
The set $\set{0,1}^{\naturalnumber}$ is uncountable.
\end{theorem}
\begin{proof}
    Define a bijection from $\powerset{\naturalnumber}$ to $\set{0,1}^{\naturalnumber}$.
\end{proof}


\section{Group Theory}

%
% group theory 
%

\begin{definition}[magma]
    A magma $(M, \cdot)$ has the property that
    \begin{equation}
        a, b \in M \Rightarrow a \odot b \in M
    \end{equation}
\end{definition}

\begin{definition}[semigroup]
    A semigroup is a magma with associativity:
    \begin{equation}
        (a \odot b) \odot c = a \odot (b \odot c)
    \end{equation}
\end{definition}

\begin{definition}[monoid]
    A monoid is a semigroup with identity:
    \begin{equation}
        e \odot a = a
    \end{equation}
\end{definition}

\begin{definition}[group]
    A group is monoid with inverse:
    \begin{equation}
        \forall a, \exists b \Rightarrow a \odot b = e
    \end{equation}
\end{definition}

\begin{definition}[Abelian group]
    An Abelian group is a group with commutativity on $\cdot$:
    \begin{equation}
        a \odot b = b \odot a
    \end{equation}
\end{definition}

\begin{definition}[ring]
    A ring $(R, +, \odot)$ is defined as:
    \begin{itemize}
        \item $+$ is an Abelian group
        \item $\odot$ is a monoid
        \item Distributivity: $a \odot (b + c) = a \odot b + a \odot c$, and $(a+b) \odot c = a \odot c + b \odot c$
    \end{itemize}
    
\end{definition}

\begin{definition}[field]
    A field is a ring that $\odot$ is a group too.
\end{definition}

\begin{definition}[ordered field]
    An ordered field is a field with order $\leq$:
    \begin{itemize}
        \item $(R, \leq)$ is totally ordered
        \item $x < y \Rightarrow x + z < y + z$
        \item $x,y > 0 \Rightarrow xy > 0$
    \end{itemize}
    
    $x \in R$ is positive if $x > 0$.
\end{definition}


\begin{definition}[coset]
    Let $N$ be a subgroup of $G$ and $g \in G$. Then $g \odot N$ is the left coset and $N \odot g$ is the right coset.
    
    The left coset is an equivalent relation so it defines $G/N$ which is the set of left cosets of $G$ modulo $N$.
\end{definition}

\begin{definition}[normal]
    A subgroup $N$ is normal if for all $g \in G$, we have 
    \begin{equation}
        g \odot N = N \odot g
    \end{equation}    
\end{definition}

\begin{definition}[quotient group]
    A normal subgroup $N$ induces a quotient group $(G/N) \times (G/N) \rightarrow G/N$ that 
    \begin{equation}
        (a \odot N, b \odot N) \rightarrow (a \odot b) \odot N
    \end{equation} 
\end{definition}

\begin{definition}[homomorphism]
    Let $G = (G, \odot)$ and $H = (H, \otimes)$ be groups. A function $\varphi: G \rightarrow H$ is a homomorphism if 
    \begin{equation}
        \varphi(g \odot h) = \varphi(g) \otimes \varphi(h)
    \end{equation}
    
    The image of a homomorphism is a subgroup.
\end{definition}

\begin{theorem}
    If $\varphi$ is a homomorphism, then $\varphi(e) = e'$ and $\varphi(g)^{-1} = \varphi(g^{-1})$.
\end{theorem}

\begin{definition}[kernel]
    Let $\varphi$ be a homomorphism. The kernel of $\varphi$, $\kernel{\varphi}$ is defined as
    \begin{equation}
        \kernel{\varphi} = \varphi^{-1}(e')
    \end{equation}
    
    The kernel is a subgroup.
\end{definition}

\begin{theorem}
Let $\varphi$ be a homomorphism and $N = \kernel{\varphi}$, then
\begin{equation}
    g \odot N = \varphi^{-1}\left(\varphi(g)\right)
\end{equation}
\end{theorem}

\begin{theorem}
    Let $N$ be a normal subgroup of $G$. Then the quotient function $p: G \rightarrow G/N$ is a surjective homomorphism, the quotient homomorphism, with $\kernel{p} = N$.
\end{theorem}


\begin{definition}[isomorphism]
    A homomorphism is isomorphism if it is bijective. If it is from $G$ to itself, it is automorphism. Examples are:
    \begin{itemize}
        \item $g: a \odot g \odot a^{-1}$
        \item for surjective homomorphism $\varphi: G \rightarrow H$, $\varphi' : G/\kernel{\varphi} \rightarrow H$
        \item Let $(G, \odot)$ be a group, but $H$ is not. And $\varphi: G \rightarrow H$ is a bijection. Then define a function $\otimes$ that $g' \otimes h' := \varphi^{-1}(g') \odot \varphi^{-1}(h')$. $\otimes$ is the operation on $H$ induced from $\odot$ by $\varphi$
    \end{itemize}
\end{definition}



% numbers
\section{Numbers}

\subsection{Integer Number}
Natural number $\naturalnumber$ did not support the inverse $-n$. Integer $\integer$ is the smallest ring that contains $(\naturalnumber, +)$. For convenience we have defined the following:
\begin{itemize}
    \item The additive identity is denoted by $0_R$
    \item The multiplicity identity is denoted by $1_R$
    \item The additive inverse of $a$ is $-a$
    \item $0 \cdot a = a \cdot 0 = 0$
\end{itemize}

\subsection{Rational Number}
Integer $\integer$ did not support the $m/n$. Rational number $\rational$ is the smallest field that contains $\integer$. $\rational$ has a limitation too. For example, there is no rational $x$ that $x^2 = 2$. So we need to extend $\rational$ too.

\begin{definition}[\cindex{order complete}]
    Let $X$ be a totally ordered set. It is order complete if every nonempty subset is bounded below. It is also called complete.
\end{definition}


\begin{theorem}
    $\rational$ is not order complete.
\end{theorem}
\begin{proof}
    Define $A:=\set{x \in \rational; x > 0 , x^2 < 2}$, and $B:=\set{x \in \rational; x > 0, x^2 > 2}$. Assume there is $c$ that $a \leq c \leq b$, define $\xi = \frac{2c+2}{c+2}$, we have
    \begin{equation}
        \begin{aligned}
            \xi^2 - 2 &= \frac{2(c^2 -2)}{(c+2)^2} \\
            \xi - c &= - \frac{c^2 -2}{c+2}    
        \end{aligned}      
    \end{equation}
    
    If $c^2 < 2$, we have $\xi^2 < 2$ and $\xi < c$, so $\xi \in A$ and $\xi > c$, contradiction.
\end{proof}


\subsection{Real Number}

Real number $\realnumber$ is the order completion of rational number $\rational$. Here we use Dedekind cut to do the completion.

\begin{definition}[Dedekind cut]
    Let $F$ be a field. A Dedekind cut $(L,R)$ on $F$ is a partition of $F$ into two disjoint set $L$ and $R$ that
    \begin{itemize}
        \item $L \neq \emptyset, F$
        \item $L$ is closed downwards: for any $x,y \in F$ that if $y \in L$ and $x < y$, then $x \in L$
        \item $L$ does not have maximum element: for any $x \in L$, there is $y\in L$ that $x < y$
    \end{itemize}
\end{definition}

Let $F=\rational$, the set of all Dedekind cut $C$ on $\rational$ is real number.

For $q \in \rational$, there is a Dedekind cut $L_q = \set{x\in \rational: x < q}$. This $L_q$ has a property that
\begin{theorem}
    A Dedekind cut $L$ has a supremum in $\rational$ if and only if $L = L_q = \set{x \in \rational: x < q}$ for some $q\in \rational$.
\end{theorem}

\begin{definition}
    Define $C_\rational = \set{L_q: q \in \rational} \subseteq C$. The $+$ operation could be defined as
\begin{equation}
    L_p + L_q = \set{x + y \in \rational: x < p, y < q}
\end{equation}
\end{definition}


We could define the $\times$ on $L_\rational$ as well. Now we extend the $+$ and $\times$ operation to all $C$ and becomes the operations on real number.

\begin{theorem}\label{dedekind_real_order_complete}
    The Dedekind cut $C$ with $+$, $\times$ is order complete.    
\end{theorem}
\begin{proof}
    For a subset $D$ of $C$ whose elements are all bounded above, define a subset $M \subset C$ that
    \begin{equation}
        M = \set{x: x \in d \text{ for some } d \in D} =\bigcup_{d \in D} d
    \end{equation}
    
    $M$ is also a Dedekind cut. So $M$ is a upper bound for $D$. Then prove it is the minimum upper bound.
\end{proof}


\begin{theorem}
Let $x \in \realnumber$. It could be expressed as the supremum of the set $x = \sup \set{q \in \rational: q \leq x}$.
\end{theorem}



\begin{theorem}
    Here are the common properties of real numbers:
    \begin{itemize}
        \item For each $x \in \realnumber$, there is a $n$ that $n > x$.
        \item If $\displaystyle 0 \leq a \leq \frac{1}{n}$ for all $n \in \naturalnumber^{\times}$, then $a = 0$
        \item For any $a >0$, there is $n \in \naturalnumber^{\times}$ that $\displaystyle \frac{1}{n} < a$
        \item For any $a,b\in \realnumber$ that $a < b$, there is $q \in \rational$ that $a < q < b$
        \item For any $a,b\in \realnumber$ that $a < b$, there is $r \in \realnumber \textbackslash \rational $ that $a < r < b$
    \end{itemize}    
\end{theorem}
\begin{proof}
    To prove $\frac{1}{n} < a$, there is $n$ that $n > \frac{1}{a}$. 
    
    To prove $a < q < b$. There is $n \in \naturalnumber$ that $n > \frac{1}{b-a}$, so $na + 1 < nb$. For $na$, there is $m \in \integer$ that $m - 1 \leq na < m$, so $na < m \leq na + 1$. So $na < m \leq na + 1 < nb$, and $a < \frac{m}{n} < b$. Let $r = \frac{m}{n} \in \rational$.
    
    To prove $a < r < b$, find $q_1, q_r \in \rational$ that $a < q_1 < q_2 < b$. Define $\xi = q_1 + \frac{q_2 - q_2}{\sqrt{2}} \in \realnumber - \rational$.
\end{proof}


\begin{example}[decimal expression of rational number]
    For an rational number $r = \frac{p}{q}$, there is $a_0$ that $p = a_0 q + b_1$ where $b_1 < q-1$. This expression is equivalent to 
    \begin{equation}
        \frac{p}{q} = a_0 + \frac{1}{10} \left(\frac{10 b_1}{q} \right)
    \end{equation}
    
    because $0 < \frac{10 b_1}{q} < 10$, so there is $b_2$ that $\frac{10 b_1}{q} = a_1 + \frac{b_2}{q}$, which means
    \begin{equation}
        \frac{p}{q} = a_0 + \frac{1}{10} \left( a_1 + \frac{b_2}{q} \right) = a_0 + \frac{a_1}{10} + \frac{1}{10^2} \left(\frac{10 b_2}{q} \right)
    \end{equation}
\end{example}

\begin{theorem}
    The decimal expression of a rational number $r$ terminates after finitely many terms or it is periodic.    
\end{theorem}
\begin{proof}
    Let $r = \frac{p}{q}$. Assume the decimal expression never terminate. Because all $b_i \in \set{0, \cdots, q -1}$, some $b_i$ must be equal.
\end{proof}





The real number $\realnumber$ still has limitation. It cannot solve $x^2 = -1$. The smallest extension field $\complexnumber$ to $\realnumber$ is the complex number.


\begin{definition}[\cindex{extended number line}]
    The set $\closure{\realnumber} = \realnumber \cup \set{\pm \infty}$ is the extended number line. 
    
    Defined operations are: $x + \infty = \infty$, $x - \infty = -\infty$, $\displaystyle \frac{x}{\pm \infty} = 0$, $\infty + \infty = \infty$, $- \infty - \infty = - \infty$, $\infty \cdot \infty = \infty$.
    \begin{equation}
        \begin{aligned}
            x \cdot \pm \infty &= \begin{cases}
                \mp \infty \text{, } x > 0 \\
                \pm \infty \text{, } x < 0 \\
            \end{cases} \\
            \frac{x}{0} &= \begin{cases}
                \infty \text{, } x > 0 \\
                -\infty \text{, } x < 0 \\
            \end{cases} \\
        \end{aligned}
    \end{equation}
    
    Undefined operations are: $\infty - \infty$, $0 \times \pm \infty$, $\displaystyle \frac{\pm \infty}{\pm \infty}$, $\displaystyle \frac{0}{0}$, $\displaystyle \frac{\pm \infty}{0}$.
    
    It has the following property:
    \begin{itemize}
        \item $\closure{\realnumber}$ is a totally ordered set
        \item $\closure{\realnumber}$ is not a field
    \end{itemize}
\end{definition}

% complex number
\subsection{Complex Number}

The complex number $\complexnumber$ is the smallest extension field of $\realnumber$ that $x^2 = -1$ is solvable.

\begin{theorem}[De Moivre's Identity]
    For $z = a + b i \neq 0$ and $n \in \integer$, we have 
    \begin{equation}
        z^n = (a+bi)^n = r^n(\cos \theta + i \sin \theta)^n = r^n (\cos n\theta + i \sin n\theta)
    \end{equation}
\end{theorem}





























































































































































































































































