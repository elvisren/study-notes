\chapter{Function}

\section{Function Limit}

\begin{definition}
    Let $f: X \rightarrow \realnumber$. The function $f$ has a limit $L\in \realnumber$ as $x \rightarrow x_0$ if for every $\epsilon > 0$, there is a $\delta(\epsilon, x_0) > 0$ such that for all $0 < \absolutevalue{x - x_0} < \delta$, we have $\absolutevalue{f(x) - L} < \epsilon$. it has expression
    \begin{equation}
        \lim_{x \rightarrow x_0} f(x) = L
    \end{equation}
    
    The key is that $x$ cannot be $x_0$. Also $\delta(\epsilon)$ depends on $x_0$.
    
\end{definition}

\begin{definition}[one-sided limits]
    For every sequence $(x_n)$ that $x_n > x_0$ and $x_n \rightarrow x_0$, the image sequence $f(x_n) \rightarrow L$. It is also denoted as
    \begin{equation}
        \lim_{x \downarrow x_0} f(x) = L
    \end{equation}
    
    Similarly, we have 
    \begin{equation}
        \lim_{x \uparrow x_0} f(x) = L
    \end{equation}
\end{definition}


% continuous
\section{Continuous Function}
\begin{definition}\label{continuous_function_definition}
    A function is continuous at $x_0$ if 
    \begin{equation}
        \lim_{x \rightarrow x_0} f(x) = f(x_0)
    \end{equation}
    
    Similarly, we have right side continuous 
    \begin{equation}
        \lim_{x \downarrow x_0} f(x) = f(x_0)
    \end{equation}
    
    and left side continuous
    \begin{equation}
        \lim_{x \uparrow x_0} f(x) = f(x_0)
    \end{equation}
\end{definition}

\begin{definition}
    There are 3 types of discontinuities:
    \begin{itemize}
        \item If $\lim_{x \downarrow x_0} f(x)$ and $\lim_{x \uparrow x_0} f(x)$ both exist and are equal, but not equal to $f(x_0)$, it is a removable discontinuity at the point $x_0$.
        \item If they both exist, but not equal to each other, it is a jump discontinuity at the point $x_0$.
        \item If either one does not exist, it is an essential discontinuity at the point $x_0$.
    \end{itemize}
\end{definition}


\begin{theorem}
    Let $X,Y$ be metric spaces. Then a function $f: X\rightarrow Y$ is continuous at $x$ if and only if it is sequentially continuous at $x$, which is for every sequence $\set{x_i}$ such that $\lim x_i = x$, we have $\lim f(x_i) = f(x)$. A better expression is:
    \begin{equation}
        \lim f(x_i) = f(\lim x_i )
    \end{equation}
\end{theorem}



\begin{definition}[\cindex{uniform continuous}]
    A function $f$ is uniform continuous if for any $x_0 \in X$ and $\epsilon > 0$, there is $\delta(\epsilon) > 0$ that for all $\absolutevalue{x - x_0} < \delta$, we have $\absolutevalue{f(x) - f(x_0)} < \epsilon$.
    
    So $\delta$ does not depends on $x_0$, compared with \defiref{continuous_function_definition}.
\end{definition}

\begin{definition}[\cindex{Lipschitz continuity}]
    A function $f$ is Lipschitz continuous if there exists $L > 0$ that for all $x,y \in X$, we have 
    \begin{equation}
        \absolutevalue{f(x) - f(y)} \leq L \absolutevalue{x - y}
    \end{equation}
    
    There is no restriction that $L < 1$.
\end{definition}

\begin{theorem}
    The relationship among Lipschitz, uniform and normal continuity is
    
    \begin{equation}
        \text{Lipschitz continuous} \Rightarrow \text{uniformly continuous} \Rightarrow \text{continuous}
    \end{equation}    
\end{theorem}


\begin{theorem}[\cindex{continuous extension}]
    Let $X$ and $Y$ be metric spaces, and $D \subseteq X$, and $f: D\rightarrow Y$ is continuous. Suppose $a \in D^c$ is a limit point of $D$ and there is $y \in Y$ that $\lim_{x \rightarrow a} f(x) = y$. Then $\closure{f}: D \cup \set{a} \rightarrow Y$ is a continuous extension of $f$ to $D \cup \set{a}$.
\end{theorem}






% function sequence
\section{Function Sequence}

\begin{definition}[pointwise convergence]
    Let $(f_n)$ be a sequence of functions $f_n : X \rightarrow \realnumber$. The sequence $(f_n)$ converges pointwise to a function $f$ if for every $x_0 \in X$ and $\epsilon > 0$, there is an $N(x_0, \epsilon) \in \naturalnumber$ such that $\absolutevalue{f_n (x_0) - f(x_0)} < \epsilon$ for all $n > N$, which is written as
    \begin{equation}
        \lim_{n \rightarrow \infty} f_n (x) = f(x)
    \end{equation}
\end{definition}
















































































































































































































































































































































































































































































































































































































