\chapter{Function Sequence}


% convergence

\section{Convergence}

There are two ways to check the property of a sequence of function $\sequence{f_n (x)}$:
\begin{itemize}
    \item Pointwise: check $f_n (x_0)$ for one $x_0$
    \item Uniformly: check $f_n$ for all $x$
\end{itemize}

For example, pointwise increasing means for any $x$, $f_i (x) < f_{i+1}$. Uniformly bounded means $f_n \leq M$ for all $n$.

\begin{definition}[\cindex{pointwise convergence}]
    Let $(f_n)$ be a sequence of functions $f_n : X \rightarrow \realnumber$. The sequence $(f_n)$ converges pointwise to a function $f$ if for every $x_0 \in X$ and $\epsilon > 0$, there is an $N_{x_0, \epsilon} \in \naturalnumber$ such that $\absolutevalue{f_n (x_0) - f(x_0)} < \epsilon$ for all $n > N_{x_0, \epsilon}$, which is written as
    \begin{equation}
        \lim_{n \rightarrow \infty} f_n (x) = f(x)
    \end{equation}
\end{definition}

Pointwise convergence is weak because it only consider the property of a single point. We often need surrounding guarantees.

\begin{definition}
    Let $\sequence{f_n}$ be a sequence of functions $f_n : X \rightarrow \realnumber$:
    \begin{itemize}
        \item If $\sequence{f_n}$ is pointwise increasing and uniformly bounded from above, it pointwisely converge to a function $f$, which is written as \cindex{$f_n \uparrow f$}.
        \item If $\sequence{f_n}$ is pointwise decreasing and uniformly bounded from below, it pointwisely converge to a function $f$, which is written as \cindex{$f_n \downarrow f$}.
        \item If $\sequence{f_n}$ is pointwise monotone and uniformly bounded, it pointwisely converge to a function $f$.
    \end{itemize}
\end{definition}


\begin{definition}[\cindex{uniformly convergence}]
    The function sequence $\sequence{f_n (x)}$ converges uniformly to $f$ if for every $\epsilon >0$, there is $N_\epsilon \in \naturalnumber$ that for all $x \in X$ and $n \geq N_\epsilon$, we have $\absolutevalue{f_n (x) - f(x)} < \epsilon$.
\end{definition}

\begin{theorem}[\cindex{Dini's theorem}]
    Let $\sequence{f_n}$ be a sequence from a compact metric space $X \rightarrow \realnumber$ that $f_n \downarrow f$ (or $f_n \uparrow f$), and $f$ is continuous. Then this convergence is uniform.
\end{theorem}
\begin{proof}
    Define $d_n = f_n - f$. Since $f_n$ and $f$ is continuous, $d_n$ is continuous and $d_n \downarrow 0$. Now prove the uniformly convergence. For any $\epsilon > 0$, The reverse image $X_n = \inverse{d_n}(-\epsilon, \epsilon)$ has the following property:
    \begin{itemize}
        \item $X_n$ is open because $d_n$ is continuous
        \item $X_n \subset X_{n+1}$ because $d_n$ is decreasing
        \item $\cup_n X_n = X$ because $d_n$ pointwisely converge to $0$
    \end{itemize}
    
    So $\cup_n X_n$ is an open cover of compact space $X$, there is finite cover. Choose the $m = \max n$. So for any $k > m$, $\absolutevalue{f_k - f} < \epsilon$.
\end{proof}


\begin{theorem}
    Let $X \subset \realnumber$ and $\sequence{f_n}$ be a sequence of functions $f: X \rightarrow \realnumber$. Then $\sequence{f_n}$ converge uniformly to $f$ if and only if 
    \begin{equation}
        \sup_{x \in X} \absolutevalue{f_n(x) - f(x)} \rightarrow 0
    \end{equation}
    
    Using Cauchy test for convergence, it could also be expressed as for any $\epsilon > 0$, there is $N \in \realnumber$ that for all $m,n>N$, we have
    \begin{equation}
        \sup_{x \in X} \absolutevalue{f_m(x) - f_n(x)} < \epsilon
    \end{equation}
    
    This simplifies the check by considering only the supremum of function difference. It convert from the function sequence to point sequence.
\end{theorem}
  
























































































































































































































































































































































































































































































































































































































































































































