\chapter{Convergence}


%
%
% convergence of sequence
%
%

\section{Convergence of Sequence}


\begin{definition}[metric]
    Let $X$ be a set. A metric $d: X \times X \rightarrow \positiverealnumber$ has the following property:
    \begin{itemize}
        \item $d(x,y) = 0 \Leftrightarrow x = y$
        \item $d(x,y) = d(y,x)$
        \item $d(x,y) \leq d(x,z) + d(z y)$
    \end{itemize}
\end{definition}

\begin{definition}
    An open ball is
    \begin{equation}
        \openball{x}{r} := \set{y \in X; d(x,y) < r}
    \end{equation}
    
    An closed ball is
    \begin{equation}
        \closedball{x}{r} := \set{y \in X; d(x,y) \leq r}
    \end{equation}
\end{definition}

\begin{definition}[neighborhood]
    A subset $U \subseteq X$ is a neighborhood of $x$ if there is $r>0$ that $\openball{x}{r} \subseteq U$. 
\end{definition}

\begin{definition}[$\epsilon$-neighborhood]
    $\openball{x}{\epsilon}$ and $\closedball{x}{\epsilon}$ are called open and closed $\epsilon$-neighborhood of $x$.
\end{definition}

\begin{definition}[sequence]
    A sequence is a function $\naturalnumber \rightarrow X$, which is written as
    \begin{equation}
        (x_n)
    \end{equation}
\end{definition}

\begin{definition}[cluster point]
    $a$ is a cluster point of $(x_n)$ if every neighborhood of $a$ contains infinitely many terms of the sequence.
    
    For example, $(-1)^n$ has two cluster points: $1$ and $-1$.
\end{definition}

\begin{definition}[limit]
    A sequence $(x_n)$ converges to limit $a$ if every neighborhood of $a$ contains almost all terms of the sequence, which is written as
    \begin{equation}
        \lim_{n \rightarrow \infty} x_n = a \text{    or    } x_n \rightarrow a
    \end{equation}
\end{definition}

\begin{theorem}
    $x_n \rightarrow a$ means for each $\epsilon>0$, there is $m$ that for all $n > m$, $x_n \in \openball{a}{\epsilon}$.
\end{theorem}

\begin{definition}[bounded]
    A subset $Y \subseteq X$ is bounded if there is $M$ that $d(x,y) \leq M$ for all $x,y \in Y$. In this case, the diameter of $Y$ is defined as
    \begin{equation}
        \text{diam }(y) := \underset{x,y \in Y}{\text{sup}} d(x,y)
    \end{equation}
    
    It is easy to show that every convergent sequence is bounded.
\end{definition}

\begin{theorem}
    Let $x_n \rightarrow a$. Then $a$ is the unique cluster point of $(x_n)$. So the limit is unique.
\end{theorem}





%
%
% real and complex and normed vector sequence
%
%

\section{Real and Complex and Normed Vector Sequence}

\begin{theorem}
    Let $(x_n)$ and $(y_n)$ be convergent sequence with limit $a$ and $b$, we have
    \begin{itemize}
        \item $(x_n + y_n)$ converge to $a + b$
        \item $c (x_n)$ converge to $ca$
        \item $(x_n y_n)$ converge to $ab$
    \end{itemize}
\end{theorem}

\begin{theorem}
    For three real sequence $(x_n)$, $(y_n)$ and $(z_n)$ with property that $x_i \leq y_i \leq z_i$ for almost all $i$. If $x_n \rightarrow a$ and $z_n \rightarrow a$, we have $y_n \rightarrow a$.
\end{theorem}

\begin{theorem}
    For a sequence $(z_n)$ in $\complexnumber$, the following are equivalent:
    \begin{itemize}
        \item $(z_n)$ converges
        \item Real and imaginary part converge
    \end{itemize}
\end{theorem}

\begin{definition}
    For a inner product vector space $E$, define a norm as 
    \begin{equation}
        d(x,y) := \vectornorm{x - y}
    \end{equation}
\end{definition}




%
%
% monotone sequence
%
%

\section{Monotone Sequence}

\begin{definition}[increasing]
    A sequence $(x_n)$ is increasing if $x_n \leq x_{n+1}$.
\end{definition}

\begin{theorem}
    Every real sequence has a monotone subsequence.    
\end{theorem}
\begin{proof}
    for a sequence $(a_n)$, find all indices $m$ that for all terms $(a_n)$ after $a_m$ is strictly smaller than $a_m$:
    \begin{equation}
        V = \set{m \in \integer: a_j < a_m \text{ for all } j > m}
    \end{equation}
    
    If $V$ is infinite, order the elements of $V$ from small to big, and we have the decreasing subsequence. If $V$ is finite, we find $N = \max V$, and build non-decreasing sequence from $a_{N+1}$.
\end{proof}


\begin{theorem}
    Every increasing (decreasing) bounded sequence $(x_n)$ in $\realnumber$ converges. 
\end{theorem}
\begin{proof}
    Since $\realnumber$ is order complete and $(x_n)$ is bounded above, it has upper bound which is the limit.
\end{proof}


\begin{theorem}[Bolzano-Weierstrass Theorem]
    If $(a_n)$ is a bounded sequence, then there exists a subsequence that is convergent.
\end{theorem}
\begin{proof}
    $(a_n)$ has a monotone subsequence which is bounded, so it is convergent.
\end{proof}


Some important limits are:
\begin{itemize}
    \item $\displaystyle \lim_{n \rightarrow \infty} \frac{n^k}{a^n} = 0$
    \item $\displaystyle \lim_{n \rightarrow \infty} \frac{a^n}{n!} = 0$
    \item $\displaystyle \lim_{n \rightarrow \infty} \sqrt[n]{n} = 1$
    \item $\displaystyle \lim_{n \rightarrow \infty} \left(1+\frac{1}{n} \right)^n = e$
\end{itemize}


% Cauchy sequence

\section{Cauchy Sequence}

\begin{definition}
    A real sequence $(a_n)$ is a Cauchy sequence if for every $\epsilon > 0$, there is a $N \in \naturalnumber$ that for every $m,n \geq N$, we have $\absolutevalue{a_m - a_n} < \epsilon$.
\end{definition}

\begin{theorem}
    A real sequence $(a_n)$ converges if and only if it is Cauchy.
\end{theorem}
\begin{proof}
    $(a_n)$ has a monotone subsequence $(a_m)$. Since $(a_n)$ is Cauchy, it is bounded. So $(a_m)$ is a bounded monotone subsequence, so it has a limit $L$. Prove $(a_n)$ converges to $L$ as well.
\end{proof}


% infinity limit
\section{Infinity Limit}

Extend $\realnumber$ to $\extendedrealnumber$ by adding $\pm \infty$ to the cluster points.

\begin{theorem}
    Every monotone sequence in $\realnumber$ converged in $\extendedrealnumber$.
\end{theorem}


\begin{definition}
    Let $(x_n)$ be a sequence in $\realnumber$. Define limit superior as
    \begin{equation}
        S = \limsup_{n \rightarrow \infty} x_n = \varlimsup_{n\rightarrow \infty} x_n = \lim_{n \rightarrow \infty} \underset{k \geq n}{\text{ sup }} x_k
    \end{equation}
    
    and limit inferior as
    \begin{equation}
        I = \liminf_{n \rightarrow \infty} x_n = \varliminf_{n\rightarrow \infty} x_n = \lim_{n \rightarrow \infty} \underset{k \geq n}{\text{ inf }} x_k
    \end{equation}    
\end{definition}

Be noted that $\displaystyle \varlimsup_{n\rightarrow \infty} x_n$ is a decreasing sequence and $\displaystyle \varliminf_{n\rightarrow \infty} x_n$ is an increasing sequence.


\begin{theorem}\label{definition_of_limit_superium_infinium}
    Let $(a_n)$ be a bounded sequence, then
    \begin{equation}
        \begin{aligned}
            \varlimsup_{n\rightarrow \infty} x_n &= \sup \set{r \in \realnumber: a_n > r \text{ for infinity many } n} \\
            &= \inf \set{r \in \realnumber: a_n > r \text{ for finitely many } n} \\
            \varliminf_{n\rightarrow \infty} x_n &= \inf \set{r \in \realnumber: a_n < r \text{ for infinity many } n} \\
            &= \sup \set{r \in \realnumber: a_n < r \text{ for finitely many } n} 
        \end{aligned}
    \end{equation}

    
    If limit superior S is a real number $s$, it means that for any $\epsilon>0$, there is a $N$ that $s + \epsilon$ is an upper bound for $x_{n > N}$. The reverse is true for limit inferior I.

    $[I,S]$ may not contain any number from $(x_n)$. But for any $\epsilon >0$, $[I - \epsilon, S + \epsilon]$ contains all but finite numbers from $(x_n)$, and this is the smallest closed interval with this property.
\end{theorem}




In general we have 
\begin{equation}
    \inf_{n \rightarrow \infty} x_n \leq \varliminf_{n \rightarrow \infty} x_n \leq \varlimsup_{n \rightarrow \infty} x_n \leq \sup_{n \rightarrow \infty} x_n
\end{equation}

The additivity of limit superior and inferior is
\begin{equation}
    \begin{aligned}
        \varliminf_{n \rightarrow \infty} (a_n + b_n) &\geq \varliminf_{n \rightarrow \infty} a_n + \varliminf_{n \rightarrow \infty} b_n \\
        \varlimsup_{n \rightarrow \infty} (a_n + b_n) &\leq \varlimsup_{n \rightarrow \infty} a_n + \varlimsup_{n \rightarrow \infty} b_n \\
    \end{aligned}
\end{equation}


\begin{theorem}
    A sequence converges in $\realnumber$ when 
    \begin{equation}
        \varlimsup_{n\rightarrow \infty} x_n = \varliminf_{n\rightarrow \infty} x_n \in \realnumber
    \end{equation}
\end{theorem}
\begin{proof}
    Assume $\varlimsup_{n\rightarrow \infty} x_n = \varliminf_{n\rightarrow \infty} x_n = L$. Consider $(L - \epsilon, L + \epsilon)$. By \theoref{definition_of_limit_superium_infinium}, there is $N$ that $a_{n > N} \in (L - \epsilon, L + \epsilon)$
\end{proof}

\begin{theorem}
    Any sequence $(x_n)$ in $\realnumber$ has a smallest cluster point $x_*$ and greatest cluster point $x^*$ in $\extendedrealnumber$ that
    \begin{equation}
        \begin{aligned}
            \limsup_{n \rightarrow \infty} x_n &= \varlimsup_{n\rightarrow \infty} x_n &= x^* \\
            \liminf_{n \rightarrow \infty} x_n &= \varliminf_{n\rightarrow \infty} x_n &= x_*
        \end{aligned}
    \end{equation}
\end{theorem}
\begin{proof}
    Let's check $x^*$. 
    
    If $x^* = - \infty$. Then for any $m < 0$, there is $n$ that $m > \underset{k \geq n}{\text{ sup }} x_k$, so $x_k < m$ for all $m$. So $x^* = -\infty$ is the only cluster point of $(x_n)$.
    
    If $x^* = \infty$, the case is the same.
        
    Now assume $x^* \in \realnumber$. Since $\underset{k \geq n}{\text{ sup }} x_k \geq x_n$, if $x^*$ is a cluster point, it will be greater than any other cluster point. Now the question becomes whether $x^*$ is a cluster point. 
    
    By \theoref{definition_of_limit_superium_infinium}, for any $\epsilon>0$, there are infinitely many $x_i$ that $x^* < x_i < x^* + \epsilon$. So $x^*$ is a cluster point.
\end{proof}

limit superior and inferior are useful because they always exist but limit may not exist. So they are good substitute when we want to study the behavior of a sequence towards infinity.




% real series
\section{Real Series}

\begin{definition}[convergent series]
    A real series $\sum_{i=1}^\infty a_i$ is convergent if the sequence of its partial sum converges, which is defined as
    \begin{equation}
        \sum_{i=1}^\infty a_i = \lim_{n \rightarrow \infty} \sum_{i=1}^n a_i \in \realnumber
    \end{equation}
\end{definition}

\begin{theorem}[Cauchy criterion test]\label{cauchy_convergent_criterion}
    The real series $\sum_{i=1}^\infty a_i$ converges if for every $\epsilon > 0$, there is an $N$ that for $m,n > N$, we have $\absolutevalue{a_{m+1} + a_{m+2} + \cdots + a_n} < \epsilon$.
\end{theorem}

\begin{definition}[absolute convergence]
    A real series $\sum_{i=1}^\infty a_i$ is called absolute convergent if the absolute series $\sum_{i=1}^\infty \absolutevalue{a_i}$ converges.
\end{definition}

\begin{definition}[conditionally convergence]
    A real series is called conditionally convergent if it is convergent but not absolutely convergent.
\end{definition}

\begin{theorem}[absolute convergence test]
    If $\sum_{i=1}^\infty a_i$ is absolutely convergent, it is convergent.
\end{theorem}
\begin{proof}
    Use \theoref{cauchy_convergent_criterion}.
\end{proof}

\begin{theorem}[alternating series test]
    An alternating series of the form $\sum_{i=1}^\infty (-1)^i b_j$ with $b_j > 0$ converges if $(b_j)$ is decreasing and $b_j \rightarrow 0$.
\end{theorem}
\begin{proof}
    Define $s_n = \sum_{i=1}^n (-1)^i b_j$. For $s_{2n}$, we have
    \begin{equation}
        \begin{aligned}
            s_{2n} &= b_1 - b_2 + b_3 - b_4 + \cdots -b_{2n-2} + b_{2n-1} - b_{2n} \\
            &= b_1 - (b_2 - b_3) - \cdots - (b_{2n-2} - b_{2n-1}) - b_{2n} \leq b_1
        \end{aligned}
    \end{equation}
    
    So $s_{2n}$ is bounded above. Also $s_{2n+2} - s_{2n} = b_{2n+2} - b_{2n+1} > 0$ so $s_{2n}$ is increasing. Therefore $s_{2n}$ is convergent. $s_{2n-1}$ is convergent too. And their different is $a_{2n} \rightarrow 0$ and two limits are the same.
\end{proof}


\begin{theorem}[ratio test]
    For a real series $\sum_{i=1}^\infty a_i$ with $a_i \neq 0$. Let $L = \lim_{j \rightarrow \infty} \absolutevalue{\frac{a_{j+1}}{a_j}}$, we have
    \begin{itemize}
        \item If $L < 1$, the series converges absolutely
        \item If $L > 1$, the series diverges
    \end{itemize}
\end{theorem}

\begin{theorem}[root test]
    For a real series $\sum_{i=1}^\infty a_i$ with $a_i \neq 0$. Let $L = \lim_{j \rightarrow \infty}  \sqrt[j]{\absolutevalue{a_j}}$, we have
    \begin{itemize}
        \item If $L < 1$, the series converges absolutely
        \item If $L > 1$, the series diverges
    \end{itemize}
\end{theorem}

\begin{theorem}[Raabe's test]
    For a real series $\sum_{i=1}^\infty a_i$ with $a_i \neq 0$. Suppose there is 
    \begin{equation}
        L = \lim_{n \rightarrow \infty} n \left( \absolutevalue{\frac{a_n}{a_{n+1}}} - 1 \right)
    \end{equation}
    
    Then:
    \begin{itemize}
        \item If $L > 1$, the series converges absolutely
        \item If $L < 1$, the series diverges
    \end{itemize}
\end{theorem}


\begin{theorem}[Riemann rearrangement theorem]
    Let $\sum_{i=1}^\infty a_i$ be a real series that converges conditionally. For any $L \in \realnumber$, there is a rearrangement of the series so that $\sum_{i=1}^\infty a_{\sigma (i)} = L$.
\end{theorem}
\begin{proof}
    For a real series $\sum_{i=1}^\infty a_i$, define the positive part $a^{+}_i = \max \set{a_i, 0}$ and the negative part$a_i^{-} = - \min \set{a_i, 0}$. We have $a_i = a_i^{+} - a_i^{-}$ and $\absolutevalue{a_i} = a_i^{+} + a_i^{-}$. 
    
    First we could prove a theorem that the original series converges absolutely if and only if both the positive and negative part converge. It uses the property that $\absolutevalue{a_i} = a_i^{+} + a_i^{-}$. So if the series is conditionally convergent, then both positive and negative part diverge.

    Now construct the arrangement. Assume $a_i \neq 0$ and $L > 0$. Split $(a_i)$ into two parts:
    \begin{itemize}
        \item $(p_i)$ are all positive numbers
        \item $(q_i)$ are all negative numbers
    \end{itemize}
    
    First find $j_1$ that
    \begin{equation*}
        \begin{aligned}
            s &= p_1 + p_2 + \cdots + p_{j_1} \geq L \\
            s - p_{j_1} &\leq L
        \end{aligned}
    \end{equation*}
    
    It is always possible because $p_i$ diverges. Now find $i_i$ that
    \begin{equation*}
        \begin{aligned}
            s_{k_1} &= s + q_1 + q_2 + \cdots + q_{i_1} \leq L \\
            s - q_{i_1} &\geq L
        \end{aligned}
    \end{equation*}
    
    Keep creating these $s_{k_n}$. We have $s_{k_n} \leq L$ and $\absolutevalue{L - s_{k_n}} < - q_{i_n} \rightarrow 0$.
\end{proof}



































































































































































































































































































































































































































































































































































































































































































































































