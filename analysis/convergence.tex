\chapter{Convergence}

The order of research:
\begin{itemize}
    \item Define sequence    
    \item Define series
    \item Analyze monotone sequence
    \item Analyze alternating sequence
    \item Analyze absolute convergent sequence
    \item Define power series and radius
\end{itemize}


How to check whether a series converges:
\begin{itemize}
    \item The item $x_i \rightarrow 0$. If not, the series is not converging
    \item If we can guess the converged value, prove it
    \item If we cannot, use Cauchy criterion test
    \item If it is monotone, check boundedness
    \item If it is alternating, check whether $\absolutevalue{x_i}$ decreases
    \item Check the convergence of absolute value, use ratio test, root test, etc
    \item If it is power series, calculate the radius of convergence
\end{itemize}

%
%
% convergence of sequence
%
%

\section{Convergence of Sequence}

First we need to borrow the definitions from topology: metric, open ball, closed ball, neighborhood, $\epsilon$-neighborhood, bounded, diameter.

The key is to define limit and cluster point (which are different terms).

\begin{definition}[\cindex{sequence}]
    A sequence is a function $f: \naturalnumber \rightarrow X$, which is written as
    \begin{equation}
        \sequence{x_n}
    \end{equation}
    
    \emph{A sequence $\sequence{x_n}$ is different from set $\set{x_n}$}.
\end{definition}

\begin{definition}[\cindex{cluster point}]
    $a$ is a cluster point of $\sequence{x_n}$ if every neighborhood of $a$ contains infinitely many terms of the sequence.
    
    For example, $\sequence{(-1)^n}$ has two cluster points: $1$ and $-1$.
\end{definition}

\begin{definition}[\cindex{limit}]
    A sequence $\sequence{x_n}$ converges to limit $a$ if every neighborhood of $a$ contains almost all terms of the sequence, which is written as
    \begin{equation}
        \lim_{n \rightarrow \infty} x_n = a \text{ or } x_n \rightarrow a
    \end{equation}
    
    $x_n \rightarrow a$ means for each $\epsilon>0$, there is $m$ that for all $n > m$, $x_n \in \openball{a}{\epsilon}$
\end{definition}

\emph{Limit point is different from cluster point}. The cluster point did not require its open ball contains all the rest of sequence.


\begin{theorem}\label{property_of_limit}
    Properties of limit and convergent sequence:
    \begin{itemize}
        \item A sequence has a unique limit point
        \item A convergent sequence is bounded
        \item A convergent sequence has a unique cluster point
        \item A point $a$ is a cluster point if there is subsequence which converges to $a$
    \end{itemize}
\end{theorem}





%
%
% real and complex and normed vector sequence
%
%

\section{Algebra of Convergent Sequence}

\begin{definition}[\cindex{null sequence}]
    A sequence $\sequence{x_n}$ is called null sequence if it converges to $0$. So a sequence $x_n \rightarrow a$ if and only if $\sequence{x_n - a}$ is a null sequence.
\end{definition}

\begin{theorem}
    Let $\sequence{x_n}$ and $(y_n)$ be convergent sequence with limit $a$ and $b$. The operations on sequence are:
    \begin{itemize}
        \item $\sequence{x_n + y_n} \rightarrow a + b$
        \item $\sequence{c \cdot x_n} \rightarrow c a$
        \item $\sequence{x_n y_n} \rightarrow ab$
        \item $\intoo{\frac{1}{x_n}} \rightarrow \frac{1}{a}$ if $a > 0$
        \item $\sequence{\absolutevalue{x_n}} \rightarrow \absolutevalue{a}$
    \end{itemize}
    
    So the convergent sequence is a subalgebra of all sequences.
\end{theorem}

\begin{theorem}
    Comparison test:
    \begin{itemize}
        \item Let $\sequence{x_n}$ and $\sequence{y_n}$ be convergent sequence in $\realnumber$ that $x_n \leq y_n$ for almost all $n$. Then $\lim x_n \leq \lim y_n$.
        \item For three real sequence $\sequence{x_n}$, $(y_n)$ and $(z_n)$ with property that $x_i \leq y_i \leq z_i$ for almost all $i$. If $x_n \rightarrow a$ and $z_n \rightarrow a$, we have $y_n \rightarrow a$.
    \end{itemize}
\end{theorem}

\begin{theorem}[\cindex{sandwich lemma}]
    Let $\sequence{a_n}$, $\sequence{b_n}$ and $\sequence{c_n}$ be three sequence such that $a_n \leq b_n \leq c_n$. Suppose $\sequence{a_n} \rightarrow L$ and $\sequence{c_n} \rightarrow L$, then $\sequence{b_n} \rightarrow L$.
\end{theorem}


\begin{theorem}
    For a sequence $(z_n)$ in $\complexnumber$, the following are equivalent:
    \begin{itemize}
        \item $(z_n)$ converges
        \item Real and imaginary part converge
    \end{itemize}
    
    So 
    \begin{equation}
        \lim x_n = \lim \realpartofcomplexnumber{x_n} + i \lim \imaginarypartofcomplexnumber{x_n}
    \end{equation}
\end{theorem}
\begin{proof}
    use $\absolutevalue{\realpartofcomplexnumber{x_n} - \realpartofcomplexnumber{x}} \leq \absolutevalue{x_n - x}$ and $\absolutevalue{x_n -x} = \sqrt{\absolutevalue{\realpartofcomplexnumber{x_n} - a}^2 + \absolutevalue{\imaginarypartofcomplexnumber{x_n} - a}^2}$
\end{proof}


For vector space, we need to define a metric on it. So either it is a metric space with metric $\length{\cdot}$, $\length{\cdot}_\infty$, or inner product $\innerproduct{\cdot}{\cdot}$.



%
%
% monotone sequence
%
%

\section{Monotone Sequence}

\begin{definition}[\cindex{increasing}]
    A sequence $\sequence{x_n}$ is increasing if $x_n \leq x_{n+1}$.
\end{definition}


\begin{theorem}
    Every increasing (decreasing) bounded sequence $\sequence{x_n}$ in $\realnumber$ converges. \theoref{property_of_limit} said boundedness is a necessary condition for the convergence of a sequence. For monotone sequence, it is sufficient condition.
\end{theorem}
\begin{proof}
    By \theoref{dedekind_real_order_complete}, since $\realnumber$ is order complete and $\sequence{x_n}$ is bounded above, it has upper bound which is the limit.
\end{proof}


\begin{theorem}
    Every real sequence has a monotone subsequence.    
\end{theorem}
\begin{proof}
    for a sequence $\sequence{a_n}$, find all indices $m$ that for all terms $\sequence{a_n}$ after $a_m$ is strictly smaller than $a_m$:
    \begin{equation}
        V = \set{m \in \integer: a_j < a_m \text{ for all } j > m}
    \end{equation}
    
    If $V$ is infinite, order the elements of $V$ from small to big, and we have the decreasing subsequence. If $V$ is finite, we find $N = \max V$, and build non-decreasing sequence from $a_{N+1}$.
\end{proof}


\begin{theorem}[\cindex{Bolzano-Weierstrass Theorem}]
    If $\sequence{a_n}$ is a bounded sequence, then there exists a subsequence that is convergent.
\end{theorem}
\begin{proof}
    $\sequence{a_n}$ has a monotone subsequence which is bounded, so it is convergent.
\end{proof}




% infinite limits

\section{Infinite Limits}

Now consider the limit over $\closure{\realnumber}$. The extended definitions are:
\begin{itemize}
    \item A neighborhood around $\infty$ is the set $(a, \infty)$
    \item $\infty$ is a cluster point if every neighborhood of $\infty$ contains infinitely many terms of $\sequence{x_n}$
    \item $\sequence{x_n}$ converges in $\closure{\realnumber}$ if there is $x \in \closure{\realnumber}$ that $\lim_{n \rightarrow \infty} x_n = x$
    \item If a sequence is convergent in $\closure{\realnumber}$ but divergent in $\realnumber$, it is said to \cindex{converge improperly}
\end{itemize}

\begin{theorem}
    Every monotone sequence in $\realnumber$ converges in $\closure{\realnumber}$.
\end{theorem}
\begin{proof}
    If it is bounded, it is convergent in $\realnumber$. Or it is convergent in $\closure{\realnumber}$.
\end{proof}




% infinite superior and inferior

\section{Limit Superior and Inferior}

\begin{definition}
    Let $\sequence{x_n}$ be a sequence in $\realnumber$. Define \cindex{limit superior} as
    \begin{equation}
        S = \limsup_{n \rightarrow \infty} x_n = \varlimsup_{n\rightarrow \infty} x_n = \lim_{n \rightarrow \infty} \underset{k \geq n}{\text{ sup }} x_k
    \end{equation}
    
    and \cindex{limit inferior} as
    \begin{equation}
        I = \liminf_{n \rightarrow \infty} x_n = \varliminf_{n\rightarrow \infty} x_n = \lim_{n \rightarrow \infty} \underset{k \geq n}{\text{ inf }} x_k
    \end{equation}    
\end{definition}

Be noted that $\sequence{\underset{k \geq n}{\text{ sup }} x_k}$ is a decreasing sequence and $\sequence{\underset{k \geq n}{\text{ inf }} x_k}$ is an increasing sequence.


\begin{theorem}\label{definition_of_limit_superium_infinium}
    Let $\sequence{a_n}$ be a bounded sequence, then
    \begin{equation}
        \begin{aligned}
            \varlimsup_{n\rightarrow \infty} x_n &= \sup \set{r \in \realnumber: a_n > r \text{ for all but finite many } n} \\
            &= \inf \set{r \in \realnumber: a_n > r \text{ for finitely many } n} \\
            \varliminf_{n\rightarrow \infty} x_n &= \inf \set{r \in \realnumber: a_n < r \text{ for all but finite many } n} \\
            &= \sup \set{r \in \realnumber: a_n < r \text{ for finitely many } n} 
        \end{aligned}
    \end{equation}

    
    If limit superior S is a real number $s$, it means that for any $\epsilon>0$, there is a $N$ that $s + \epsilon$ is an upper bound for $x_{n > N}$. The reverse is true for limit inferior I.

    $[I,S]$ may not contain any number from $\sequence{x_n}$. \emph{But for any $\epsilon >0$, $[I - \epsilon, S + \epsilon]$ contains all but finite numbers from $\sequence{x_n}$, and this is the smallest closed interval with this property}.
\end{theorem}

\begin{theorem}
    Here is the algebra over limit superior and inferior (multiplication is for non-negative sequence):
    
    \begin{equation}
        \begin{aligned}
        \inf_{n \rightarrow \infty} x_n \leq \varliminf_{n \rightarrow \infty} x_n &\leq \varlimsup_{n \rightarrow \infty} x_n \leq \sup_{n \rightarrow \infty} x_n \\
        \varliminf_{n \rightarrow \infty} (a_n + b_n) &\geq \varliminf_{n \rightarrow \infty} a_n + \varliminf_{n \rightarrow \infty} b_n \\
        \varlimsup_{n \rightarrow \infty} (a_n + b_n) &\leq \varlimsup_{n \rightarrow \infty} a_n + \varlimsup_{n \rightarrow \infty} b_n \\
        \varlimsup_{n \rightarrow \infty} (a_n  b_n) &\leq \varlimsup_{n \rightarrow \infty} (a_n) \varlimsup_{n \rightarrow \infty} (a_n) \\
        \varliminf_{n \rightarrow \infty} (a_n  b_n) &\geq \varliminf_{n \rightarrow \infty} (a_n) \varliminf_{n \rightarrow \infty} (a_n)
    \end{aligned}
    \end{equation}
\end{theorem}
\begin{proof}
    Fix a $n$, for any $j \geq n$, we have $a_j + b_j \leq \sup_{m\geq n} a_m + \sup_{m\geq n} b_m$. Take supremum over $j \geq n$, we have $\sup_{j \geq n} (a_j + b_j ) \leq \sup_{m\geq n} a_m + \sup_{m\geq n} b_m$. Then take $n \rightarrow \infty$.
\end{proof}



\begin{theorem}
    A sequence converges in $\realnumber$ when 
    \begin{equation}
        \varlimsup_{n\rightarrow \infty} x_n = \varliminf_{n\rightarrow \infty} x_n \in \realnumber
    \end{equation}
\end{theorem}
\begin{proof}
    Assume $\varlimsup_{n\rightarrow \infty} x_n = \varliminf_{n\rightarrow \infty} x_n = L$. Consider $(L - \epsilon, L + \epsilon)$. By \theoref{definition_of_limit_superium_infinium}, there is $N$ that $a_{n > N} \in (L - \epsilon, L + \epsilon)$
\end{proof}

\begin{theorem}
    Any sequence $\sequence{x_n}$ in $\realnumber$ has a smallest cluster point $x_*$ and greatest cluster point $x^*$ in $\extendedrealnumber$ that
    \begin{equation}
        \begin{aligned}
            \limsup_{n \rightarrow \infty} x_n &= \varlimsup_{n\rightarrow \infty} x_n &= x^* \\
            \liminf_{n \rightarrow \infty} x_n &= \varliminf_{n\rightarrow \infty} x_n &= x_*
        \end{aligned}
    \end{equation}
    
    So every sequence has at least one cluster point, but it may not have limit.
\end{theorem}
\begin{proof}
    Let's check $x^*$. 
    
    If $x^* = - \infty$. Then for any $m < 0$, there is $n$ that $m > \underset{k \geq n}{\text{ sup }} x_k$, so $x_k < m$ for all $m$. So $x^* = -\infty$ is the only cluster point of $\sequence{x_n}$.
    
    If $x^* = \infty$, the case is the same.
        
    Now assume $x^* \in \realnumber$. Since $\underset{k \geq n}{\text{ sup }} x_k \geq x_n$, if $x^*$ is a cluster point, it will be greater than any other cluster point. Now the question becomes whether $x^*$ is a cluster point. 
    
    By \theoref{definition_of_limit_superium_infinium}, for any $\epsilon>0$, there are infinitely many $x_i$ that $x^* < x_i < x^* + \epsilon$. So $x^*$ is a cluster point.
\end{proof}

limit superior and inferior are useful because \emph{they always exist but limit may not exist}. So they are good substitute when we want to study the behavior of a sequence towards infinity.



% Cauchy sequence

\section{Completeness}

A complete space gives us a way to prove the convergence of a sequence without knowing the actual limit.

\begin{definition}[\cindex{Cauchy sequence}]
    A real sequence $\sequence{a_n}$ is a Cauchy sequence if for every $\epsilon > 0$, there is a $N \in \naturalnumber$ that for every $m,n \geq N$, we have $\absolutevalue{a_m - a_n} < \epsilon$.
\end{definition}

\begin{theorem}
    A real sequence $\sequence{a_n}$ converges if and only if it is Cauchy.
\end{theorem}
\begin{proof}
    $\sequence{a_n}$ has a monotone subsequence $\sequence{a_m}$. Since $\sequence{a_n}$ is Cauchy, it is bounded. So $\sequence{a_m}$ is a bounded monotone subsequence, so it has a limit $L$. Prove $\sequence{a_n}$ converges to $L$ as well.
\end{proof}

\begin{definition}[\cindex{Banach space}]
    A complete normed vector space is called Banach space.
\end{definition}

\begin{definition}[\cindex{Hilbert space}]
    A complete inner product space is called Hilbert space.
\end{definition}

\begin{theorem}
    Let $X$ be non-empty set and $E$ a Banach space. Then the set of all bounded function $\set{f: X \rightarrow E}$ is a Banach space.
\end{theorem}
\begin{proof}
    For any $\epsilon$, we have $\norm{u_m - u_n}_\infty \leq \epsilon$, so for any $x \in X$, we have $\norm{u_m (x) - u_n (x)} \leq \norm{u_m - u_n}_\infty \leq \epsilon$. So each $\sequence{u_m (x)}$ is a Cauchy sequence in $E$, so it converges to $a_x$. Define the function $f: x \rightarrow a_x$, and prove they converge to it.
\end{proof}



% real series
\section{Series}

\begin{definition}[\cindex{series}]
    Let $\sequence{x_k}$ be a sequence in $E$. A series is a new $\sequence{s_k}$ in $E$ which is defined as
    \begin{equation}
        s_n = \sum_{k=0}^n x_k
    \end{equation}
    
    Each $s_k$ is called \cindex{partial sum} and $x_k$ is called \cindex{summand}. The series $\sum x_k$ converges if the sequence $\sequence{s_k}$ of its partial sum converges.
\end{definition}

\begin{theorem}\label{convergent_series_is_null_sequence}
    if the series $\sum x_k$ converges, then $\sequence{x_k}$ is a null sequence.
\end{theorem}
\begin{proof}
    Since the partial sum $\sequence{s_k}$ converges, the partial sum is a Cauchy sequence, so $\absolutevalue{s_{k+1} - s_k} = x_k < \epsilon$.
\end{proof}


\begin{theorem}[\cindex{Cauchy criterion test}]\label{cauchy_convergent_criterion}
    The real series $\sum_{i=1}^\infty a_i$ in Banach space converges if for every $\epsilon > 0$, there is an $N$ that for $m > n > N$, we have $\absolutevalue{a_{m+1} + a_{m+2} + \cdots + a_n} < \epsilon$.
\end{theorem}


% alternating series

\section{Alternating Series}

\begin{definition}[\cindex{alternating}]
    A series $\sum y_k$ is alternating if $y_k$ and $y_{k+1}$ have opposite sign for all $k$. It could be written as
    \begin{equation}
        \pm \sum_{k=1}^\infty (-1)^k x_k \text{, with } x_k \geq 0
    \end{equation}
\end{definition}


\begin{theorem}[\cindex{Lebniz criterion}]
    An alternating series of the form $\sum_{i=1}^\infty (-1)^i b_j$ with $b_j > 0$ converges if $(b_j)$ is decreasing and $b_j \rightarrow 0$.
\end{theorem}
\begin{proof}
    Define $s_n = \sum_{i=1}^{n+1} (-1)^i b_i$. For $s_{2n}$, we have
    \begin{equation}
        \begin{aligned}
            s_{2n} &= b_1 - b_2 + b_3 - b_4 + \cdots -b_{2n-2} + b_{2n-1} - b_{2n} \\
            &= b_1 - (b_2 - b_3) - \cdots - (b_{2n-2} - b_{2n-1}) - b_{2n} \leq b_1
        \end{aligned}
    \end{equation}
    
    So $\sequence{s_{2n}}$ is bounded above. Also $s_{2n+2} - s_{2n} = b_{2n+2} - b_{2n+1} > 0$ so $\sequence{s_{2n}}$ is increasing. Therefore $s_{2n}$ is convergent. $\sequence{s_{2n-1}}$ is convergent too. And their different is $a_{2n} \rightarrow 0$ so two limits are the same.
\end{proof}




% absolute convergence

\section{Absolute Convergence}

The test for absolute convergence is built on the test of geometric series.

\begin{definition}[\cindex{absolute convergence}]
    A real series $\sum_{i=1}^\infty a_i$ is called absolute convergent if the absolute series $\sum_{i=1}^\infty \absolutevalue{a_i}$ converges.
\end{definition}

\begin{definition}[\cindex{conditionally convergence}]
    A real series is called conditionally convergent if it is convergent but not absolutely convergent.
\end{definition}

\begin{theorem}\label{conditional_converge_requirement}
    For a real series $\sum_{i=1}^\infty a_i$, define the \cindex{positive part} $a^{+}_i = \max \set{a_i, 0}$ and the \cindex{negative part} $a_i^{-} = - \min \set{a_i, 0}$. We have
    \begin{equation}
        \begin{aligned}
            a_i &= a_i^{+} - a_i^{-} \\
            \absolutevalue{a_i} &= a_i^{+} + a_i^{-}
        \end{aligned}
    \end{equation}
    
    A series converges absolutely if and only if both the positive and negative part converge. 
\end{theorem}
\begin{proof}
    $\absolutevalue{a_i} = a_i^{+} + a_i^{-}$. So if the series is conditionally convergent, then both positive and negative part diverge.   
\end{proof}


\begin{theorem}[absolute convergence test]
    If $\sum_{i=1}^\infty a_i$ is absolutely convergent, it is convergent.
\end{theorem}
\begin{proof}
    Use \theoref{cauchy_convergent_criterion}.
\end{proof}


\begin{theorem}[\cindex{ratio test}]
    For a real series $\sum_{i=1}^\infty a_i$ with $a_i \neq 0$,
    \begin{itemize}
        \item If $\displaystyle \varlimsup_{j \rightarrow \infty} \absolutevalue{\frac{a_{j+1}}{a_j}} < 1$, the series converges absolutely
        \item If $\displaystyle \varliminf_{j \rightarrow \infty} \absolutevalue{\frac{a_{j+1}}{a_j}} > 1$, the series diverges
    \end{itemize}
\end{theorem}

\begin{theorem}[\cindex{root test}]
    For a real series $\sum_{i=1}^\infty a_i$ with $a_i \neq 0$. Let $\displaystyle L = \varlimsup_{j \rightarrow \infty}  \sqrt[j]{\absolutevalue{a_j}}$, we have
    \begin{itemize}
        \item If $L < 1$, the series converges absolutely
        \item If $L > 1$, the series diverges
    \end{itemize}
\end{theorem}
\begin{proof}
    If $L < 1$, choose $a$ that $L < a < 1$. According to the definition of limit superior, there is $N$ that for $n > N$, we have $\sqrt[n]{\absolutevalue{a_n}} < a$, so $\absolutevalue{a_n} < a^n$.
\end{proof}

\begin{theorem}
    Let $\sequence{a_n}$ be a sequence with $a_i \neq 0$. We have
    \begin{equation}
        \varliminf_{j \rightarrow \infty} \absolutevalue{\frac{a_{j+1}}{a_j}} \leq \varliminf_{j \rightarrow \infty}  \sqrt[j]{\absolutevalue{a_j}} \leq \varlimsup_{j \rightarrow \infty}  \sqrt[j]{\absolutevalue{a_j}} \leq \varlimsup_{j \rightarrow \infty} \absolutevalue{\frac{a_{j+1}}{a_j}}
    \end{equation}
\end{theorem}


\begin{theorem}[\cindex{Raabe's test}]
    For a real series $\sum_{i=1}^\infty a_i$ with $a_i \neq 0$. Define a new sequence $\sequence{b_n}$ that
    \begin{equation}
        b_n = n \left( \absolutevalue{\frac{a_n}{a_{n+1}}} - 1 \right)
    \end{equation}
    
    Then:
    \begin{itemize}
        \item If $\varliminf_{j \rightarrow \infty} b_n > 1$, the series converges absolutely
        \item If $\varlimsup_{j \rightarrow \infty} b_n < 1$, the series diverges
    \end{itemize}
    
    \emph{Rabbe's test is useful when ratio and root test fail}. For example, the sum $\sum \frac{1}{n^2}$ will have limit $1$ in both root and ratio test.
\end{theorem}



% rearrangement of series

\section{Rearrangement}

Addition $+$ is not commutative when there are infinitely many summands.

\begin{theorem}
    Every rearrangement of absolutely convergent series is absolutely convergent and they have the same value as original series.    
\end{theorem}
\begin{proof}
    Since $\sum x_k$ is absolutely convergent, it is a Cauchy series. So for any $\epsilon$, there is $N$ that $\sum_{k=N+1}^\infty \absolutevalue{x_k} < \epsilon$.
    
    Let $\sigma$ be a permutation of $\naturalnumber$. Let $M = \max \set{\inverse{\sigma}(0), \inverse{\sigma}(1), \cdots, \inverse{\sigma}(N) }$. We have
    \begin{equation}
        \absolutevalue{\sum_{k=0}^m \absolutevalue{x_{\sigma (k)}} - \sum_{k=0}^N \absolutevalue{x_k} } \leq \absolutevalue{\sum_{k=0}^m x_{\sigma (k)} - \sum_{k=0}^N x_k } \leq \sum_{k=N+1}^\infty \absolutevalue{x_k} < \epsilon
    \end{equation}
    
    Take $m \rightarrow \infty$.
\end{proof}

\begin{theorem}[\cindex{Riemann rearrangement theorem}]
    Let $\sum_{i=1}^\infty a_i$ be a real series that converges conditionally. For any $L \in \realnumber$, there is a rearrangement of the series so that $\sum_{i=1}^\infty a_{\sigma (i)} = L$.
\end{theorem}
\begin{proof}
    
    Now construct the arrangement. Assume $a_i \neq 0$ and $L > 0$. Split $\sequence{a_i}$ into two parts:
    \begin{itemize}
        \item $\sequence{p_i}$ are all positive numbers
        \item $\sequence{q_i}$ are all negative numbers
    \end{itemize}
    
    According to \theoref{conditional_converge_requirement}, both $\sequence{p_i}$ and $\sequence{q_i}$ diverge. 
    
    First find $j_1$ that
    \begin{equation*}
        \begin{aligned}
            s &= p_1 + p_2 + \cdots + p_{j_1} \geq L \\
            s - p_{j_1} &\leq L
        \end{aligned}
    \end{equation*}
    
    It is always possible because $p_i$ diverges. Now find $i_i$ that
    \begin{equation*}
        \begin{aligned}
            s_{k_1} &= s + q_1 + q_2 + \cdots + q_{i_1} \leq L \\
            s - q_{i_1} &\geq L
        \end{aligned}
    \end{equation*}
    
    Keep creating these $s_{k_n}$. We have $s_{k_n} \leq L$ and $\absolutevalue{L - s_{k_n}} < - q_{i_n} \rightarrow 0$ (according to \theoref{convergent_series_is_null_sequence}).
\end{proof}





% power series

\section{Power Series}

\begin{definition}[\cindex{power series}]
    A power series is a function $f$ that the following series converges:
    \begin{equation}
        f(x) = \sum_{k=0}^n a_k \cdot x^k
    \end{equation}
\end{definition}

\begin{theorem}[\cindex{radius of convergence}]
    There is a $\rho \in [0, \infty]$ that power series $\sum a_k x^k$ converges absolutely if $\absolutevalue{x} < p$ and diverges if $\absolutevalue{x} > p$. $\rho$ is called the radius of convergence. It could be calculated by the \cindex{Hadamard's formula}:
    \begin{equation}
        \rho = \frac{1}{\displaystyle \varlimsup_{k \rightarrow \infty} \sqrt[k]{\absolutevalue{a_k}} } \in \closure{\realnumber}
    \end{equation}
    
    So the power series $\sum a_k x^k$ will converge if
    \begin{equation}
        x \in \left(- \frac{1}{\displaystyle \varlimsup_{k \rightarrow \infty} \sqrt[k]{\absolutevalue{a_k}} }, \frac{1}{\displaystyle \varlimsup_{k \rightarrow \infty} \sqrt[k]{\absolutevalue{a_k}} } \right)
    \end{equation}
    
    It could also be calculated by 
    \begin{equation}
        \rho = \lim_{k \rightarrow \infty} \absolutevalue{\frac{a_k}{a_{k+1}}}
    \end{equation}
    
    Note:
    \begin{itemize}
        \item $\rho$ could be $\infty$
        \item There is no statement of convergence on boundary that $x = \pm \rho$. Counterexamples are $\sum x^k$, $\sum \frac{1}{k} x^k$, $\sum \frac{1}{k^2} x^k$. On boundary condition $x=1$, the first two diverge and the third one converges.
    \end{itemize}
    
     
\end{theorem}






























































































































































































































































































































































































































































































































































































































































































































































