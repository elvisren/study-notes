\chapter{Exercise}


% find limit
\section{Find Limit}

\begin{theorem}
    There is equivalence relation when $x \rightarrow 0$:
    \begin{equation}
        x \sim \sin x \sim \tan x \sim \arcsin x \sim \arctan x \sim \ln (1+x) \sim e^x -1 \sim \frac{a^x - 1}{\ln a} \sim \frac{(1+x)^b -1}{b}
    \end{equation}
    
    And 
    \begin{equation}
        1 - \cos x \sim \frac{1}{2} x^2
    \end{equation}
\end{theorem}

\begin{example}
    When $a \rightarrow 1$ and $b \rightarrow \infty$, the limit of $a^b$ could be calculated as
    \begin{equation}
        a^b = (1 + (a-1))^b = \mleft(\mleft(1 + (a-1)\mright)^{\frac{1}{a-1}}\mright)^{(a-1)b} \rightarrow e^{(a-1)b}
    \end{equation}
    
    So the problem now becomes how to calculate $\lim (a-1)b$.
\end{example}


\begin{theorem}
    $f(x) \sim x$ when $x \rightarrow 0$. Prove that for $a > 0$, we have
    \begin{equation}
        \lim_{n \rightarrow \infty} \sum_{i=1}^n f\mleft ( \frac{2i-1}{n^2} a \mright ) = a
    \end{equation}
    
    Here $f$ could be $\sin x$, $\tan x$, $\arcsin x$, $e^x -1$, $\ln (1+x)$. For example, prove that
    \begin{equation}
        \lim_{n \rightarrow \infty} \prod_{i=1}^n \mleft( 1 + \frac{2i-1}{n^2} a^2 \mright) = e^{a^2}
    \end{equation}
\end{theorem}
\begin{proof}
    $a = \sum_{i=1}^n \frac{2i-1}{n^2} a$. Let $x_n = \sum_{i=1}^n f\mleft( \frac{2i-1}{n^2} a \mright)$. We have
    \begin{equation*}
        \absolutevalue{x_n - a} = \absolutevalue{ \sum_{i=1}^n f\mleft( \frac{2i-1}{n^2} a \mright) - \sum_{i=1}^n \frac{2i-1}{n^2} a} \leq \sum_{i=1}^n \absolutevalue{ f\mleft( \frac{2i-1}{n^2} a \mright) - \frac{2i-1}{n^2} a}
    \end{equation*}
    
    The next step is to prove that $\forall \epsilon > 0$, there is $m$ that $\absolutevalue{ f\mleft( \frac{2i-1}{m^2} a \mright) - \frac{2i-1}{m^2} a} < \frac{2i-1}{m^2} \epsilon$, so the total sum is less that $\epsilon$.
\end{proof}

\begin{example}
    The limit of $\displaystyle \sum_{i=1}^n \frac{1}{n}f\mleft(\frac{i}{n}\mright)$ could be changed to integration $\displaystyle \int_{0}^1 f(x)$. For example:
    \begin{equation}
        \lim_{n \rightarrow \infty} \mleft( \frac{1}{n+1} + \frac{1}{n+2} + \cdots + \frac{1}{n+n} \mright) = \lim_{n \rightarrow \infty} \sum_{i=1}^n \frac{1}{\displaystyle 1+\frac{i}{n}} \times \frac{1}{n} = \int_0^1 \frac{1}{1+x} \dif x = \ln 2
    \end{equation}
    
    Tricks: we could choose any $\epsilon \in \mleft[\frac{i}{n}, \frac{i+1}{n}\mright]$ in $\displaystyle \sum_{i=1}^n \frac{1}{n}f(\epsilon)$.
\end{example}


\begin{example}
    Famous inequalities:
    \begin{equation}
        \frac{1}{1+n} < \ln \mleft(1+\frac{1}{n} \mright) < \frac{1}{n}
    \end{equation}
    \begin{equation}
        2(\sqrt{n+1} - \sqrt{n}) = \frac{2}{\sqrt{n} + \sqrt{n+1}} < \frac{1}{\sqrt{n}} < \frac{2}{\sqrt{n} + \sqrt{n-1}} = 2(\sqrt{n} - \sqrt{n-1} )
    \end{equation}
\end{example}

\begin{example}
    For $0 < a < 1$, $b > 1$, $k \in \naturalnumber$, $n \rightarrow \infty$, we have
    \begin{equation}
        \ln \ln n \ll \ln n \ll n^a \ll n^k \ll b^n \ll n! \ll n^n
    \end{equation}
\end{example}

\begin{theorem}[\cindex{Euler constant}]
    The Euler constant is defined as
    \begin{equation}
        \gamma = \lim_{n \rightarrow \infty} \mleft(1 + \frac{1}{2} + \frac{1}{3} + \cdots + \frac{1}{n} - \ln n \mright) \approx 0.577215664
    \end{equation}
    
    Note: do not confuse with $\ln 2 = 1 - \frac{1}{2} + \frac{1}{3} - \frac{1}{4} + \cdots$.
\end{theorem}

\begin{theorem}
    If $\displaystyle \lim_{n \rightarrow \infty}(b_{n+1} - b_n) = l$, then $\displaystyle \lim_{n\rightarrow \infty} \frac{b_n}{n} = l$.
\end{theorem}


\begin{theorem}[Stolz-Cesaro theorem, $\frac{*}{\infty}$ case]\label{stolz_infity_case}
    Let $\sequence{a_n}$ and $\sequence{b_n}$ be two sequence. Assume that $\sequence{b_n}$ is strictly monotone and divergent, and the following limit exists:
    \begin{equation}
        \lim_{n \rightarrow \infty} \frac{a_{n+1} - a_n}{b_{n+1} - b_n} = l
    \end{equation}
    
    Then the following limit exists:
    \begin{equation}
        \lim_{n \rightarrow \infty} \frac{a_n}{b_n} = l
    \end{equation}
\end{theorem}



\begin{theorem}[Stolz-Cesaro theorem, $\frac{0}{0}$ case]
    Let $\sequence{a_n}$ and $\sequence{b_n}$ be two sequence. Assume that $a_n \rightarrow 0$ and $b_n \rightarrow 0$, and $\sequence{b_n}$ is strictly decreasing. If the following limit exists:
    \begin{equation}
        \lim_{n \rightarrow \infty} \frac{a_{n+1} - a_n}{b_{n+1} - b_n} = l
    \end{equation}
    
    Then the following limit exists:
    \begin{equation}
        \lim_{n \rightarrow \infty} \frac{a_n}{b_n} = l
    \end{equation}
\end{theorem}

\begin{theorem}[Cauchy result]\label{cauchy_result}
    If $x_n \rightarrow l$, then
    \begin{equation}
        \lim_{n \rightarrow \infty} \frac{x_1 + x_2 + \cdots + x_n}{n} = l
    \end{equation}
\end{theorem}


\begin{theorem}
    Assume the sequence $\sequence{a_n}$ satisfies that $0 < a_1 < 1$ and $a_{n+1} = a_n (1- a_n)$. Prove that $\displaystyle \lim_{n \rightarrow \infty} n a_n = 1$.
\end{theorem}
\begin{proof}
    calculate $\lim \frac{n}{\frac{1}{a_n}}$ use \theoref{stolz_infity_case}.
\end{proof}

\begin{example}
    Tricks in proving expression involves $\lim \sum f(\sqrt{n^2 + k})$. The inequalities we could use are:
    \begin{equation}
        n^2 < n^2 + k < n^2 + n < n^2 + 2n + 1
    \end{equation} 
    
    Sometimes we could change the $n$, such as $k^2 + k < n^2 + k$. For example, when calculating $S = \lim \sum \frac{1}{n + \sqrt{k}}$, we have $S < \lim \sum \frac{1}{k + \sqrt{k}} = \lim \sum \mleft( \frac{1}{\sqrt{k}} - \frac{1}{\sqrt{k} + 1} \mright)$
\end{example}








































































































































































































































































































































































































































































































































































































































































































































































































































































































































