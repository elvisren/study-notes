\chapter{Exercise}


% find limit
\section{Find Limit}

\begin{theorem}
    There is equivalence relation when $x \rightarrow 0$:
    \begin{equation}
        x \sim \sin x \sim \tan x \sim \arcsin x \sim \arctan x \sim \ln (1+x) \sim e^x -1 \sim \frac{a^x - 1}{\ln a} \sim \frac{(1+x)^b -1}{b}
    \end{equation}
    
    And 
    \begin{equation}
        1 - \cos x \sim \frac{1}{2} x^2
    \end{equation}
\end{theorem}

\begin{example}
    When $a \rightarrow 1$ and $b \rightarrow \infty$, the limit of $a^b$ could be calculated as
    \begin{equation}
        a^b = (1 + (a-1))^b = \left(\left(1 + (a-1)\right)^{\frac{1}{a-1}}\right)^{(a-1)b} \rightarrow e^{(a-1)b}
    \end{equation}
    
    So the problem now becomes how to calculate $\lim (a-1)b$.
\end{example}


\begin{theorem}
    $f(x) \sim x$ when $x \rightarrow 0$. Prove that for $a > 0$, we have
    \begin{equation}
        \sum_{i=1}^\infty f\left ( \frac{2i-1}{n^2} a \right ) = a
    \end{equation}
    
    Here $f$ could be $\sin x$, $\tan x$, $\arcsin x$, $e^x -1$, $\ln (1+x)$. For example, prove that
    \begin{equation}
        \lim_{n \rightarrow \infty} \prod_{i=1}^n \left( 1 + \frac{2i-1}{n^2} a^2 \right) = e^{a^2}
    \end{equation}
\end{theorem}
\begin{proof}
    $a = \sum_{i=1}^n \frac{2i-1}{n^2} a$. Let $x_n = \sum_{i=1}^n f\left( \frac{2i-1}{n^2} a \right)$. We have
    \begin{equation*}
        \absolutevalue{x_n - a} = \absolutevalue{ \sum_{i=1}^n f\left( \frac{2i-1}{n^2} a \right) - \sum_{i=1}^n \frac{2i-1}{n^2} a} \leq \sum_{i=1}^n \absolutevalue{ f\left( \frac{2i-1}{n^2} a \right) - \frac{2i-1}{n^2} a}
    \end{equation*}
    
    The next step is to prove that $\forall \epsilon > 0$, there is $m$ that $\absolutevalue{ f\left( \frac{2i-1}{m^2} a \right) - \frac{2i-1}{m^2} a} < \frac{2i-1}{m^2} \epsilon$, so the total sum is less that $\epsilon$.
\end{proof}

\begin{example}
    The limit of $\displaystyle \sum_{i=1}^n \frac{1}{n}f\left(\frac{i}{n}\right)$ could be changed to integration $\displaystyle \int_{0}^1 f(x)$. For example:
    \begin{equation}
        \lim_{n \rightarrow \infty} \left( \frac{1}{n+1} + \frac{1}{n+2} + \cdots + \frac{1}{n+n} \right) = \lim_{n \rightarrow \infty} \sum_{i=1}^n \frac{1}{\displaystyle 1+\frac{i}{n}} \times \frac{1}{n} = \int_0^1 \frac{1}{1+x} \dif x = \ln 2
    \end{equation}
    
    Tricks: we could choose any $\epsilon \in \left[\frac{i}{n}, \frac{i+1}{n}\right]$ in $\displaystyle \sum_{i=1}^n \frac{1}{n}f(\epsilon)$.
\end{example}


\begin{example}
    Famous inequalities:
    \begin{equation}
        \frac{1}{1+n} < \ln \left(1+\frac{1}{n} \right) < \frac{1}{n}
    \end{equation}
    \begin{equation}
        2(\sqrt{n+1} - \sqrt{n}) = \frac{2}{\sqrt{n} + \sqrt{n+1}} < \frac{1}{\sqrt{n}} < \frac{2}{\sqrt{n} + \sqrt{n-1}} = 2(\sqrt{n} - \sqrt{n-1} )
    \end{equation}
\end{example}

\begin{example}
    For $0 < a < 1$, $b > 1$, $k \in \naturalnumber$, $n \rightarrow \infty$, we have
    \begin{equation}
        \ln \ln n \ll \ln n \ll n^a \ll n^k \ll b^n \ll n! \ll n^n
    \end{equation}
\end{example}












































































































































































































































































































































































































































































































































































































































































































































































































































































































































