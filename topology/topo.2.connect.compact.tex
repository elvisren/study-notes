\section{Connectedness and Compactness}

\subsection{Connected Spaces}

\begin{definition}[\cindex{separation}]
    A separation of $X$ is a pair $U,V$ of disjoint nonempty open subsets of $X$ whose union is $X$.
\end{definition}

\begin{definition}[\cindex{connected}]
    A space $X$ is connected if there is no separation for it.
\end{definition}

\begin{theorem}
    A space $X$ is connected if and only if the only open and closed sets are $X$ and $\emptyset$.
\end{theorem}

\begin{theorem}
    A subspace $Y$ of $X$ is connected if a pair $A,B$ of subsets of $Y$ whose union is $Y$, neither of which contains the limit point of the other.    
\end{theorem}

\begin{theorem}
    If $C,D$ forms a separation of $X$, and if $Y$ is a connected subspace of $X$, then $Y$ lies entirely within either $C$ or $D$.
\end{theorem}

\begin{theorem}
The union of a collection of connected subspace of $X$ that have a point in common is connected.    
\end{theorem}

\begin{theorem}
Let $A$ be a connected subspace of $X$. If $A \subset B \subset \closure{A}$, then $B$ is also connected. So the result of adding limit points to a connected subspace is connected.
\end{theorem}
\begin{proof}
    Assume there is a separation of $B$. Then $A$ must be in one of them, which is a closed set.
\end{proof}

\begin{theorem}
The image of a connected space under continuous map is connected.    
\end{theorem}

\begin{theorem}
A finite cartesian product of connected space is connected.
\end{theorem}
\begin{proof}
For $X,Y$, find a line $x \times Y$. This line is homomorphic to $Y$ and connected. Then find a line $X \times y$ which is connected. They share a point $(x,y)$.
\end{proof}

\begin{example}[$\realnumber^\omega$ is not connected in box topology]
    Let $A$ be the set of all bounded sequence, and $B$ all unbounded sequence. $A,B$ are open and they form the separation.
\end{example}

\begin{example}[$\realnumber^\omega$ is connected in product topology]
    Assume $\realnumber$ is connected. Let $\realnumber_n$ be the set of all $(x_1, x_2, ...)$ that $x_{i > n} = 0$. It is homeomorphic to $\realnumber^n$ and connected. They share the same point $(0,0,...)$ so $\bigcup_i \realnumber_i$ is connected. $\forall x \in \realnumber^\omega, x \in U$, prove $U \cap \closure{\bigcup_i \realnumber_i} \neq \emptyset$ in product topology.
\end{example}


\subsection{$\realnumber$ is Connected}

\begin{definition}[\cindex{linear continuum}]
    A simply ordered set $\mathcal{L}$ having more than one element is called linear continuum if the following hold:
    \begin{enumerate}
        \item $\mathcal{L}$ has the least upper bound property.
        \item $x < y \Rightarrow \exists z, x < z < y$.
    \end{enumerate}
\end{definition}

\begin{theorem}
    If $\mathcal{L}$ is a linearly continuum topology, then $\mathcal{L}$ is connected, so are the convex subset in $\mathcal{L}$.
\end{theorem}
\begin{proof}
    Assume there is a separation $A,B$ of $Y$. Find $c = \supremum{A}$. If $c \in B$, find conflict in $(d, c]$. If $c \in A$, find conflict in $[c, e)$.
\end{proof}

\begin{theorem}
    $\realnumber$ is connected
\end{theorem}

\begin{theorem}[\cindex{intermediate value theorem}]
    Let $f : X \rightarrow Y$ be a continuous map, where $X$ is connected and $Y$ is an ordered set. If $a,b \in X$ and $f(a) < r < f(b)$, $\exists c \in X, f(c) = r$.
\end{theorem}
\begin{proof}
    Define two open set $f(x) \cap (-\infty, r)$ and $f(x) \cap (r, +\infty)$. They are a separation if there is no $c$.
\end{proof}

\begin{definition}[\cindex{path}]
    A path in $X$ from $x$ to $y$ is a continuous map $f: [a,b] \rightarrow X$ that $f(a) = x$ and $f(b) = y$.
    
    A space is said to be \cindex{path connected} if every pair of points of $X$ can be joined by a path in $X$.
\end{definition}

\begin{example}
    The ordered square $I_o^2$ is connected but not path connected. So is it for the \cindex{topologist's sine curve} $\closure{S} = \closure{\set{\displaystyle x \times \sin \left(\frac{1}{x} \right) | 0 < x \leq 1}}$. So if $X$ is path connected, $\closure{X}$ may not be path connected.
\end{example}


\subsection{Components}

\begin{definition}[\cindex{component}s]
    Define a equivalent relation $x \sim y$ if there is a connected subspace of $X$ that contains both $x$ and $y$. The equivalent class is called components of $X$.
\end{definition}

\begin{theorem}
The components of $X$ are connected disjoint subspace of $X$ whose union is $X$. Each nonempty connected subspace of $X$ intersects only one of them.    
\end{theorem}

\begin{definition}[\cindex{path component}s]
    Like component, but the connected subspace is now path connected.
\end{definition}

Note: 
\begin{enumerate}
    \item Each component of $X$ is closed in $X$.
    \item The component needs not be open. ($\mathbb{Q}$).
    \item Path component could be neither open or closed.
\end{enumerate}

\begin{definition}[\cindex{locally connected}]
    A space is locally connected at $x$ if for all neighborhood $U$ of $x$, there is a connected neighborhood $V$ of $x$ that $V \subset U$. It is called locally connected if it is locally connected at every point.
\end{definition}

\begin{definition}[\cindex{locally path connected}]
    A space is locally path connected at $x$ if for all neighborhood $U$ of $x$, there is a path connected neighborhood $V$ of $x$ that $V \subset U$. It is called locally path connected if it is locally path connected at every point.
\end{definition}

\begin{theorem}
A space $X$ is locally connected if and only if for every open set $U$ of $X$, each component of $U$ is open in $X$.    
\end{theorem}
\begin{proof}
    The open components are for any open set $U$, not $X$.
\end{proof}

\begin{theorem}
A space $X$ is locally path connected if and only if for every open set $U$ of $X$, each path component of $U$ is open in $X$.    
\end{theorem}

\begin{theorem}
Each path component of $X$ lies in component of $X$. If $X$ is locally path connected, the components and the path components are the same.    
\end{theorem}



\subsection{Compact Spaces}

\begin{definition}[\cindex{cover}]
    A collection $\mathcal{A}$ of subsets of a space $X$ is said to cover $X$ if $\bigcup \mathcal{A} = X$. It is called \cindex{open covering} of $X$ if its elements are all open subsets of $X$.
\end{definition}

\begin{definition}[\cindex{compact}]
    A space $X$ is said to be compact if every open covering of $X$ contains a finite subcollection that covers $X$.
\end{definition}

\begin{theorem}
    Every close subspace of a compact space is compact.    
\end{theorem}

\begin{theorem}
    Every compact subspace of Hausdorff space is closed.    
\end{theorem}

\begin{theorem}
    If $Y$ is a compact subspace of the Hausdorff space $X$ and $x_0$ is not in $Y$, then there exists disjoint open sets $U$ and $V$ of $X$ containing $x_0$ and $Y$.    
\end{theorem}

\begin{theorem}
    The image of a compact space under a continuous map is compact.    
\end{theorem}

\begin{theorem}
    Let $f: X \rightarrow Y$ be a bijective continuous function. If $X$ is compact and $Y$ is Hausdorff, then $f$ is a homeomorphism.    
\end{theorem}

\begin{theorem}
    The product of finitely many compact space is compact.    
\end{theorem}

\begin{theorem}[the tube lemma]
    If $Y$ is compact. If $N$ is an open set of $X \times Y$ containing the slice $x_0 \times Y$, then $N$ contains some tube $W \times Y$ that contains $x_0 \times Y$, where $W$ is a neighborhood of $x_0$.
\end{theorem}

\begin{definition}
    A collection $\mathbf{C}$ of subsets of $X$ is said to have the \cindex{finite intersection property} if for every finite subcollection $C_i$ of $\mathbf{C}$, $\bigcap_{i \leq n} C_i \neq \emptyset$ .
\end{definition}

\begin{theorem}
    $X$ is compact if and only if for every $\mathbf{C}$ of closed sets in $X$ having the finite intersection property, $\bigcap_{C \in \mathbf{C}} C \neq \emptyset$.
    
    If $C_i$ form a nested sequence $C_1 \supset C_2 \supset C_3 ...$ of closed sets in a compact space $X$, $\bigcap_{i \in \mathbb{Z}_{+}} C_i \neq \emptyset$.
\end{theorem}


\subsection{Compact Subspace of $\realnumber$}

\begin{theorem}
    Let $X$ be a simply ordered set with least upper bound property. In the order topology, every close interval in $X$ is compact.    
    
    Conclusion:
    \begin{enumerate}
        \item Every closed interval in $\realnumber$ is compact.
        \item A subspace $A$ of $\realnumber^n$ is compact if and only if it is closed and bounded in euclidean metrid $d$ or the square metric $\rho$.
    \end{enumerate}
\end{theorem}

\begin{theorem}[\cindex{extreme value theorem}]
    Let $f: X \rightarrow Y$ be continuous, where $Y$ is an ordered set in the order topology. If $X$ is compact, then there exists $c,d \in X$ that $\forall x \in X, f(c) \leq f(x) \leq f(d)$.
\end{theorem}

\begin{definition}
    The distance from $x$ to $A$ is $d(x,A) = \infimum{d(x,a) | a \in A}$.
\end{definition}

\begin{theorem}[\cindex{Lebesgue number}]
    Let $\mathcal{A}$ be an open covering of the metric space $(X, d)$. If $X$ is compact, $\exists \delta > 0$ that for each subset of $X$ having diameter less than $\delta$, there exists an element of $\mathcal{A}$ containing it. The $\delta$ is called the Lebesgue number for the covering $\mathcal{A}$.
\end{theorem}
\begin{proof}
    Define a continuous function $f : X \rightarrow \realnumber$. For each finite covering $\set{A_i}$, choose $C_i = X - A_i$. $\forall x, f(x) = \frac{1}{n} \sum d(x,C_i)$. Then $f(x) > 0$ and has a minimum value $\delta$.
\end{proof}

\begin{definition}
    A function $f$ from metric space $(X, d_X)$ to the metric space $(Y, d_Y)$ is \cindex{uniformly continuous} if $\forall \epsilon > 0, \exists \delta > 0, \forall a,b \in X, d_X (a,b) < \delta \Rightarrow d_Y \left(f(a), f(b)\right) < \epsilon$.
\end{definition}
\begin{theorem}
    Let $f: X \rightarrow Y$ be a continuous function. $X$ is compact metric and $Y$ is metric. Then $f$ is uniformly continuous.    
\end{theorem}
\begin{proof}
    The compactness in $X$ is to use the Lebesgue number. The metrics in $X$ and $Y$ is needed to generate all the $\epsilon$-balls.
\end{proof}



\begin{definition}[\cindex{isolated point}]
    A point $x$ of $X$ is an isolated point of $X$ if the one point set ${x}$ is open in $X$.
\end{definition}

\begin{theorem}
    Let $X$ be a nonempty compact Hausdorff space. If $X$ has no isolated points, then $X$ is uncountable.
    So closed interval in $\realnumber$ is uncountable.
\end{theorem}




\subsection{Limit Point Compactness}

\begin{definition}[\cindex{limit point compact}]
    A space $X$ is limit point compact if every infinite subset of $X$ has a limit point. It is also called \cindex{Frechet compactness} or \cindex{Bolzano-Weierstrass property}.
\end{definition}

\begin{theorem}
    Compactness implies limit point compactness, but not conversely.    
\end{theorem}
\begin{proof}
    If a subspace contains no limit point, it contains all its limit points and hence closed.
\end{proof}

\begin{definition}[\cindex{sequentially compact}]
    A space $X$ is said to be sequentially compact if every sequence of points of $X$ has a convergent subsequence.
\end{definition}

\begin{theorem}
    Let $X$ be a metrizable space. Then the followings are equivalent:
    \begin{enumerate}
        \item $X$ is compact.
        \item $X$ is limit point compact.
        \item $X$ is sequentially compact.
    \end{enumerate}    
\end{theorem}
\begin{proof}
    For (2) $\rightarrow$ (3), use the metrics to generate a sequence of $\frac{1}{n}$ $\epsilon$-balls.
    
    For (3) $\rightarrow$ (1), first prove it has Lebesgue number. Assume the converse, find a series of open sets and $x_i$, and use sequential compact to find contradiction that these $x_i$ are contained in a $\epsilon$-balls. Then prove there is a finite open covering by $\epsilon$-balls. Assume all the covering are infinite, find a converging subsequence. Finally prove it by first find a Lebesgue number, and then find a finite open cover.
\end{proof}


\subsection{Local Compactness}

\begin{definition}[\cindex{locally compact}]
    A space $X$ is said to be locally compact at $x$ if there is some compact subspace $C \subset X$ that contains a neighborhood of $x$. If $X$ is locally compact at every points, it is locally compact.
\end{definition}


\begin{theorem}
    $X$ is locally compact Hausdorff if and only if $\exists Y$ that:
    \begin{enumerate}
        \item $Y = X \cup \set{\infty}$.
        \item $Y$ is a compact Hausdorff space.
    \end{enumerate}
    
    If there are $Y$ and $Y'$ satisfying these conditions, there there is a homeomorphism between them. This $\infty$ is a limit point of $X$.
\end{theorem}
\begin{proof}
    The open set in $Y$ is defined by:
    \begin{enumerate}
        \item The open sets of $X$.
        \item $Y -C$ where $C$ is compact in $X$.
    \end{enumerate}
    
    Then $Y$ is a compact space. For any covering of $Y$, it must cover $\infty$ by $U$. Then $Y-U$ is compact and has finite cover.
\end{proof}

\begin{definition}[\cindex{compactification}]
    If $Y$ is a compact Hausdorff space and $X \subset Y$ and $\closure{X} = Y$, then $Y$ is said to be a compactification of $X$. If $\absolutevalue{Y - X} = 1$, $Y$ is called the \cindex{one-point compactification} of $X$.
\end{definition}

\begin{theorem}
    Let $X$ be a Hausdorff space. $X$ is locally compact if and only if $\forall x \in X$ and a neighborhood $U$ of $x$, there is a neighborhood $V$ of $x$ that $\closure{V}$ is compact and $\closure{V} \subset U$.
    The key:
    \begin{enumerate}
        \item $x \in V$.
        \item $\closure{V}$ is compact.
        \item $\closure{V} \subset U$.
    \end{enumerate}
\end{theorem}
\begin{proof}
    Let $Y$ be the one-point compactification of $X$. So $C = Y - U$ is a closed subspace and compact. So there are $v$ and $W$ that $x \in v$ and $C \subset W$, and $V \cap W = \emptyset$. So $\closure{V}$ is compact in $Y$ and $\closure{V} \subset U$.
\end{proof}

\begin{theorem}
    Let $X$ be a locally compact Hausdorff and $A \subset X$. If $A$ is closed in $X$ or open in $X$, then $A$ is locally compact.
\end{theorem}
\begin{proof}
    If $A$ is open, find $V$ that $\closure{V} \subset A$ and is compact. If $A$ is closed, find a compact $x \in C$ and calculate $C \cap A$.
\end{proof}

\begin{theorem}
    $X$ is homeomorphic to an open subspace of a compact Hausdorff space if and only if $X$ is locally compact Hausdorff.
\end{theorem}
\begin{proof}
    If $X$ is locally compact Hausdorff, find its one-point compactification. A compact Hausdorff space is also locally compact Hausdorff, so its open subset is locally compact Hausdorff.
\end{proof}



















