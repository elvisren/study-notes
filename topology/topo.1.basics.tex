\section{Basics}

\subsection{Definitions}

\begin{definition}[\cindex{topology}]
    A topology on a set $X$ is a collection $\mathcal{T}$ of subsets of $X$ having the following property:
    \begin{enumerate}
        \item $\emptyset$ and $X$ are in $\mathcal{T}$.
        \item The union of any subcollection of $\mathcal{T}$ is in $\mathcal{T}$.
        \item the intersection of finite subcollection of $\mathcal{T}$ is in $\mathcal{T}$.
    \end{enumerate}

The set $X$ with a topology is called \cindex{topological space}. The set $U$ in $\mathcal{T}$ is called \cindex{open set}.
\end{definition}

\begin{definition}[\cindex{closed set}]
    A subset $A$ of a topological space $X$ is closed if the set $X - A$ is open.
\end{definition}

\begin{definition}
    Let $X$ be a topological space. Then:
    \begin{enumerate}
        \item $\emptyset$ and $X$ are closed.
        \item Arbitrary intersections of closed sets are closed.
        \item Finite unions of closed sets are closed.
    \end{enumerate}
\end{definition}

\begin{definition}[\cindex{basis}]
    A basis for a topology is a collection $\mathcal{B}$ of subsets of $X$ that:
    \begin{enumerate}
        \item For each $x \in X$, there is at least one basis element $B$ containing $x$.
        \item If $x \in B_1 \cap B_2$, there is a basis element $B_3$ that $x \in B_3 \subset B_1 \cap B_2$.
    \end{enumerate}
\end{definition}

The topology $\mathcal{T}$ generated by basis $\mathcal{B}$ is: a subset $U$ of $X$ is open in $X$ if for each $x \in U$, there is a basis element $B \in \mathcal{B}$ such that $x \in B \subset U$.


\begin{definition}[\cindex{subbasis}]
    A subbasis $\mathcal{S}$ for a topology on $X$ is a collection of subsets of $X$ whose union equals $X$. The topology generated by the subbasis $\mathcal{S}$ is the collection $\mathcal{T}$ of all unions of finite intersections of elements of $\mathcal{S}$.
\end{definition}


Common topologies:
\begin{itemize}
    \item \cindex{discrete topology}: the collection of all subsets of $X$.
    \item \cindex{finite complement topology}: a collection of subsets $U$ of $X$ that $X - U$ is either finite or all of $X$.
    \item \cindex{standard topology} on order set: a collection of all open intervals $(a,b) = \set{x | a < x < b}$.
    \item \cindex{lower limit topology} \cindex{$\mathcal{R}_l$}: $[a,b) = \set{x | a \leq x < b}$.
    \item \cindex{K-topology}: let $K = \set{\displaystyle \frac{1}{n} | n \in \mathbb{Z}_{+}}$. The basis are the collection of all open intervals $(a,b)$ and $(a,b) - K$.
\end{itemize}


\begin{theorem}
    Let $\mathcal{B}$ be a basis for a topology $\mathcal{T}$ of $X$. Then $\mathcal{T}$ equals the collection of all unions of elements of $\mathcal{B}$.
\end{theorem}

\begin{theorem}
    Suppose $\mathcal{C}$ is a collection of open sets of $X$ such that for each open set $U$ of $X$ and $x \in U$, there is an element $C$ of $\mathcal{C}$ that $x \in C \subset U$. Then $\mathcal{C}$ is a basis for the topology of $X$.    
\end{theorem}


How to understand the relation among topology, basis and subbasis:
\begin{itemize}
    \item topology is closed with arbitrary union and finite intersection
    \item basis only needs to do arbitrary union
    \item subbasis has to do finite intersection
\end{itemize}

\begin{definition}[\cindex{order topology}]
    Let $X$ be a set with a simple order relation. Let $\mathcal{B}$ be the collection of all sets:
    \begin{enumerate}
        \item All open intervals $(a,b)$ in $X$.
        \item All intervals $[a_0, b)$ where $a_0$ is the smallest element (if any) of $X$.
        \item All intervals $(a, b_0]$ where $b_0$ is the largest element (if any) of $X$.
    \end{enumerate}
    The topology generated by basis $\mathcal{B}$ is called the order topology.
\end{definition}

\begin{definition}[\cindex{subspace topology}]
    Let $(X, \mathcal{T})$ be a topology. If $Y$ is a subset of $X$, the collection $\mathcal{T}_{Y} = \set{Y \cap U | U \in \mathcal{T}}$ is a topology on $Y$ which is called the subspace topology.
\end{definition}

\begin{theorem}
    If $\mathcal{B}$ is a basis for the topology of $X$ then the collection $\mathcal{B}_Y = \set{B \cap Y | B \in \mathcal{B}}$ is a basis for the subspace topology on $Y$.
\end{theorem}

\begin{theorem}
    A subset $Y$ of $X$ is \cindex{convex} in $X$ if for each pair of points $a < b$ of $Y$, the entire interval $(a,b)$ of points of $X$ is in $Y$. 
    Let $X$ be an ordered set in order topology, and let $Y$ be a subset of $X$ that is convex in $X$. Then the order topology on $Y$ is the same as the topology inherits as a subspace of $X$.
\end{theorem}

\begin{definition}
    Given a subset $A$ of a topological space $X$, the \cindex{interior} $\mathring{A}$ of $A$ is defined as the union of all open sets contained in $A$, and the \cindex{closure} $\closure{A}$ of $A$ is defined as the intersection of all closed sets containing $A$.
\end{definition}

\begin{definition}
    A \cindex{neighborhood} of $x$ is an open set $U$ containing $x$.
\end{definition}

\begin{theorem}
Let $A$ be a subset of the topological space $X$. Then $x \in \overline{A}$ if and only if every open set (or basis) $U$ containing $x$ intersects $A$.
\end{theorem}

\begin{definition}
    Let $A$ be a subset of topological space $X$. $x$ is a \cindex{limit point} of $A$ if every neighborhood of $x$ intersects $A$ in some point other than $x$. In other words,  $x$ is a limit point of $A$ if it belongs to $\closure{A - \set{x}}$.
\end{definition}

\begin{theorem}
    Let $A'$ be the set of all limit points of $A$. Then $\closure{A} = A \cup A'$.
\end{theorem}
\begin{proof}
    Let $x \in \closure{A}$. $\forall U, x \in U, U \cap A = C \neq \emptyset$, so either $x \in C \Rightarrow x \in A$, or $x \notin C \Rightarrow x \in A'$.
\end{proof}

\begin{theorem}
A subset of $A$ is closed if and only if it contains all its limit points.    
\end{theorem}
\begin{definition}
    A topological space $X$ is \cindex{Hausdorff space} if for each pair $x_1, x_2$ of distinct points of $X$, there exists neighborhood $U_1$ and $U_2$ of $x_1$ and $x_2$ that are disjoint.
\end{definition}

\begin{definition}[\cindex{$T_1$} axiom]
Every finite point set is closed.    
\end{definition}

\begin{theorem}
Hausdorff space is $T_1$.
\end{theorem}


\begin{theorem}
Let $X$ be a space satisfying the $T_1$ axiom. Let $A$ be a subset of $X$. Then the point $x$ is a limit point of $A$ if and only if every neighborhood of $x$ contains infinite many points of $A$.    
\end{theorem}
\begin{proof}
    Assume the intersection is finite, which is $\set{x_i}$. Calculate the set $U \cap (X - \set{x_i})$.
\end{proof}


\begin{definition}
    A sequence $x_1, x_2, ... $ of points of space $X$ \cindex{converge}s to point $x$ of $X$ if for each neighborhood $U$ of $x$, there is a positive integer $N$ such that $x_n \in U$ for all $n \geq N$.
\end{definition}

\begin{theorem}
If $X$ is a Hausdorff space, then a sequence of points of $X$ converges to at most one point of $X$.    
\end{theorem}

\begin{theorem}
Properties of Hausdorff space:
\begin{itemize}
    \item Every simple ordered set is Hausdorff space in order topology.
    \item The produce of two Hausdorff space is a Hausdorff space.
    \item A subspace of a Hausdorff space is a Hausdorff space. 
\end{itemize}
\end{theorem}



\subsection{Continuous Functions}


\begin{definition}
    A function $f: X \rightarrow Y$ is \cindex{continuous} if for each open subset $V$ of $Y$, the set $f^{-1}(V)$ is an open subset of $X$.
\end{definition}

To prove a function $f$ is continuous, it is suffice to prove the inverse image of subbasis is open.

\begin{theorem}
Let $f: X \rightarrow Y$. Then the following are equivalent:
\begin{enumerate}
    \item $f$ is continuous.
    \item For every subset $A$ of $X$, one has $f(\closure{A}) \subset \closure{f(A)}$.
    \item For every closed set $B$ of $X$, the set $f^{-1}(B)$ is closed in $X$.
    \item For each $x\in X$ and each neighborhood $V$ of $f(x)$, there is a neighborhood $U$ of $x$ such that $f(U) \subset V$.
\end{enumerate}    

If the condition (4) holds for a point $x$ of $X$, then $f$ is continuous at the point $x$.
\end{theorem}
\begin{proof}
    For (1) $\rightarrow$ (2), $\forall y \in f(\closure{A}), \exists x \in \closure{A}, y = f(x)$. Take $U$ that $y \in U$, there is $V = f^{-1}(U)$ that is open and $x \in V$. Since $x \in \closure{A}$, $\exists t \in V \cap A$. So $f(t) \in U \cap f(A) \neq \emptyset$.
    
    For (2) $\rightarrow$ (3), $f(\closure{A}) \subset \closure{f(A)} = B$, so $f^{-1} \left(f(\closure{A}) \right) \subset f^{-1}(B) = A$. So $\closure{A} \subset f^{-1} \left(f(\closure{A}) \right) = A$.
\end{proof}


\begin{definition}
    Let $f:X \rightarrow Y$ be a bijection. if $f$ and $f^{-1}$ are continuous, $f$ is called \cindex{homeomorphism} and the map $f$ is a \cindex{topological imbedding}, or \cindex{imbedding} of $X$ in $Y$.
\end{definition}


\begin{theorem}[\cindex{pasting lemma}]
    Let $X = A \cup B$ where $A$ and $B$ are closed in $X$. Let $f: A \rightarrow Y$ and $g: B \rightarrow Y$ be continuous. If $\forall x \in A \cap B, f(x) = g(x)$, then $f$ and $g$ combine to give a continuous function $h: X \rightarrow Y$, defined by $\forall x \in A, h(x) = f(x) $ and $\forall x \in B , h(x) = g(x)$.
\end{theorem}

\begin{theorem}
Let $f: A \rightarrow X \times Y$ defined as $f(a) = \left(f_1(a), f_2(a) \right)$. Then $f$ is continuous if and only if $f_1$ and $f_2$ are continuous. The map $f_1$ and $f_2$ are called the \cindex{coordinate functions} of $f$.
\end{theorem}


\subsection{Product Space}

\begin{definition}
    Let $\set{A_{\alpha}}_{\alpha \in \mathcal{J}}$ be an indexed family of sets. The \cindex{cartesian product} $\displaystyle \prod_{\alpha \in \mathcal{J}} A_{\alpha}$ of this indexed family is defined to be the set of all functions $\set{f : \mathcal{J} \rightarrow \displaystyle \bigcup_{\alpha \in \mathcal{J}} A_{\alpha}}$ that $f(\alpha) \in A_{\alpha}$.
\end{definition}

\begin{definition}
    Let $\set{X_{\alpha}}_{\alpha \in \mathcal{J}}$ be an index family of topological spaces. The \cindex{box topology} generated by the basis of all sets of the form $\displaystyle \prod_{\alpha \in \mathcal{J}} U_{\alpha}$ where $U_{\alpha}$ is open (or a basis) in $X_{\alpha}$. 
\end{definition}

\begin{definition}[\cindex{projection mapping}]
    A projection mapping $\pi_\beta$ is defined as $\pi_\beta \left( (x_{\alpha})_{\alpha \in \mathcal{J}} \right) = x_\beta$.
\end{definition}

\begin{definition}
    Define $\mathcal{S}_{\beta} = \set{\pi_{\beta}^{-1}(U_{\beta}) | U_{\beta}\text{ is open in }X_{\beta}}$ and let $\mathcal{S} = \displaystyle \bigcup_{\beta \in \mathcal{J}} \mathcal{S}_{\beta}$. The topology generated by the subbasis $\mathcal{S}$ is called the \cindex{product topology}. So the product topology on $\displaystyle \prod_{\alpha \in \mathcal{J}} X_{\alpha}$ has the basis all sets of the form $\displaystyle \prod_{\alpha \in \mathcal{J}} U_{\alpha}$, where $U_{\alpha}$ is open in $X_{\alpha}$ and $U_{\alpha} = X_{\alpha}$ except for finite many value of $\alpha$.
\end{definition}

\begin{theorem}
Let $A_{\alpha} \subset X_{\alpha}$. $\displaystyle \prod \closure{A_{\alpha}} = \closure{\prod A_{\alpha}}$ in either box topology or product topology.
\end{theorem}


\begin{theorem}
Let $\displaystyle f : A \rightarrow \prod_{\alpha} X_{\alpha}$ be given by $f(a) = \left( f_{\alpha}(a) \right)$ where $f_{\alpha} : A \rightarrow X_{\alpha}$. In product topology $f$ is continuous if and only if $f_{\alpha}$ is continuous.
\end{theorem}

\begin{example}[counterexample in box topology where $f$ is not continuous]
    Define $f: \realnumber \rightarrow \realnumber^{\omega}$ by $f(t) = (t,t,t, ...)$. Define a open set in box topology $B = (-1, 1) \times (- \frac{1}{2}, \frac{1}{2}) \times  (- \frac{1}{3}, \frac{1}{3}) \times \cdots$. $f^{-1}(B)$ is not open in $R$.
\end{example}

\subsection{Metric Topology}

\begin{definition}
    A \cindex{metric} on a set $X$ is a function $d: X \times X \rightarrow \realnumber$ with the following property:
    \begin{enumerate}
        \item $d(x,y) \geq 0$.
        \item $d(x,y) = 0 \Leftrightarrow x = y$.
        \item $d(x,y) = d(y,x)$.
        \item $d(x,y) + d(y,z) \geq d(x,z)$.
    \end{enumerate}
    $d(x,y)$ is also called the \cindex{distance} between $x$ and $y$.
\end{definition}

\begin{definition}
    For a metric space $X$ with distance function $d$, the \cindex{$\epsilon$-ball} centered at $x$ is defined as $B_d(x, \epsilon) = \set{y | d(x,y) < \epsilon}$.
\end{definition}

\begin{definition}
    For a metric space $X$ with distance function $d$, the \cindex{metric topology} is defined by taking as basis all the $\epsilon$-balls $B_d(x,\epsilon)$ for all $x$ and $\epsilon > 0$. A topological space $X$ is \cindex{metrizable} if there exists a metric $d$ on $X$ that induces the same topology.
\end{definition}

\begin{definition}
    A subset $A$ of $X$ is \cindex{bounded} if $\exists m, \forall a_1, a_2 \in A: d(a_1, a_2) \leq m$. If $A$ is bounded and nonempty, the \cindex{diameter} of $A$ is defined as $\text{diam } A = \supremum{d(a_1, a_2)| a_1, a_2 \in A} $.
\end{definition}

\begin{definition}
    Let $X$ be a metric space with metric $d$. The \cindex{standard bounded metric} corresponding to $d$ is defined as $\closure{d}(x,y) = \text{min}\set{d(x,y), 1} $.
\end{definition}

\begin{definition}
    For $x \in \realnumber^n$, the \cindex{norm} of $x$ is $\norm{x} = \sqrt{x_1^2 + ... + x_n^2}$. The \cindex{euclidean metric} $d$ on $\realnumber^n$ is defined as $d(x,y) = \norm{x - y}$. The \cindex{square metric} $\rho$ is defined as $\rho (x,y) = \text{max}\set{\absolutevalue{x_1 - y_1}, ... , \absolutevalue{x_n - y_n} }$.
\end{definition}

\begin{definition}
    When $n \rightarrow \infty$, the euclidean and square metric fail to work. So need to define a metric that works for any $\mathcal{J}$. Define $\closure{d}(x,y) = \text{min}\set{\absolutevalue{x - y} , 1}$ and the \cindex{uniform metric} $\closure{\rho}(x,y) = \supremum{ \closure{d}(x_{\alpha}, y_{\alpha}) | \alpha \in \mathcal{J}} $. The topology generated by uniform metric is called \cindex{uniform topology}.
\end{definition}

\begin{theorem}
The uniform topology on $\realnumber_{\mathcal{J}}$ is finer than the product topology and coarser than the box topology. They are all different if $\mathcal{J}$ is infinite.
\end{theorem}
\begin{proof}
    For a basis in product topology, there are finite $B_i$ that is not $R$. Find the minimum $\epsilon$ and make $\closure{\rho} < \epsilon$. For the box topology, find basis that is half of the size of $\closure{\rho}$.
\end{proof}

\begin{theorem}[$\realnumber^{\omega}$ is metrizable]
    Define $\displaystyle D(x,y) = \supremum{\frac{\closure{d}(x_i, y_i )}{i}} $. Then $D$ is a metric that induces the product topology on $\realnumber^\omega$.
\end{theorem}
\begin{proof}
    (the metric is changed from $\closure{\rho}(x,y) = \supremum{ \closure{d}(x_{\alpha}, y_{\alpha})} $ to $\supremum{\frac{\closure{d}(x_i, y_i )}{i}} $) First need to prove $D$ is a metric by proving the triangle inequality. The trick is to choose a $N$ that $\frac{1}{N} < \epsilon$ so all $\realnumber$ is included in the $\epsilon$-ball.
\end{proof}

\begin{theorem}
Let $f: X \rightarrow Y$. Let $X$ and $Y$ be metrizable with $d_X$ and $d_Y$. $f$ is continuous is equivalent to the requirement that $\forall x \in X, \forall \epsilon > 0, \exists \delta > 0, d_X (x,y) < \delta \Rightarrow d_Y \left(f(x), f(y) \right) < \epsilon $.
\end{theorem}

\begin{theorem}[sequence lemma]
    Let $A \subset X$. If there is a sequence of points of $A$ converging to $x$, then $x \in \closure{A}$. The converse holds if $X$ is metrizable.
\end{theorem}
Note: the sequence lemma is quite useful in proving a topology space is not metrizable.

\begin{example}[$\realnumber^\omega$ is not metrizable in box topology]
    Let $A = \set{(x_1, x_2, ... ) | x_i > 0}$. We have $(0,0,0,...) \in \closure{A}$. For a sequence $a_n = (x_{n_1}, x_{n_2}, ...)$, the open set $\set{(- x_{1_1}, x_{1_1}) \times (- x_{2_2}, x_{2_2}), ...}$ contains no $a_i$.
\end{example}

\begin{example}[$\realnumber^{\mathcal{J}}$ is not metrizable in product topology]
    Assume $\mathcal{J}$ is uncountable. Let $A$ contains all $x$ that $\pi_{\beta}(x) = 1$ for all but finite many $\beta$. Then $(0,0,...)$ is in $\closure{A}$.
    
    Assume there is a sequence $a_n$ that converge to $(0,0,...)$. Let $J_n = \set{\beta | \pi_\beta (a_n) \neq 1 }$. Then $\Sigma_n \absolutevalue{J_n}$ is countable of finite value, which is countable. Because $\mathcal{J}$ is uncountable, there exists a $\gamma$ that $\forall i, \pi_\gamma (a_i) = 1$. Take a basis $\pi_{\gamma}^{-1}(-1,1)$, which contains $(0,0,...)$ but intersect with no $a_i$.
\end{example}



\begin{theorem}
    Let $f: X \rightarrow Y$. If $f$ is continuous, then for every convergent sequence $x_n \rightarrow x$ in $X$, the sequence $f(x_n) \rightarrow f(x)$. The converse holds if $X$ is metrizable (not $Y$).
\end{theorem}
\begin{proof}
    For a $A \subset X$,  $x_n \rightarrow x \Rightarrow x \in \closure{A}$. $f(x_n) \rightarrow f(x) \Rightarrow f(x) \in \closure{f(A)}$. What we need is a way to find a sequence of $x_n$, and this is why $X$ is metrizable.
\end{proof}


\begin{definition}
    A sequence $f_n$ \cindex{converge uniformly} to $f$ if $\forall \epsilon > 0,\forall x \in X, \exists N \in \mathbb{Z}_{+}, \forall n > N, d \left( f_n (x) , f(x) \right) < \epsilon$.
\end{definition}

\begin{theorem}
    Let $f_n: X \rightarrow Y$ be a sequence of continuous functions from $X$ into metric space $Y$. If $f_n$ converges uniformly to $f$, then $f$ is continuous. (no requirement on $X$)
\end{theorem}
\begin{proof}
    First find a $N$ that $f_n$ and $f$ are close enough. Then for $f_n$, find a $\epsilon$-ball small enough to restrict how big the $f$ is.
\end{proof}











