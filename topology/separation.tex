\chapter{Separation}



% countability

\section{Countability}

Here is the conclusion:


\begin{center} 
   \begin{tabular}{ccccc}
   & first-countable & second-countable & Lindel\"of & separable \\
    \hline
    subset & $\checkmark$ & $\checkmark$ & $\times$ (need closed subset) & $\times$ \\
    countable product & $\checkmark$ & $\checkmark$ & $\times$ & $\checkmark$ \\
    image of continuous function & $\times$ & $\times$ & $\checkmark$ & $\checkmark$ \\
    \hline
\end{tabular} 
\end{center}


\begin{definition}[\cindex{first countability axiom}]
    A space $X$ is said to have a \cindex{countable basis} at $x$ if there is a countable collection $B$ of neighborhoods of $x$ such that each neighborhood of $x$ contains at least one of the element of $B$.
    
    A space that has a countable basis at each of its points is said to satisfy the first countability axiom, or to be \emph{first-countable}.
\end{definition}

\begin{theorem}
    Every metrizable space satisfy the first countability axiom.    
\end{theorem}
\begin{proof}
    Choose $B(x, \frac{1}{n})$.
\end{proof}


\begin{theorem}[\cindex{sequence lemma}]\label{sequence_lemma}
    Let $A \subset X$. If there is a sequence of points of $A$ converging to $x$, then $x \in \closure{A}$. The converse holds if $X$ is first-countable.
\end{theorem}
\begin{proof}
    For the reverse, for neighborhood $B_{i<\infty}$ of $x\in \closure{A}$, $B_i \cap A \neq \emptyset$. So choose one $x_i \in B_i \cap A$.
\end{proof}


\begin{theorem}
    Let $f: X \rightarrow Y$. If $f$ is continuous, then for every convergent sequence $x_n \rightarrow x$ in $X$, the sequence $f(x_n) \rightarrow f(x)$. The converse holds if $X$ is first-countable (\emph{$Y$ does not need to be metrizable}).
\end{theorem}
\begin{proof}
    For a $A \subset X$,  $x_n \rightarrow x \Rightarrow x \in \closure{A}$. $f(x_n) \rightarrow f(x) \Rightarrow f(x) \in \closure{f(A)}$. Use \theoref{sequence_lemma}, for every $x \in \closure{A}$, we could find $x_i \rightarrow x$.
\end{proof}

\begin{definition}[\cindex{second countability axiom}]
    If a space $X$ has a countable basis for its topology, then $X$ is said to satisfy the second countability axiom, or to be \cindex{second-countable}.
\end{definition}

\begin{example}[\emph{$R^\omega$ is second-countable}]
    $R$ is second-countable. For every $a,b \in \rational$ and $a < b$, form basis $(a,b)$. So $R^\omega$ also has a countable basis.
\end{example}

\begin{theorem}
    The subspace and countable product of first-countable (or second-countable) space is first-countable (or second-countable).
\end{theorem}


\begin{definition}[\cindex{Linderl\"of space}]
    A space for which every open covering contains a countable subcovering is called Linderl\"of space.
\end{definition}

\begin{theorem}
    Second-countable space is a Linderl\"of space and a separable space. The converse works if it is metrizable space.
\end{theorem}
\begin{proof}
    Let $\sequence{B_n}$ be the countable basis of $X$.
    
    To prove Linderl\"of space. Let $A$ be the open covering of $X$. For each $x \in X$, there is $x \in a \in A$, and there is a $x \in B_{i_x} \subset a$. The set $\set{B_{i_x}}$ covers $X$ and is countable. For each $B_{i_x}$, choose one $a$ that contains it. These $\set{a}$ is also countable.
    
    To prove separable. Choose $x_n \in B_n$. The set $\set{x_n}$ is countable, and every open set will intersect it.
    
    To prove the reverse of Linderl\"of space in metric space. Cover $X$ with $B(x, \frac{1}{n})$ and there is a countable subcover. Take the union of all these subcover $A$. Now prove it is a basis. For each $x$ and open set $x \in U$. There is $B(x, \epsilon)$ that $x \in B(x,\epsilon) \subset U$. Choose a $a \in A$ with diameter less than $\frac{\epsilon}{2}$ that covers $x$. $a \in B(x, \epsilon)$.
    
    To prove the reverse of separable with metric space. Let $A$ be the dense set. For each $x \in A$, choose $B(x, \frac{1}{n})$ and it is a countable basis. For any $y \in X$ with neighborhood $B(y, \epsilon)$, it intersect with an $z \in A$. Let $d(y,z) = \delta$. Then $B(z, \frac{\epsilon - \delta}{2})$ is contained in $B(y, \epsilon)$.
\end{proof}

\begin{example}
    For space $\realnumber_l$. It satisfies:
    \begin{itemize}
        \item First countable theorem. For each $x$, choose $[x, x+\frac{1}{n})$.
        \item No countable basis. For each $x$ and an open set $[x,x+1)$, there must be a basis $b \in \realnumber_l$ that $b = [x,\epsilon)$. Since $x$ is uncountable, $b$ is uncountable.
        \item Linderl\"of. Assume $R_l$ is covered by $A = \set{[a_\alpha, b_\alpha)}$. Now check what $\set{(a_\alpha, b_\alpha)}$ did not cover. Let $C = \cup \set{(a_\alpha, b_\alpha)}$ and $D = \realnumber - C$. For each $x \in D$, there is $\alpha$ that $x \in [a_\alpha, b_\alpha)$ because it is a covering. Choose a rational number $y>x$ that $y \in [a_\alpha, b_\alpha)$. $y$ is countable, so $D$ is countable. Then because $\realnumber$ is second-countable, use similar method to find countable set from $C$ that covers C. Now take the countable cover of $D$ and $C$.
    \end{itemize}
\end{example}

\begin{theorem}[the product of two Lindel\"of space needs not be Lindel\"of]
    Check the space $\realnumber_l \times \realnumber_l = \realnumber_l^2$. It is called \cindex{Sorgenfrey plane}. The line $y = -x$ is closed in $\realnumber_l^2$. Now construct an open covering. First choose an open space $C = \realnumber_l^2 - \set{y=-x}$. The choose $D = [a,b) \times [-a, d)$. $C \cup D$ is an open covering of $\realnumber_l^2$. But each $d\in D$ only intersect $y = -x$ at one point, therefore uncountable.
\end{theorem}

\begin{theorem}[subspace of Lindel\"of needs not be Lindel\"of]
    The space $I_o^2$ is compact, so it is Lindel\"of. The subset $I \times (0,1)$ is not. Build uncountable subset $\set{x} \times (0,1)$.
\end{theorem}

\begin{theorem}[subspace of separable needs not be separable]
    The set $R^\omega$ is separable. Check the subset $\set{0,1}^\omega$. 
\end{theorem}

\begin{theorem}
    Countable product of separable space is separable.
\end{theorem}
\begin{proof}
    Let $B = \set{x_n} \subset A$ be the countable dense subset of $A$. Choose a base point $x_0 \in A$. For each ${n \in \naturalnumber}$, define a set $C_n = \prod_n B \times \prod_\omega \set{x_0}$. The set $\cup_n C_n$ is separable.
\end{proof}


\begin{theorem}
    Let $f:X \rightarrow Y$ be continuous. If $X$ is Lindel\"of or separable, then $Y$ is also Lindel\"of or separable.
\end{theorem}


\begin{theorem}
    Let $X$ be separable. Then every collection of disjoint subset of $X$ is countable.
\end{theorem}
\begin{proof}
    Let $\set{x_n}$ be the countable dense subset. For the collection of disjoint set $U = \set{U_\delta}$, each $U_\delta$  intersect $\set{x_n}$ on point $x_\delta$. For any $U_\delta$ and $U_\gamma$, their $x_\delta \neq x_\gamma$ because $U_\delta \cap U_\gamma = \emptyset$. Since $\set{x_n}$ is countable, $U$ is countable.
\end{proof}



% separation axiom

\section{Separation Axiom}

\begin{definition}[\cindex{regular}]
    Assume $X$ is $T_1$. $X$ is said to be regular if each pair of point $x$ and a closed set $B$ disjoint from $x$, there exist disjoint open sets containing $x$ and $B$.
\end{definition}

\begin{definition}[\cindex{normal}]
    Assume $X$ is $T_1$. For each pair $A,B$ of disjoint close sets of $X$, there exist disjoint open sets containing $A$ and $B$.
\end{definition}

\begin{theorem}
    Normal space is regular, and regular space is Hausdorff. $T_1$ is required.    
\end{theorem}
\begin{proof}
    \emph{$R_K$ is Hausdorff but not regular}. The set $K$ in $R_K$ is closed. There are two disjoint open set that contains $0$ and $K$. So $T_1$ is required.
\end{proof}


\begin{theorem}
    Let $X$ be $T_1$. Then
    \begin{itemize}
        \item $X$ is regular if and only if for open set $U$ that $x \in U \subset X$, there is neighborhood $x \in V$ that $\closure{V} \subset U$.
        \item $X$ is normal if and only if for a closed set $A$ and open set $U$ that $A \subset U \subset X$, there is an open set $V$ that $A \subset V$ and $\closure{V} \subset U$.
    \end{itemize}    
\end{theorem}
\begin{proof}
    For regular case. Because $X-U$ is closed, there are disjoint $V$ and $W$ that $x \in V$ and $X-U \subset W$. For the reverse, let $B$ be the disjoint close set from $x$. The set $X-B$ is open, so there is $\closure{V} \subset X - B$. The set $V$ and $X - \closure{V}$ are the two disjoint set.
\end{proof}

\begin{theorem}
    A subspace of regular space is regular. A product of regular space is regular.    
\end{theorem}



% normal spaces

\section{Normal Spaces}

\begin{theorem}
    Second-countable regular space is normal.    
\end{theorem}
\begin{proof}
    Let $X$ be the regular space with countable basis $B$. Let $C$ and $D$ be disjoint closed set of $X$. For each point $x \in C$, there is a neighborhood $U_x$ that do not intersect $D$. Being regular, there is neighborhood $x \in V$ that $\closure{V} \subset U$. Choose a basis from $U_x \subset B$ that is contained in $V$. These $U_x$ is countable, so we could order it as $\set{U_n}$. Similarly there is $\set{V_n}$ for $D$. Define $U_n^{'} = U_n - \cup_{i=1}^n \closure{V_i}$ and $V_n^{'} = V_n - \cup_{i=1}^n \closure{U_i}$. The set $U^{'} = \cup U_n^{'}$ and $V^{'} = \cup V_n^{'}$ are disjoint open sets.
\end{proof}

\begin{theorem}
    Every metrizable space is normal.    
\end{theorem}
\begin{proof}
    Let $A,B$ be disjoint closed set of metric space $X$. For each $a \in A$, choose $B(a, \epsilon_a)$ that do not intersect $B$. And do the same for $B$. The set $U = \cup B(a, \frac{\epsilon_a}{2})$ and $V = \cup B(b, \frac{\epsilon_b}{2})$ are the disjoint open set. 
\end{proof}

\begin{theorem}
    Every compact Hausdorff space is normal.    
\end{theorem}
\begin{proof}
    \theoref{disjoint_openset_of_x_and_compact_set} says it is regular. So for each $a \in A$, there is open set $x \in U_a$ and $B \subset V_a$ that are disjoint. Since $A$ is compact, there is finite many set $\set{U_a}$ and $\set{V_a}$. Define $U = \cup \set{U_a}$ and $V = \cap \set{V_a}$.
\end{proof}

\begin{theorem}
    Every order topology is normal.    
\end{theorem}


\begin{theorem}
    If $J$ is uncountable, $\realnumber^J$ is not normal.
\end{theorem}

































































































































































































































































































































































































































































































































































































































































































































































































































































































































































































































































































































































































































































































































































































































































































































































































































































































































































































































































































































































































































































































































































































































































































































































































































































































































































































































































































































































































































































































































































































































































































































































































































































