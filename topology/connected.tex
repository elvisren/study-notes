\chapter{Connected}



Note: 
\begin{enumerate}
    \item Each component of $X$ is closed in $X$ because it contains all limit points
    \item If the number of components is finite, each component is also open
    \item The component needs not be open. ($\mathbb{Q}$)
    \item Path component could be neither open or closed
    \item If $X$ is locally (path) connected, each (path) component of open set is open
    \item If $X$ is locally path connected, its components and path components are the same
\end{enumerate}


\section{Connected Spaces}

\begin{definition}[\cindex{separation}]\label{separation}
    A separation of $X$ is a pair $U,V$ of disjoint nonempty open subsets of $X$ whose union is $X$.
    
    
    Note: separation is different from separable (see \defiref{separable}).
\end{definition}

\begin{definition}[\cindex{connected}]
    A space $X$ is connected if there is no separation for it.
\end{definition}

\begin{theorem}
    A space $X$ is connected if and only if the only open and closed sets are $X$ and $\emptyset$.
\end{theorem}

\begin{theorem}
    A subspace $Y$ of $X$ is connected if a pair $A,B$ of subsets of $Y$ whose union is $Y$, neither of which contains the limit point of the other.
    
    $A$ and $B$ do not have to be open or closed. But we can prove they are both open and closed.
\end{theorem}
\begin{proof}
    Assume $A$ and $B$ did not contain limit point of each other. $\closure{A} \cap B = \emptyset$, so $\closure{A} \cap Y = A$. So $A$ is closed. $B$ is closed as well.
\end{proof}

\begin{theorem}
    If $C,D$ forms a separation of $X$, and if $Y$ is a connected subspace of $X$, then $Y$ lies entirely within either $C$ or $D$.
\end{theorem}

\begin{theorem}
The union of a collection of connected subspace of $X$ that have a point in common is connected.    
\end{theorem}

\begin{theorem}
Let $A$ be a connected subspace of $X$. If $A \subset B \subset \closure{A}$, then $B$ is also connected. So \emph{adding limit points to a connected subspace is connected}.
\end{theorem}
\begin{proof}
    Assume there is a separation of $B$. So $C \cup D = B$. Then $A$ must be in one of them, assume it is $C$. So $\closure{A} \subseteq \closure{C}$, a contradiction.
\end{proof}

\begin{theorem}\label{cont_func_from_cont_to_cont}
The image of a connected space under continuous map is connected. So it means connectedness is a topological property.
\end{theorem}

\begin{theorem}
A finite cartesian product of connected space is connected.
\end{theorem}
\begin{proof}
For $X,Y$, find a line $x \times Y$. This line is homomorphic to $Y$ and connected. Then find a line $X \times y$ which is connected. They share a point $(x,y)$.
\end{proof}

\begin{theorem}
    An arbitrary product of connected space is connected in product topology.    
\end{theorem}


\begin{example}[$\realnumber^\omega$ is not connected in box topology]
    Let $A$ be the set of all bounded sequence, and $B$ all unbounded sequence. $A,B$ are open and they form the separation.
\end{example}

\begin{example}[$\realnumber^\omega$ is connected in product topology]
    Assume $\realnumber$ is connected. Let $\realnumber_n$ be the set of all $(x_1, x_2, ...)$ that $x_{i > n} = 0$. It is homeomorphic to $\realnumber^n$ and connected. They share the same point $(0,0,...)$ so $\bigcup_i \realnumber_i$ is connected. $\forall x \in \realnumber^\omega, x \in U$, prove $U \cap \closure{\bigcup_i \realnumber_i} \neq \emptyset$ in product topology.
\end{example}


\section{$\realnumber$ is Connected}

\begin{definition}[\cindex{linear continuum}]
    A simply ordered set $\mathcal{L}$ having more than one element is called linear continuum if the following hold:
    \begin{enumerate}
        \item $\mathcal{L}$ has the least upper bound property.
        \item $x < y \Rightarrow \exists z, x < z < y$.
    \end{enumerate}
\end{definition}

\begin{theorem}
    If $\mathcal{L}$ is a linearly continuum topology, then $\mathcal{L}$ is connected, so are the convex subset in $\mathcal{L}$. 
    
    So $\realnumber$ is connected.
\end{theorem}
\begin{proof}
    Assume there is a separation $A,B$ of $Y$. Find $a \in A$ and $b \in B$, and check the interval $A_0 = A \cap [a,b]$ and $B_0 = B \cap [a,b]$. Find $c = \supremum{A_0}$. 
    
    If $c \in B_0$. Because $B_0$ is open, there is a set $(d,c] \subseteq B_0$ and $(d,c] \cap A_0 = \emptyset$. By linear continuum, there is $d < e < c$, so $e$ is a smaller upper bound.
    
    If $c \in A_0$, there is open set $(f,c] \subseteq A_0$. By linear continuum, there is a $f < g < c$, a contradiction as well.
\end{proof}


\begin{theorem}[\cindex{intermediate value theorem}]
    Let $f : X \rightarrow Y$ be a continuous map, where $X$ is connected and $Y$ is an ordered set. If $a,b \in X$ and $f(a) < r < f(b)$, $\exists c \in X, f(c) = r$.
\end{theorem}
\begin{proof}
    By \theoref{cont_func_from_cont_to_cont}, the image is connected. Define two open set $f(x) \cap (-\infty, r)$ and $f(x) \cap (r, +\infty)$. They are a separation if there is no $c$ in the image.
\end{proof}

\begin{definition}[\cindex{path}]
    A path in $X$ from $x$ to $y$ is a continuous map $f: [a,b] \rightarrow X$ that $f(a) = x$ and $f(b) = y$. 
\end{definition}

\begin{definition}[\cindex{path connected}]
    A space is said to be path connected if every pair of points of $X$ can be joined by a path in $X$.
\end{definition}

\begin{example}
    The ordered square $I_o^2$ is connected but not path connected. So is it for the \cindex{topologist's sine curve} $\closure{S} = \closure{\set{\displaystyle x \times \sin \left(\frac{1}{x} \right) | 0 < x \leq 1}}$. So if $X$ is path connected, $\closure{X}$ may not be path connected.
    
    So path connected is connected, but connected may not be path connected.
\end{example}


\section{Components}

\begin{definition}[\cindex{component}s]
    Define a equivalent relation $x \sim y$ if there is a connected subspace of $X$ that contains both $x$ and $y$. The equivalent class is called components of $X$.
\end{definition}

\begin{theorem}
The components of $X$ are connected disjoint subspace of $X$ whose union is $X$. Each nonempty connected subspace of $X$ intersects only one of them.    
\end{theorem}

\begin{definition}[\cindex{path component}s]
    Like component, but the connected subspace is now path connected.
\end{definition}


\begin{definition}[\cindex{locally connected}]
    A space is locally connected at $x$ if for all neighborhood $U$ of $x$, there is a connected neighborhood $V$ of $x$ that $V \subset U$. It is called locally connected if it is locally connected at every point.
\end{definition}

\begin{definition}[\cindex{locally path connected}]
    A space is locally path connected at $x$ if for all neighborhood $U$ of $x$, there is a path connected neighborhood $V$ of $x$ that $V \subset U$. It is called locally path connected if it is locally path connected at every point.
\end{definition}

\begin{theorem}
A space $X$ is locally connected if and only if for every open set $U$ of $X$, each component of $U$ is open in $X$.    
\end{theorem}
\begin{proof}
    For any $x$ and open set $U$, find an open component $V$ of $U$ that contains $x$. So for any $x \in X$ and open set $x \in U$, there is connected neighborhood of $x$, so $X$ is locally connected.
    
    If $X$ is connected, for any open set $U$ and $x \in U$ and $x$ is in a component $W$, there is a connected neighborhood $V$ that $x \in V \subseteq U$. Because $V$ is connected, it is in $W$. So for any $x \in W$, there is an open neighborhood of $x$ that is in $W$, so $W$ is open.
\end{proof}

\begin{theorem}
A space $X$ is locally path connected if and only if for every open set $U$ of $X$, each path component of $U$ is open in $X$.    
\end{theorem}

\begin{theorem}
    Each path component of $X$ lies in component of $X$. If $X$ is locally path connected, the components and the path components are the same.    
\end{theorem}
\begin{proof}
    Because $X$ is locally path connected, all its path connected components are open. For a component $U$ of $X$, choose one of its path component $P$. So for all remaining path components of $U$, their union $Q$ is open, so $P$ and $Q$ form a partition of $U$.
\end{proof}
















