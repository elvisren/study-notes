\chapter{Connected}



Note: 
\begin{enumerate}
    \item The closure of connected subspace is connected
    \item Each component of $X$ is closed in $X$ because it contains all limit points
    \item If the number of components is finite, each component is also open
    \item Path component could be neither open or closed
    \item If $X$ is locally (path) connected, each (path) component of open set is open
    \item If $X$ is locally path connected, its components and path components are the same
    \item Prove a space is connected: show it is a closure of a connected space.
\end{enumerate}


\section{Connected Spaces}

\begin{definition}[\cindex{separation}]\label{separation}
    A separation of $X$ is a pair $U,V$ of disjoint nonempty open subsets of $X$ whose union is $X$.
    
    
    Note: separation is different from separable (see \defiref{separable}).
\end{definition}

\begin{definition}[\cindex{connected}]
    A space $X$ is connected if there is no separation for it.
\end{definition}

\begin{theorem}
    A space $X$ is connected if and only if the only open and closed sets are $X$ and $\emptyset$.
\end{theorem}

\begin{theorem}
    A subspace $Y$ of $X$ is connected if there is no pair $A,B$ of disjoint subsets of $Y$ whose union is $Y$, neither of which contains the limit point of the other.
    
    $A$ and $B$ do not have to be open or closed. But we can prove they are both open and closed.
\end{theorem}
\begin{proof}
    Assume $A$ and $B$ did not contain limit point of each other. $\closure{A} \cap B = \emptyset$, so $\closure{A} \cap Y = A$. So $A$ is closed. $B$ is closed as well.
\end{proof}

\begin{theorem}
    If $C,D$ forms a separation of $X$, and if $Y$ is a connected subspace of $X$, then $Y$ lies entirely within either $C$ or $D$.
\end{theorem}

\begin{theorem}
The union of a collection of connected subspace of $X$ that have a point in common is connected.    
\end{theorem}

\begin{theorem}\label{closure_is_connected_too}
    Let $A$ be a connected subspace of $X$. If $A \subset B \subset \closure{A}$, then $B$ is also connected. So \emph{adding limit points to a connected subspace is connected}.
\end{theorem}
\begin{proof}
    Assume there is a separation of $B$. So $C \cup D = B$. Then $A$ must be in one of them, assume it is $C$. So $\closure{A} \subseteq \closure{C}$ and $\closure{C} \cap D = \emptyset$, a contradiction.
\end{proof}

\begin{theorem}\label{cont_func_from_cont_to_cont}
The image of a connected space under continuous map is connected. So it means connectedness is a topological property.
\end{theorem}

\begin{theorem}
A finite cartesian product of connected space is connected.
\end{theorem}
\begin{proof}
For $X,Y$, find a line $x \times Y$. This line is homomorphic to $Y$ and connected. Then find a line $X \times y$ which is connected. They share a point $(x,y)$.
\end{proof}

\begin{theorem}
    An arbitrary product of connected space is connected in product topology.    
\end{theorem}

\begin{example}
    If $A$ is connected, $\interior{A}$ may not be connected, and $\boundary{A}$ is connected.
\end{example}
\begin{proof}
    Assume $C+D = \boundary{A}$, then $C+ \interior{A}$ and $D$ is a separation of $\closure{A}$.
\end{proof}


\begin{example}[$\realnumber^\omega$ is not connected in box topology]
    Let $A$ be the set of all bounded sequence, and $B$ all unbounded sequence. $A,B$ are open in box topology. For any $x$, if it is abounded, the set of $\set{x_i -1, x_i + 1}$ is bounded. So they form the separation.
\end{example}

\begin{example}[$\realnumber^\omega$ is connected in product topology]
    Assume $\realnumber$ is connected. Let $\realnumber_n$ be the set of all $(x_1, x_2, ...)$ that $x_{i > n} = 0$. It is homeomorphic to $\realnumber^n$ and connected. They share the same point $(0,0,...)$ so $\realnumber^\infty = \bigcup_i \realnumber_i$ is connected. Now prove $\closure{\bigcup_i \realnumber_i} = \realnumber^\omega$. For any $x \in \realnumber^\omega$, choose a basis $U$ around $x$. There is $i$ that $U_i = \realnumber$, so $U \subseteq R_i$. According to \theoref{closure_is_connected_too}, $R^\omega = \closure{\cup R_i}$ is connected.
\end{example}


\section{$\realnumber$ is Connected}

\begin{definition}[\cindex{linear continuum}]
    A simply ordered set $\mathcal{L}$ having more than one element is called linear continuum if the following hold:
    \begin{enumerate}
        \item $\mathcal{L}$ has the least upper bound property.
        \item $x < y \Rightarrow \exists z, x < z < y$.
    \end{enumerate}
\end{definition}

\begin{theorem}
    If $\mathcal{L}$ is a linearly continuum topology $\Leftrightarrow$ $\mathcal{L}$ is connected. 
    
    So are the convex subset in $\mathcal{L}$. So $\realnumber$ is connected.
\end{theorem}
\begin{proof}
    Assume the linearly continuum topology and there is a separation $A,B$ of $Y$. Find $a \in A$ and $b \in B$, and check the interval $A_0 = A \cap [a,b]$ and $B_0 = B \cap [a,b]$. Find $c = \supremum{A_0}$. 
    
    If $c \in B_0$. Because $B_0$ is open, there is a set $(d,c] \subseteq B_0$ and $(d,c] \cap A_0 = \emptyset$. By linear continuum, there is $d < e < c$, so $e$ is a smaller upper bound.
    
    If $c \in A_0$, there is open set $(f,c] \subseteq A_0$. By linear continuum, there is a $f < g < c$, a contradiction as well.
    
    Now assume it is connected. If there is a set $A$ which is bounded above without maximum value, then $X - A$ is open, which form a separation of $\mathcal{L}$. If there is $a < b$ that there is no $c$ that $a < c < b$. Then the set $(-\infty, b)$ and $(a, \infty)$ form a separation.
\end{proof}


\begin{theorem}[\cindex{intermediate value theorem}]
    Let $f : X \rightarrow Y$ be a continuous map, where $X$ is connected and $Y$ is an ordered set. If $a,b \in X$ and $f(a) < r < f(b)$, $\exists c \in X, f(c) = r$.
\end{theorem}
\begin{proof}
    By \theoref{cont_func_from_cont_to_cont}, the image is connected. Define two open set $f(x) \cap (-\infty, r)$ and $f(x) \cap (r, +\infty)$. They are a separation if there is no $c$ in the image.
\end{proof}


% path connected

\section{Path Connected}

\begin{definition}[\cindex{path}]
    A path in $X$ from $x$ to $y$ is a continuous map $f: [a,b] \rightarrow X$ that $f(a) = x$ and $f(b) = y$. 
\end{definition}

\begin{definition}[\cindex{path connected}]
    A space is said to be path connected if every pair of points of $X$ can be joined by a path in $X$.
\end{definition}


\begin{example}[connected does not mean path connected]
    $I_o^2$ is connected but not path connected. It is linearly continuum, so it is connected. Assume it is path connected. Assume there is a continuous function $f$ from $[0,1]$ to $(0,0)$ and $(1,1)$. Since $f$ will pass all points of $I_o^2$ by the intermediate value theorem, for each $x \in (0,1)$, the inverse image $\inverse{f}(\set{x} \times (0,1))$ is an open set $U_x$ in $[0,1]$. $U_x$ will contain a point in $\rational$. Since $\rational$ is countable, but $\set{x} \times (0,1)$ is uncountable, a contradiction.
\end{example}

\begin{example}[closure of path connected may not be path connected]
    The topologist's sine curve (\defiref{topologist_sine_curve}) is connected but not path connected. So if $X$ is path connected, $\closure{X}$ may not be path connected.
    
    So path connected is connected, but connected may not be path connected.
\end{example}





\section{Components}

\begin{definition}[\cindex{component}]
    Define a equivalent relation $x \sim y$ if there is a connected subspace of $X$ that contains both $x$ and $y$. The equivalent class is called components of $X$.
\end{definition}

\begin{theorem}
\emph{The components of $X$ are connected disjoint closed subspace of $X$} whose union is $X$. Each nonempty connected subspace of $X$ intersects only one of them.    
\end{theorem}


\begin{definition}[\cindex{path component}]
    Like component, but the connected subspace is now path connected.
\end{definition}

\begin{example}
    For topologist's sine curve, it has one component and two path components. The vertical line is a path component that is closed but not open. and the sine curve is open but not closed. If we delete all rational points from vertical line, the result is one component with uncountably many path components.    
\end{example}



\begin{definition}[\cindex{locally connected}]
    A space is locally connected at $x$ if for all neighborhood $U$ of $x$, there is a connected neighborhood $V$ of $x$ that $V \subset U$. It is called locally connected if it is locally connected at every point.
\end{definition}

\begin{definition}[\cindex{locally path connected}]
    A space is locally path connected at $x$ if for all neighborhood $U$ of $x$, there is a path connected neighborhood $V$ of $x$ that $V \subset U$. It is called locally path connected if it is locally path connected at every point.
\end{definition}

\begin{theorem}
A space $X$ is locally connected if and only if for \emph{every open set} $U$ of $X$, each component of $U$ is open in $X$.    
\end{theorem}
\begin{proof}
    For any $x$ and open set $U$, find an open component $V$ of $U$ that contains $x$. So for any $x \in X$ and open set $x \in U$, there is connected neighborhood of $x$, so $X$ is locally connected.
    
    If $X$ is connected, for any open set $U$ and $x \in U$ and $x$ is in a component $W$, there is a connected neighborhood $V$ that $x \in V \subseteq U$. Because $V$ is connected, it is in $W$. So for any $x \in W$, there is an open neighborhood of $x$ that is in $W$, so $W$ is open.
\end{proof}

\begin{theorem}
A space $X$ is locally path connected if and only if for every open set $U$ of $X$, each path component of $U$ is open in $X$.    
\end{theorem}

\begin{theorem}
    Each path component of $X$ lies in component of $X$. If $X$ is locally path connected, the components and the path components are the same.    
\end{theorem}
\begin{proof}
    Because $X$ is locally path connected, all its path connected components are open. For a component $U$ of $X$, choose one of its path component $P$. So for all remaining path components of $U$, their union $Q$ is open, so $P$ and $Q$ form a partition of $U$.
\end{proof}
















