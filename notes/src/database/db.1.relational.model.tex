\section{Relational Model}

Relational model is defined by Ted Codd in 1970 \cite{Codd1983a}.

\subsection{Basics of the Relational Model}

\begin{definition}
    A \cindex{relation} is a two-dimensional table.
\end{definition}

\begin{definition}
    The columns of a relation are named by \cindex{attributes}.
\end{definition}

\begin{definition}
    The name of a relation and the set of attributes for a relation is called the \cindex{schema} of that relation.
\end{definition}

\begin{definition}
    The rows of a relation are called \cindex{tuples}. A tuple has one component for each relation attribute.
\end{definition}

The relation is a set of tuples, not lists of tuples.

\begin{definition}
    The attribute value type is called \cindex{domain}.
\end{definition}

\begin{definition}
    A set of tuples for a given relation is called \cindex{instance}, or \cindex{current instance}.
\end{definition}

\begin{definition}
    A database that keeps all historical version of data is called \cindex{temporal database}.
\end{definition}

\begin{definition}
    A set of attributes form a \cindex{key} for a relation if we do not allow two tuples with the same values in all the attributes of the key. We usually use underscore to indicate a key, for example $R(\underline{a},\underline{b},c,d)$
\end{definition}



% for sql

\subsection{SQL}

SQL(\cindex{the structured query language}) have two responsibilities:
\begin{enumerate}
    \item The data definition sublanguage for declaring database schema.
    \item The data manipulation sublanguage for querying and modifying the database.
\end{enumerate}


There are three relations in SQL:

\begin{enumerate}
    \item table, which stores the relation.
    \item \cindex{view}, which is a relation defined by a computation. It is not stored but constructed.
    \item \cindex{temporary table}, which is constructed by the SQL language processor. It is not stored.
\end{enumerate}

There are two kinds of keys:
\begin{enumerate}
    \item Primary key, which cannot contain NULL value.
    \item Unique key, which could contain NULL value.
\end{enumerate}


% relation algebra

\subsection{Relational Algebra}

The operations of the traditional relation algebra falls into 4 classes:
\begin{enumerate}
    \item Set operation. $R \cup S$, $R \cap S$, $R - S$.
    \item Operations that remove rows or columns of a relation. 
        \begin{enumerate}
            \item \cindex{projection}: $\pi_{A_1, A_2, \dots, A_n} (R)$ which takes only columns $A_1, A_2, \dots, A_n$ of $R$.
            \item \cindex{selection}: $\sigma_C (R)$, $C$ is a conditional expression. Only those tuples that satisfies $C$ will be kept.
        \end{enumerate}
    \item Operations that combine tuples of two relations.
        \begin{enumerate}
            \item \cindex{cartesian product}: $R \times S$.
            \item \cindex{natural join}: $R \Join S$, a join operation followed by a selection that if two columns have the same name, their values should be the same.
            \item \cindex{theta join}: $R \Join_C S$, a cartesian join followed by a selection $C$.
        \end{enumerate}
    \item Rename columns of a relation. $\rho_{S(A_1, A_2, \dots, A_n)} (R)$
\end{enumerate}
























