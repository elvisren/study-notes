\section{Examples}


\begin{example}[\cindex{Condorcet Paradox}]
    There are players 1, 2 and 3 who vote on $a$, $b$ and $c$. Their preferences are:
    \begin{enumerate}
        \item Player 1 : $a \succ b \succ c$
        \item Player 2 : $c \succ b \succ a$
        \item Player 3 : $b \succ a \succ c$
    \end{enumerate}
    The choice is made by majority vote:
    \begin{enumerate}
        \item For $a$ and $c$, player 1 and 3 vote for $a$
        \item For $b$ and $c$, player 1 and 2 vote for $c$
        \item For $a$ and $b$, player 2 and 3 vote for $b$
    \end{enumerate}
    So $a \succ b \succ c \succ a$, which is a paradox.
\end{example}

\begin{example}[\cindex{the prisoner's dilemma}]
    Two suspects at the police station and questioning each in a different room. Each suspect is offered a deal and he will either confess, or flinks, or say nothing and remain mum.
    \begin{description}
        \item [Players] $N = \set{1,2}$.
        \item [Strategy] $S_i = \set{M, F}$. $F$ is flinks,, $M$ is to remain mum.
        \item [Payoff] Let $v_i(s_1, s_2)$ be the payoff to player $i$. \begin{itemize}
            \item $v_1 (M,M) = v_2(M,M) = -2$
            \item $v_1 (F,F) = v_2(F,F) = -4$
            \item $v_1 (M,F) = v_2(M,F) = -5$
            \item $v_1 (F,M) = v_2(F,M) = -1$
        \end{itemize}
    \end{description}
\end{example}

\begin{example}[\cindex{cournot duopoly}]
    Introduced by Augustin Cournot (1838). Two identical firms produce some goods. They will choose the quantity of production, which will determine the price and profit.
\end{example}

matching pennies: has no Nash equilibrium

