\chapter{Examples}



\begin{example}
    Suppose $m$ coupons are selected from $n$ different types of coupons. Each coupon is equally likely to be selected. What is the expected number of different types?
\end{example}

\begin{proof}
    Let $X$ denote the number of types in $m$ selection. We have $X = X_1 + ... + X_n$ where
    \begin{equation*}
        X_i = \begin{cases}
            1 & \text{ , if } i \text{ occurs in selection} \\
            0 & \text{ , otherwise}
        \end{cases}
    \end{equation*}
    We have 
    \begin{equation*}
        \expect{X_i} = 1 - \left(\frac{n - 1}{n} \right)^m
    \end{equation*}
    So $\expect{X} = \expect{X_1} + \dots + \expect{X_n} = \displaystyle n \left(1 - \left(\frac{n - 1}{n} \right)^m \right)$
\end{proof}

\begin{example}
    Let $X_i$ be independent and identically distributed continuous random variable. Let $X_{(i)}$ denote the $i$th smallest of these random variables. Then $X_{(i)}$ is called \cindex{order statistics}. Note that $X_{(i)} \leq x$ if and only if at least $i$ of $X_j$ are not larger than $x$, so
    \begin{equation*}
        \probability{X_{(i)} \leq x} = \sum_{k=i}^n \binom{n}{k} \left(F(x) \right)^k \left(1 - F(x) \right)^{n-k}
    \end{equation*}
    
    So
    \begin{equation*}
        \begin{aligned}
            f_{X_{(i)}} (x) &= \dod{\probability{X_{(i)} \leq x}}{x} \\
            &= \frac{n!}{(n-i)!(i-1)!} f(x) \left( F(x) \right)^{i-1} \left(1-F(x) \right)^{n-i}
        \end{aligned}
    \end{equation*}
    The result could be read as there are $\left( F(x) \right)^{i-1}$ which are less than $x$, $\left(1-F(x) \right)^{n-i}$ which are larger than $x$, and $f(x)$ which equals $x$.
\end{example}

\begin{example}
A particle moves along a circle of $m+1$ nodes. Each time it has equality of moving clockwise or counterclockwise. What is the probability that node $i$ is the last visited node?    
\end{example}
\begin{proof}
    Suppose $i$ is the last visited nodes and consider 2 nodes $i-1$ and $i+1$ around it. If $i-1$ is visited before $i+1$, the probability is the same as gambling without losing money when it reaches $i+1$, so the probability is the same for every nodes. So the result is $\displaystyle \frac{1}{m}$.
\end{proof}



% example
\section{Example}

\begin{example}
    A miner is trapped in a mine containing 3 doors. The first door needs to a tunnel that takes him to safety after 2 hours. The second door take him back after 3 hours and the third door takes him back after 5 hours. What is the expected length of time until the miner reaches out when he chooses the door equally?
\end{example}

\begin{example}
    $n$ men through their hat into the room and randomly select one. They will leave the game if they have selected their own hat.
    \begin{enumerate}
        \item \label{hatq1} what is the probability of no match in first round?
        \item \label{hatq2} what is the probability of k match in first round?
        \item \label{hatq3} what is the expected match in first round?
        \item \label{hatq4} what is the variance of match in first round?
        \item \label{hatq5} what is the expected number of rounds?
        \item \label{hatq6} what is the variance of rounds?
        \item \label{hatq7} what is the expected total number of selection?
        \item \label{hatq8} what is the expected number of wrong selection for each man?
    \end{enumerate}    
\end{example}
\begin{proof}
    define these variables:
    \begin{itemize}
        \item $R_n$: the number of rounds necessary for $n$ men.
        \item $S_n$: the total number of all selection.
        \item $C_i$: the number of all wrong selection by $i$th man.
        \item $X_n$: the number of matches in the first round.
    \end{itemize}
    
    For question (\ref{hatq1}) let $E_n$ be the event that no match occurs in $n$ men scenario, $M$ be the event that the first man select its own hat. We have
    \begin{equation*}
        \probability{E_n} = \probability{E_n|M} \probability{M} + \probability{E_n|M^c} \probability{M^c}
    \end{equation*}

    Because $\probability{E|M} = 0$, $\probability{E_n} = \dfrac{n-1}{n} \probability{E_n|M^c} $.
    
    In $M^c$ case, the first man selected the hat of another person. there are two possibilities on whether the other person select the hat of the first man:
    \begin{enumerate}
        \item the "another" man did not select the hat of first man. In this case, $\probability{E_{n}|?} = \probability{E_{n-1}}$.
        \item the "another" man did select the hat of the first man. In this case, $\probability{E_{n}|?} = \probability{E_{n-2}} \probability{?} = \dfrac{1}{n-1} \probability{E_{n-2}}$.
    \end{enumerate}
    
    So 
    \begin{equation*}
        \probability{E_n} = \probability{E_{n-1}} + \frac{1}{n-1} \probability{E_{n-2}}
    \end{equation*}
    
    and $\probability{E_n} = \displaystyle \sum_{i=2}^n \frac{(-1)^i}{n!}$
    
    
    For question (\ref{hatq2}), for any fixed group of $k$ men that select their own hat, the probability is 
    \begin{equation*}
        \frac{1}{n} \frac{1}{n-1} \cdots \frac{1}{n-(k-1)} \probability{E_{n-k}} = \frac{(n-k)!}{n!} \probability{E_{n-k}}
    \end{equation*}
    
    Because there are $\binom{n}{k}$ choices of $k$ men, the result is
    \begin{equation*}
        \frac{(n-k)!}{n!} \probability{E_{n-k}} \binom{n}{k} = \frac{1}{k!} \sum_{i=2}^{n-k} (-1)^{n-k} \frac{1}{i!}
    \end{equation*}
    
    Another way is to calculate the length $C$ of the cycle that contains the first man. we have 
    
    \begin{equation*}
    \begin{aligned}
        \probability{E_n} &= \sum_{k=1}^n \probability{E_n|C=k} \probability{C=k} \\
        &= \sum_{k=1}^n \probability{E_{n-k}} \probability{C=k}
    \end{aligned}
    \end{equation*}
    
    We have 
    \begin{equation*}
        \probability{C=k} = \frac{n-1}{n} \frac{n-2}{n-1} \cdots \frac{n-k+1}{c-k+2} \frac{1}{n-k+1} = \frac{1}{n}
    \end{equation*}
    
    So $\displaystyle \probability{E_n} = \frac{1}{n} \sum_{k=2}^n \probability{E_{n-k}}$.
    
    
    
    For equestion question (\ref{hatq3}), the probability $i$th man select its hat is $\probability{H_i} = \dfrac{1}{n}$, so 
    \begin{equation*}
        \begin{aligned}
            \expect{\sum_i H_i} &= \sum_i \expect{H_i} \\
            &= \sum_i \frac{1}{n} \\
            &= 1
        \end{aligned}
    \end{equation*}
    
    For question (\ref{hatq4}), the $\variance{ \sum_i H_i} = 1$
    
    For question (\ref{hatq5}), 
    \begin{equation*}
    \begin{aligned}
        \expect{R_n} &= \sum_{i=0}^n \expect{R_n | X_n = i} \probability{X_n = i} \\
        &= \sum_{i=0}^n (1 + \expect{R_{n - i}}) \probability{X_n = i} \\
        &= 1 + \expect{R_n} \probability{X_n = 0} + \sum_{i=0}^n \expect{R_{n-i}} \probability{X_n=i} \\
        &= \expect{R_n} \probability{X_n = 0} + n(1 - \probability{X_n = 0} )
    \end{aligned}
    \end{equation*}
    
    So the only solution is $\expect{R_n} = n$.
    
    For question (\ref{hatq6}), for the variance, we have
    \begin{equation*}
        \expect{R_n|X} = 1 + \expect{R_{n-X}} = 1 + n - X
    \end{equation*}
    
    Also we have $\variance{R_n|X} = \variance{R_{n-X}}$. So we have
    \begin{equation*}
        \begin{aligned}
            \variance{R_n} &= \expect{\variance{R_n|X}} + \variance{\expect{R_n|X}} \\
            &= \expect{\variance{R_{n-X}}} + \variance{X} \\
            &= \sum_{i=0}^n \variance{R_{n-j}} \probability{X=j} + \variance{X} \\
            &= \variance{R_n}\probability{X = 0} + \sum_{j=1}^n \variance{R_{n-j}} \probability{X=j} + \variance{X}
        \end{aligned}
    \end{equation*}
    
    The solution is $\variance{R_n} = n$.
    
    
    For question (\ref{hatq7}), for $S_n$, we have
    \begin{equation*}
        \expect{S_n} = n + \expect{S_{n - X_n}}
    \end{equation*}
    And the solution is $\expect{S_n} = n + \dfrac{n^2}{2}$.
    
    For question (\ref{hatq8}),
    \begin{equation*}
        \begin{aligned}
            \sum_{j=1}^n (C_j + 1) & = S_n \\
            \expect{C_j} &= \expect{C_j + 1} -1 \\
            &= \dfrac{\expect{S_n}}{n}  -1\\
            &= \left( 1 + \frac{n}{2} \right) - 1 \\
            &= \frac{n}{2}
        \end{aligned}
    \end{equation*}
\end{proof}

\begin{example}
what is the number of necessary trial to get $k$ consecutive success? Each trial has probability $p$ of being successful.
\end{example}
\begin{proof}
    Let $N_k$ the number of $k$ consecutive success. $A_{k-1, k}$ be the additional trial from $k-1$ success to $k$ success, so
    \begin{equation*}
        \expect{N_k} = \expect{N_{k-1}} + \expect{A_{k-1, k}}
    \end{equation*}
    
    We have 
    \begin{equation*}
        \expect{A_{k-1,k}} = 1 \times p + (1 + \expect{N_k}) (1-p) = 1 + (1-p) \expect{N_k}
    \end{equation*}
    So
    \begin{equation*}
        \expect{N_k} = \frac{1}{p} + \frac{\expect{N_{k-1}}}{p}
    \end{equation*}
    with 
    \begin{equation*}
        \expect{N_1} = \frac{1}{p}
    \end{equation*}
    So
    \begin{equation*}
        \expect{N_k} = \sum_{i=1}^k \frac{1}{p^i}
    \end{equation*}
\end{proof}


\begin{example}
    Analyse the quick sort algorithm    
\end{example}
\begin{proof}
    Suppose there are $n$ numbers and all permutation are of equally likely. Let $M_n$ be the expected number of comparison needed for $n$ numbers. So
    \begin{equation*}
    \begin{aligned}
        M_n &= \sum_{j=1}^n \expect{\text{number of comparison} | \text{pivot is the } i \text{th smallest}} \frac{1}{n} \\
        &= \sum_{j=1}^n (n - 1 + M_{j-1} + M_{n - j}) \frac{1}{n} \\
        &= n - 1 + \frac{2}{n} \sum_{k=1}^{n-1} M_k
    \end{aligned}
    \end{equation*}
    
    So
    \begin{equation*}
        \begin{aligned}
            M_{n+1} &= 2(n+2) \sum_{i=1}^n \frac{i}{(i+1)(i+2)} \\
            &\approx 2(n+2) \log{(n+2)}
        \end{aligned}
    \end{equation*}
\end{proof}


\begin{example}[compound random variable]
    Let $X_i$ be identical independent random variable with mean $\mu$ and variance $\sigma^2$. Let $N$ be random variable. The random variable $\displaystyle \sum_{i=1}^N X_i$ is called \cindex{compound random variable}. Its expectation is $\displaystyle \expect{\sum_{i=1}^N X_i} = \expect{N} \expect{X}$, and the variance is $\variance{\sum_{i=1}^N X_i} = \sigma^2 \expect{N} + \mu^2 \variance{N}$
\end{example}
\begin{proof}
    Let $\displaystyle S = \sum_{i=1}^N X_i $  be the random variable. 
    \begin{equation*}
        \begin{aligned}
            \displaystyle \expect{S| N = n} &= \expect{\sum_{i=1}^n X_i | N = n} \\
            &= \expect{\sum_{i=1}^n X_i} \\
            &= n \expect{X}
        \end{aligned}
    \end{equation*}
    Thus
    \begin{equation*}
        \expect{S} = \expect{N \expect{X}} = \expect{N} \expect{X}
    \end{equation*}
    
    \begin{equation*}
        \begin{aligned}
            \displaystyle \variance{S|N=n} &= \variance{\sum_{i=1}^n X_i | N = n} \\
            &= \variance{\sum_{i=1}^n X_i} \\
            &= n \variance{X}
        \end{aligned}
    \end{equation*}
    So $\variance{S|N} = N \variance{X}$, $\expect{S|N} = N \expect{X}$, and
    \begin{equation*}
        \begin{aligned}
            \variance{S} &= \expect{\variance{S|N}} + \variance{\expect{S|N}} \\
            &= \expect{N} \variance{S} + \variance{N}\left(\expect{S}\right)^2
        \end{aligned}
    \end{equation*}
\end{proof}

\begin{example}
    For $n$ distinct value, select the biggest one using the following rules:
    \begin{enumerate}
        \item determine the maximum of first $k$ value. reject all of them.
        \item select the first one that is larger than the maximum value just found.
    \end{enumerate}
    What is the probability that the rule select the maximum among all $n$ values?
\end{example}
\begin{proof}
    Let $X$ be the position of largest value and $P_k(\text{best})$ be the probability that the best value is elected using the rule. We have
    \begin{equation*}
        \begin{aligned}
            P_k (\text{best}) &= \sum_{i=1}^n P_k (\text{best} | X = i) p(X=i) \\
            &= \frac{1}{n} \sum_{i=1}^n P_k (\text{best}|X=i)
        \end{aligned}
    \end{equation*}
    Because the largest will be selected if the largest of first $k$ is also the largest of first $i-1$, we have
    \begin{equation*}
        P_k(\text{best} | X=i) = \frac{k}{i-1}
    \end{equation*}
    So
    \begin{equation*}
        \begin{aligned}
            P_k(best) &= \frac{1}{n} \sum_{i=1}^n P_k (\text{best}|X=i) \\
            &= \frac{k}{n} \sum_{i=k+1}^n \frac{1}{k-1} \\
            &\approx \frac{k}{n} \int_k^{n-1} \frac{1}{x} \dif{x} \\
            &= \frac{k}{n} \log{\frac{n-1}{k}}
        \end{aligned}
    \end{equation*}
    The best $k$ is $\dfrac{1}{e} n$.
\end{proof}

\begin{example}[The Ballot Problem]
    In an election candidate A received $n$ notes and B received $m$ notes where $n > m$. What is the probability that A is always ahead of B?
\end{example}
\begin{proof}
    Let $P_{n,m}$ denote the probability. By conditioning on who receive the last vote, we have:
    \begin{equation*}
        P_{n,m} = \frac{n}{n+m} P_{n-1,m} + \frac{m}{n+m}P_{n,m-1}
    \end{equation*}    
    The solution is $P_{n,m} = \dfrac{n-m}{n+m}$
\end{proof}


\begin{example}
    A coin has $p$ of being head. What is the probability that the total number of head equals tail after $2n$ flip?    
\end{example}
\begin{proof}
    \begin{equation*}
        \begin{aligned}
            \probability{\text{first time } = 2n} &= \probability{\text{first time } = 2n | n \text{ heads in first } 2n} \binom{2n}{n} p^n (1-p)^n \\
            &= P_{n,n-1} \binom{2n}{n} p^n (1-p)^n \\
            &=  \frac{\displaystyle \binom{2n}{n} p^n (1-p)^n}{2n - 1}
        \end{aligned}
    \end{equation*}
\end{proof}

\begin{example}
    What is the probability that the first time there are $i$ more heads than tails occurs after the $2n+i$   flip?
\end{example}
\begin{proof}
    it occurs $\iff$ starting from the final flip and working backwards, the head is always in the lead.
\end{proof}





