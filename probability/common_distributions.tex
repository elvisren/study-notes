\chapter{Common Distributions}




\section{Summary}


\begin{center} 
   \begin{tabular}{ccccc}
  \hline
  name & density & $\phi(t)$ & mean & var \\

  \cindex{binomial} & $\displaystyle \binom{n}{x} p^x (1-p)^{n-x}$  &$(p e^t + q)^n$ & np & npq \\

  \cindex{poisson} & $\displaystyle e^{-\lambda} \frac{\lambda^x}{x!}$  & $\displaystyle e^{\lambda (e^t -1)}$& $\lambda$ & $\lambda$ \\

  \cindex{geometric} & $p (1-p)^{x-1}$& $\displaystyle \frac{pe^t}{1-(1-p)e^t}$ & $\displaystyle \frac{1}{p}$ & $\displaystyle \frac{1-p}{p^2}$ \\

  \cindex{uniform} &$\displaystyle \frac{1}{b-a}$ & $\displaystyle \frac{e^{tb} - e^{ta}}{t(b-a)}$ & $\displaystyle \frac{a + b}{2}$ & $\displaystyle \frac{(b-a)^2}{12}$ \\

  \cindex{normal} & $\displaystyle \frac{1}{\sqrt{2 \pi} \sigma} \exponential{- \frac{(x - \mu)^2}{2 \sigma^2}}$ & $\exponential{\displaystyle \mu t + \frac{\sigma^2 t^2}{2}}$&$\mu$& $\sigma^2$ \\
  
  \cindex{exponential} & $\lambda e^{-\lambda x}$ & $\displaystyle \frac{\lambda}{\lambda - t}$ & $\displaystyle \frac{1}{\lambda}$ & $\displaystyle \frac{1}{\lambda^2}$ \\

  \cindex{gamma} & $\displaystyle \frac{\beta e^{-\beta x} (\beta x)^{n - 1}}{\Gamma(\alpha)}$ & $\displaystyle \left(\frac{\beta }{\beta - t} \right)^\alpha$ & $\displaystyle \frac{\alpha}{\beta}$ & $\displaystyle \frac{\alpha}{\beta^2}$ \\
  
  
  \cindex{beta} & $\displaystyle \frac{\Gamma(\alpha + \beta)}{ \Gamma(\alpha) \Gamma(\beta) } x^{\alpha -1} (1-x)^{\beta - 1}$ & & $\displaystyle \frac{\alpha}{\alpha + \beta}$ & $\displaystyle \frac{\alpha\beta}{(\alpha+\beta)^2 (\alpha+\beta+1)}$ \\
  
  \cindex{chi-square} & $\displaystyle \frac{x^{\frac{k}{2} - 1}e^{-\frac{x}{2}}}{2^{\frac{k}{2}} \Gamma(\frac{k}{2})}$ & $\displaystyle (1-2s)^{- \frac{k}{2}}$ & k & 2k \\

  \hline
\end{tabular} 
\end{center}



% poisson
\section{Poisson Distribution}

Poisson distribution is an approximate to binomial distribution. Set $\lambda = n p $ for large $n$ and small $p$.


% geometric
\section{Geometric Distribution}

Geometric distribution is memoryless, that is 
\begin{equation}
    \probability{X > s | x > t} = \probability{X > s - t}
\end{equation}


% normal
\section{Normal Distribution}

\begin{theorem}
    If $X$ is normally distributed $\normaldistribution{\mu}{\sigma^2}$, then $Y = aX + b$ is a normal distribution with $a\mu + b$ and $(a\sigma)^2$. So $Y =\displaystyle \frac{X - \mu}{\sigma}$ is normally distributed $\normaldistribution{0}{1}$, which is called \cindex{standard normal distribution}.
\end{theorem}



% gamma
\section{Gamma Distribution}

\begin{definition}[gamma function]
    \cindex{gamma function} is an extension of $n!$ to $\realnumber$ which is defined as
    \begin{equation}
        \Gamma(\alpha) = \int_0^\infty e^{-x} x^{\alpha - 1} \dif{x}
    \end{equation}

    Gamma function has the following property:
    \begin{equation}
        \Gamma \left(\frac{1}{2} \right) = \sqrt{\pi}
    \end{equation}
    
    \begin{equation}
        \Gamma (a + 1) = a \Gamma(a)
    \end{equation}
    \begin{equation}
        \Gamma (n) = (n-1)!
    \end{equation}
\end{definition}


\begin{definition}[gamma distribution]
    A random variable $X \in [0,\infty$ is a \cindex{gamma distribution} $\gammadistribution{\alpha}{\beta})$ if the pdf is:
    \begin{equation}
        f(x|\alpha, \beta) = \frac{\beta e^{-\beta x} (\beta x)^{\alpha - 1}}{\Gamma(\alpha)}
    \end{equation}   
    
    Here $\alpha$ is the shape parameter which controls the height of the distribution. $\beta$ is the sale parameter which control the spread.
\end{definition}

\begin{example}
    Let $x_i$ be $n$ independent random variables with exponential distribution $\lambda$, then
    \begin{equation}
        \sum_{i=1}^n x_i \sim \gammadistribution{n}{\lambda}
    \end{equation}
    
    exponential distribution is gamma distribution with $\alpha = 1$.
\end{example}

\begin{theorem}
    If $X_i$ are independent gamma distribution $\gammadistribution{\alpha_i}{\beta}$, then the sum $\displaystyle \sum_{i=1}^n X_i$ is a gamma distribution $\gammadistribution{\sum \alpha_i}{\beta}$.
\end{theorem}


If we set $\alpha = \frac{k}{2}$ and $\beta = \frac{1}{2}$, the result is $\chisquaredistribution{k}$. If we set $\alpha = 1$, the result is exponential distribution with $\beta$.


% beta
\section{Beta Distribution}

\begin{definition}[beta distribution]
    $X \in (0,1)$ is a \cindex{beta distribution} $\betadistribution{\alpha}{\beta}$ if the pdf is:
    \begin{equation}
        f(x|\alpha, \beta) = \frac{\Gamma(\alpha + \beta)}{\Gamma(\alpha) \Gamma(\beta)} x^{\alpha - 1} (1-x)^{\beta - 1}
    \end{equation}
    
    If $X \sim \gammadistribution{\alpha}{\theta}$, $Y \sim \gammadistribution{\beta}{\theta}$, and they are independent, then
    \begin{equation}
        \frac{X}{X+Y} \sim \betadistribution{\alpha}{\beta}
    \end{equation}
\end{definition}



% Chi-square
\section{Chi-Square Distribution}

\begin{definition}[Chi-Square distribution]
    If $\set{x_i}$ are independent $\normaldistribution{0}{1}$. The sum of their square $\sum_{i=1}^k x_i^2$ is the \cindex{chi-square distribution} with $k$ degrees of freedom. So
    \begin{equation}
        \sum_{i=1}^k x_i^2 \sim \chisquaredistribution{k}
    \end{equation}
\end{definition}
































































































































































































































































































































































































































































